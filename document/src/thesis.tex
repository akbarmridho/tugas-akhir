%--------------------------------------------------------------------%
%
% Berkas utama templat LaTeX.
%
% author Petra Barus, Peb Ruswono Aryan, Faris Rizki Ekananda
%
%--------------------------------------------------------------------%
%
% Berkas ini berisi struktur utama dokumen LaTeX yang akan dibuat.
%
%--------------------------------------------------------------------%
\documentclass[bahasa, 12pt, a4paper, onecolumn, oneside, final]{report}

\hyphenpenalty=100000
\tolerance=10000

\input{config/if-itb-thesis.sty}
\input{config/hypenation-id.tex}

\makeatletter

\makeatother

\addbibresource{references.bib}

\begin{document}

\title{Pengoptimalan Sistem Tiket Acara Berskala Besar dengan Basis Data Relasional Terdistribusi dan Pengendalian Aliran Pemrosesan Pesanan}
\date{}
\author{
  Akbar Maulana Ridho \\
  NIM: 13521093
}
\newcommand\tanggalpengesahan{17 Juli 2025}

\pagenumbering{roman}
\setcounter{page}{1}

\clearpage
\pagestyle{empty}

\begin{center}
    \smallskip

    \Large \bfseries \MakeUppercase{\thetitle}
    \vfill

    % \Large Proposal Tugas Akhir
    % \vfill

    \Large Laporan Tugas Akhir
    \vfill

    \large Disusun sebagai syarat kelulusan tingkat sarjana
    \vfill

    \large Oleh

    \Large \theauthor

    \vfill
    \begin{figure}[htbp]
        \centering
        \includegraphics[width=0.15\textwidth]{cover-ganesha.jpg}
    \end{figure}
    \vfill

    \large
    \uppercase{
        Program Studi Teknik Informatika \\
        Sekolah Teknik Elektro \& Informatika \\
        Institut Teknologi Bandung
    }

    Agustus 2025

\end{center}

\clearpage

\input{chapters/approval-1}
\input{chapters/statement}

\pagestyle{plain}

\clearpage
\chapter*{ABSTRAK}
\addcontentsline{toc}{chapter}{ABSTRAK}
\begin{center}
  \center
  \begin{singlespace}
    \large\bfseries\MakeUppercase{\thetitle}

    \normalfont\normalsize
    Oleh:

    \bfseries \theauthor
  \end{singlespace}
\end{center}

\begin{singlespace}
  \small
  Penjualan tiket untuk acara populer seperti konser sering kali menghadapi tantangan teknis. Karakteristik utamanya adalah jumlah peminat yang jauh melebihi ketersediaan tiket, sehingga sistem harus menangani lonjakan kueri ketersediaan secara masif sambil menjaga stabilitas. Penelitian ini bertujuan untuk menganalisis dan membandingkan strategi arsitektur untuk mengoptimalkan sistem tersebut. Tiga pendekatan utama dieksplorasi: pertama, perbandingan kinerja basis data relasional terdistribusi (CitusData dan YugabyteDB) dengan kluster PostgreSQL sebagai tolok ukur. Kedua, pengoptimalan operasi baca ketersediaan tiket per area dengan menyimpan data agregat pada Redis. Ketiga, evaluasi skema pengendalian aliran yang terdiri dari penolakan permintaan lebih awal dan penggunaan antrean untuk membatasi pemrosesan pesanan secara serentak guna menjaga stabilitas basis data. Pada pengujian beban dengan 15 ribu pengguna virtual, kluster PostgreSQL menunjukkan kinerja paling baik, mencapai laju pemrosesan 466 rps dengan latensi P50 192-382 ms dan penggunaan sumber daya yang efisien. CitusData memberikan hasil yang dapat diterima, tetapi dengan latensi sekitar 2x lebih tinggi dan konsumsi sumber daya yang lebih besar. YugabyteDB menunjukkan kinerja yang tidak memadai, dengan laju pemrosesan setengah dari PostgreSQL, penggunaan CPU 2.4x dan memori 10x lebih banyak, serta tingkat kegagalan yang tinggi. Pengoptimalan operasi baca ketersediaan tiket menggunakan Redis terbukti sangat efektif, mampu melayani permintaan puncak hingga 1.700 rps dengan latensi rata-rata 2.5-4.5 ms, yang secara signifikan mengurangi beban pada basis data. Di sisi lain, skema pengendalian aliran pada varian terbukti bermanfaat. Strategi penolakan permintaan lebih awal berhasil menurunkan latensi untuk pesanan yang ditolak dari $>$1000 ms menjadi $<$100 ms. Implementasi antrean membuat latensi rata-rata mencapai 10 detik karena \textit{overhead} komunikasi dan beban pengujian tidak cukup tinggi untuk menunjukkan perbedaan kinerja.

  \textbf{\textit{Kata kunci: Sistem Tiket, Kluster PostgreSQL, CitusData, YugabyteDB, Pengendalian Aliran}}

\end{singlespace}
\clearpage
\clearpage
\chapter*{ABSTRACT}
\addcontentsline{toc}{chapter}{ABSTRACT}

\begin{center}
  \center
  \begin{singlespace}
    \large\bfseries\MakeUppercase{Large-Scale Event Ticket System Optimization with Distributed Relational Database and Transaction Processing Flow Control}

    \normalfont\normalsize
    By:

    \bfseries \theauthor
  \end{singlespace}
\end{center}


\begin{singlespace}
  \small
  The process of selling tickets for high-demand events like concerts does not always run smoothly. A unique characteristic of this problem is that the sale of a single seat is highly contested. Additionally, the system must serve a very large number of requests for ticket availability. On the other hand, system stability is also a crucial aspect to maintain the continuity of ticket sales. Therefore, this research aims to explore what optimizations are suitable for this system.

  The first optimization is to improve transaction rate scaling by using a distributed relational database, such as CitusData and YugabyteDB. The baseline for this system is a PostgreSQL cluster. Furthermore, the flow control of order processing is managed by rejecting requests that meet certain criteria. Orders are then processed according to the system's capacity.

  The test results show that.

  \textbf{\textit{Keywords: Ticket System, PostgreSQL Cluster, CitusData, YugabyteDB, Flow Control}}
\end{singlespace}
\clearpage

\clearpage
\chapter*{Kata Pengantar}
\addcontentsline{toc}{chapter}{KATA PENGANTAR}

Puji dan syukur penulis panjatkan kepada Tuhan Yang Maha Esa atas berkat dan rahmatnya, laporan tugas akhir yang berjudul "\thetitle" dapat diselesaikan dalam rangka memenuhi syarat kelulusan tingkat sarjana. Perlu diakui pengerjaan tugas akhir ini didukung oleh banyak pihak. Khususnya, penulis ingin mengucapkan terima kasih kepada:

\begin{enumerate}
  \item Bapak Achmad Imam Kistijantoro, S.T, M.Sc., Ph.D. selaku dosen pembimbing atas segala bentuk dukungan yang telah diberikan dan kesabarannya dalam membimbing penulis serta memberikan saran dalam pengerjaan tugas akhir.
  \item Bapak Dr.techn. Saiful Akbar, S.T., M.T. selaku dosen penguji atas segala masukan dan kritik yang telah diberikan terhadap tugas akhir penulis.
  \item Ibu Robithoh Annur, S.T., M.Eng., Ph.D. dan Tricya Esterina Widagdo, ST., M.Sc. selaku dosen koordinator tim tugas akhir atas usahanya mengingatkan mahasiswa program studi Teknik Informatika untuk mengerjakan tugas akhirnya.
  \item Seluruh dosen program studi Teknik Informatika ITB yang telah memberikan ilmu pengetahuan yang sangat berharga bagi penulis.
  \item Teman-teman SUDO 2021 yang telah menemani, memberikan inspirasi, serta dukungan mora kepada penulis dalam menempuh kuliah pada program studi Teknik Informatika.
  \item Teman-teman penulis khususnya anggota dari grup "lesgo wahoo!", anggota Laboratorium Sistem Terdistribusi, serta para penghuni sekretariat HMIF ITB Lantai 4 yang telah memberikan kenangan berharga, motivasi, hiburan, serta bantuan untuk segala situasi.
  \item Seluruh pihak lain yang tidak bisa disebutkan disini yang telah membantu dalam proses pengerjaan tugas akhir.
\end{enumerate}

Akhir kata, penulis mengucapkan terima kasih kepada semua pihak yang telah terlibat dalam pengerjaan tugas akhir ini. Penulis juga ingin menyampaikan mohon maaf apabila terdapat kesalahan maupun kekurangan dalam laporan tugas akhir ini. Penulis berharap semoga tugas akhir ini dapat bermanfaat bagi pembaca dan riset-riset kedepannya.

\begin{flushright}
  \vspace{0.5cm}
  Bandung, \tanggalpengesahan


  \vspace{1.5cm}

  Akbar Maulana Ridho
\end{flushright}

% \titleformat*{\section}{\centering\bfseries\Large\MakeUpperCase}
\titlespacing*{\chapter}{0pt}{0pt}{4pt}

% Setting judul toc, lot, lof, bib
\renewcommand{\contentsname}{DAFTAR ISI}
\renewcommand{\listfigurename}{DAFTAR GAMBAR}
\renewcommand{\listtablename}{DAFTAR TABEL}
\renewcommand{\bibname}{DAFTAR PUSTAKA}

% daftar isi, lampiran, gambar, table
\tableofcontents
\listofappendices
\listoffigures
\listoftables

\newpage

% \titleformat*{\section}{\bfseries\large}
\pagenumbering{arabic}

%----------------------------------------------------------------%
% Konfigurasi Bab
%----------------------------------------------------------------%
\setcounter{page}{1}
\renewcommand{\chaptername}{BAB}
\renewcommand{\thechapter}{\Roman{chapter}}
%----------------------------------------------------------------%

%----------------------------------------------------------------%
% Dafter Bab
% Untuk menambahkan daftar bab, buat berkas bab misalnya `chapter-6` di direktori `chapters`, dan masukkan ke sini.
%----------------------------------------------------------------%

% Adjust spacing around section titles
\titlespacing*{\section}{0pt}{12pt}{8pt}
\titlespacing*{\section}{0pt}{8pt}{6pt}
\titlespacing*{\section}{0pt}{6pt}{4pt}

\input{chapters/chapter-1}

\chapter{Kajian Pustaka}
\label{chapter:kajian-pustaka}

Bab ini akan diisi oleh kajian pustaka yang berkaitan dengan topik persoalan tugas akhir untuk memberikan informasi mengenai dasar teori dan studi yang dipakai. Bab ini diharapkan dapat membantu pembaca untuk lebih mengerti tentang penelitian tugas akhir ini.

\section{Teknik Penskalaan Basis Data Relasional}

Terdapat berbagai teknik umum yang digunakan untuk melakukan penskalaan pada basis data relasional. Berikut adalah beberapa di antaranya:

\subsection{Replikasi basis data}

Replikasi berarti menyimpan salinan data yang sama pada beberapa mesin yang berbeda dan terhubung melalui jaringan \parencite{dataIntensiveApplications}. Terdapat beberapa alasan mengapa hal ini lazim dilakukan, yaitu:

\begin{enumerate}
    \item Untuk menjaga data tetap dekat secara geografis kepada pengguna, sehingga latensi berkurang.
    \item Agar sistem dapat terus berjalan meski terjadi kegagalan pada sebagian sistem, sehingga \textit{availability} meningkat.
    \item Untuk melakukan \textit{scale out} banyaknya mesin yang bisa melayani \textit{read queries}, sehingga meningkatkan \textit{read throughput}.
\end{enumerate}

Salah satu pendekatan yang umum diimplementasikan pada basis data relasional seperti PostgreSQL adalah replikasi berbasiskan \textit{leader and follower}. Satu \textit{node} ditugaskan sebagai \textit{leader} yang menerima operasi \textit{read and write}, lalu setiap perubahan yang terjadi akan direplikasi oleh replika (\textit{follower}). Dengan pola seperti ini, umumnya operasi \textit{write} hanya dapat ditangani oleh \textit{leader} dan operasi \textit{read} dapat ditangani oleh semua \textit{node}.

\begin{figure}[ht]
    \centering
    \includegraphics[width=0.8\textwidth]{resources/chapter-2/leader-based-replication.png}
    \caption{\textit{Leader-based (master-slave) replication \parencite{dataIntensiveApplications}}}
    \label{fig:leader-based-replication}
\end{figure}

Selain itu, proses replikasi juga terbagi menjadi dua, yaitu \textit{synchronous replication} dan \textit{asynchronous replication}. Pada \textit{synchronous replication}, data yang akan ditulis juga harus sudah ditulis oleh semua (atau mayoritas) replika sebelum dapat di-\textit{acknowledge}. Pada \textit{asynchronous replication}, data akan ditulis terlebih dahulu pada \textit{leader} lalu perubahannya dipropagasikan kepada \textit{replika}. Setiap pendekatan ini memiliki \textit{tradeoff} tersendiri. \textit{Synchronous replication} menjamin mayoritas \textit{node} memiliki data paling terbaru, tetapi latensi pada proses penulisan akan meningkat, sedangkan pada \textit{asynchronous replication} latensi penulisan jauh lebih kecil, tetapi data pada \textit{replica} menjadi \textit{eventually consistent}. Kedua mode replikasi ini didukung oleh PostgreSQL.

\subsection{Pemartisian Basis Data}

Pemartisian (\textit{sharding}) merupakan sebuah cara untuk membagi basis data yang besar menjadi bagian-bagian yang lebih kecil \parencite{dataIntensiveApplications}. Beberapa partisi dapat disimpan pada \textit{node} yang berbeda, sehingga dataset yang besar dapat didistribusikan pada beberapa media penyimpanan. Selain itu, beban kueri juga dapat didistribusikan pada beberapa mesin yang berbeda.

Pemartisian dapat dibagi menjadi dua, yaitu pemartisian horizontal yang membagi data berdasarkan baris dan pemartisian vertikal yang membagi data berdasarkan kolom.

Tujuan dari pemartisian data adalah untuk menyebarkan data dan beban secara adil kepada \textit{node} yang berbeda. Untuk itu, pendekatan untuk membagi data juga menjadi hal yang penting dalam partisi data. Berikut adalah beberapa pendekatan yang umum digunakan:

\begin{enumerate}
    \item Partisi berdasarkan kunci, seperti judul buku yang berawalan A-E, F-J, dan seterusnya. Meskipun begitu, perlu diiperhatikan apakah distribusi kunci tersebar secara merata. Gambar \ref{fig:partition-by-key-range} menunjukkan pemartisian berdasarkan rentang kunci. Skema ini berpotensi menciptakan distribusi beban tidak merata bila data terkonsentrasi pada rentang tertentu.
          \begin{figure}[H]
              \centering
              \includegraphics[width=0.8\textwidth]{resources/chapter-2/partition-by-key-range.png}
              \caption{Pemartisian berdasarkan rentang kunci \parencite{dataIntensiveApplications}}
              \label{fig:partition-by-key-range}
          \end{figure}

    \item Partisi berdasarkan hash kunci. Fungsi hash yang baik dapat mengubah distribusi kunci yang tidak simetris menjadi merata. Gambar \ref{fig:partition-by-hash} menggambarkan pemartisian berbasis hash, yang menyebarkan data lebih merata di seluruh \textit{node} sehingga mengurangi risiko \textit{hotspot}.
          \begin{figure}[H]
              \centering
              \includegraphics[width=0.8\textwidth]{resources/chapter-2/partition-by-hash.png}
              \caption{Pemartisian berdasarkan hash \parencite{dataIntensiveApplications}}
              \label{fig:partition-by-hash}
          \end{figure}

\end{enumerate}

\subsection{Citus: Distributed PostgreSQL}

Citus merupakan \textit{distributed database engine} untuk PostgreSQL yang menangani kebutuhan skalabilitas pada ekosistem PostgreSQL \parencite{citus}. Sebagai sebuah \textit{extension}, Citus menjaga \textit{compatibility} dengan PostgreSQL, sehingga dapat digunakan dengan PostgreSQL versi terbaru.

Sebuah \textit{cluster} memiliki banyak \textit{nodes} yang terspesialisasi menjadi \textit{coordinator} dan \textit{worker}. Aplikasi mengirimkan \textit{query} kepada \textit{coordinator} dan diteruskan kepada \textit{worker} terkait atau menggabungkan hasil. Pendekatan ini memungkinkan setiap \textit{node} mampu memproses \textit{write} request sehingga pemrosesan bisa dilakukan secara parallel.

Selain itu, berikut adalah fitur lain \textit{extension} Citus:

\begin{enumerate}
    \item \textit{Distributed tables} dibagi ke seluruh kluster node untuk menggabungkan \textit{machine resources}.
    \item \textit{Referenced tables} direplikasi ke seluruh node untuk kebutuhan \textit{join} dan \textit{foreign key}, sehingga kinerja \textit{read} semakin baik.
    \item \textit{Distributed query engine routes} dan memparallelkan operasi pada tabel terdistribusi ke seluruh kluster.
\end{enumerate}

Citus memiliki dua model \textit{sharding}, yaitu \textit{row-based sharding} dan \textit{schema based sharding}.

\begin{figure}[ht]
    \centering
    \includegraphics[width=0.8\textwidth]{resources/chapter-2/row-vs-schema-sharding.png}
    \caption{\textit{Schema-based sharding vs row-based sharding \parencite{schemaBasedSharding}}}
    \label{fig:row-vs-schema-sharding}
\end{figure}



\section{\textit{Event-Driven Architecture}}

\textit{Event-driven architecture} merupakan paradigma arsitektur perangkat lunak yang berkaitan dengan produksi dan deteksi \textit{event}. Arsitektur ini mudah dievolusikan dan menawarkan toleransi kegagalan (\textit{fault tolerance}), kinerja yang baik, dan pemskalaan yang baik. Meskipun begitu, arsitektur ini kompleks dan sulit untuk diuji. Arsitektur ini cocok untuk kasus yang kompleks dan dinamis \parencite{softwareArchitecture}.

Arsitektur ini terdiri atas tiga komponen, yaitu \textit{event producer}, \textit{event router}, dan \textit{event consumer}. Sebuah produsen mengirimkan \textit{event} ke \textit{router}, lalu disaring dan dikirimkan kepada konsumen.

\subsection{Redpanda}

Redpanda merupakan \textit{event streaming platform}. Platform ini menyediakan infrastruktur untuk \textit{streaming real-time data}. Produsen mengirimkan data berupa \textit{event} ke Redpanda, kemudian Redpanda menyimpan \textit{event} tersebut lalu mengaturnya ke dalam sebuah topik. Topik ini merupakan log \textit{event} yang dapat diputar ulang. Konsumen mengonsumsi \textit{event} pada topik Redpanda secara asinkron \parencite{redpandaIntro}.

\begin{figure}[htbp]
    \centering
    \includegraphics[width=0.8\textwidth]{resources/chapter-2/redpanda.png}
    \caption{Apa itu Redpanda? \parencite{whatIsRedpanda}}
    \label{fig:what-is-redpanda}
\end{figure}

Redpanda merupakan \textit{event streaming platform} alternatif dari Apache Kafka. Selain itu, platform ini menawarkan kompatibilitas API yang sama dengan Kafka sehingga memudahkan migrasi penggunanya. Meskipun begitu, terdapat beberapa perbedaan antara Redpanda dengan Apache Kafka.

Perbedaan pertama adalah algoritma konsensus yang digunakan. Apache Kafka menggunakan ZooKeeper (versi lama) sedangkan Redpanda menggunakan Raft. Meskipun begitu, versi terbaru Kafka sudah menggunakan algoritma konsensus Kraft yang merupakan varian dari Raft dengan perbedaan pada mekanisme replikasi log \parencite{raftKraft}.

Selain itu, Redpanda berfokus pada pengoptimalan kinerja yang lebih baik dibandingkan dengan Apache Kafka. Redpanda ditulis dalam bahasa C++, sedangkan Apache Kafka ditulis dalam bahasa Java dan berjalan pada JVM. Dalam hal ini, Redpanda menggunakan bahasa sistem sehingga minim \textit{overhead}.

Berikut adalah contoh pengoptimalan yang dilakukan pada Redpanda: \textit{Direct Memory Access (DMA)} untuk I/O, distribusi pemrosesan \textit{interrupt request} (IRQ) pada beberapa core CPU, penggunaan model \textit{thread per core}, dan pengoptimalan lainnya. Penggunaan model \textit{thread per core} memungkinkan Redpanda untuk menjalankan \textit{thread} aplikasi pada inti CPU yang sama sehingga \textit{context switching} dan \textit{blocking} dapat dihindari \parencite{redpandaArchitecture}.

\subsection{\textit{Change Data Capture}}

Menurut \cite{dataIntensiveApplications}, \textit{change data capture (CDC)} merupakan sebuah proses yang mengobservasi setiap perubahan pada data yang ditulis ke dalam basis data dan mengekstraknya ke dalam bentuk yang bisa direplikasi oleh sistem lain. Sebagai contoh, perubahan pada database bisa di-\textit{capture} lalu diterapkan pada \textit{search index} untuk menyamakan data pada basis data. Apabila \textit{log} diaplikasikan dalam urutan yang sama, data pada \textit{search index} dan basis data bisa dipastikan sama.

\begin{figure}[ht]
    \centering
    \includegraphics[width=0.8\textwidth]{resources/chapter-2/cdc.png}
    \caption{\textit{CDC Illustration \parencite{dataIntensiveApplications}}}
    \label{fig:cdc-illustration}
\end{figure}

PostgreSQL juga mendukung CDC dengan istilah \textit{logical replication}. Mekanisme ini menggunakan model \textit{publish} dan \textit{subscribe}. PostgreSQL yang mengirimkan \textit{log} bertindak sebagai \textit{publisher}, lalu terdapat \textit{subscriber} lain yang mengonsumsi \textit{log} yang dipublikasikan. \textit{Subscriber} ini bisa berupa replika PostgreSQL lagi atau aplikasi lainnya \parencite{pgLogicalReplication}. PostgreSQL mendukung dua mode operasi untuk replikasi, yaitu replikasi secara \textit{asynchronous} dan \textit{synchronous} \parencite{insideLogicalReplication}. Pada mode \textit{synchronous}, \textit{subscriber} harus merespons terlebih dahulu terhadap perubahan data sebelum PostgreSQL dapat melakukan \textit{commit}.

\begin{figure}[ht]
    \centering
    \includegraphics[width=0.8\textwidth]{resources/chapter-2/postgres-logical-replication.png}
    \caption{\textit{Logical Replication Architecture \parencite{insideLogicalReplication}}}
    \label{fig:logical-replication-architecture}
\end{figure}



\subsection{\textit{Event Sourcing}}

Mirip seperti CDC, \textit{event sourcing} juga menyimpan setiap perubahan pada keadaan aplikasi sebagai \textit{log of events}. Perbedaan terbesarnya terletak pada level abstraksinya. Pada CDC, aplikasi menggunakan basis data \textit{in a mutable way} dan dapat memperbarui atau menghapus catatan sesuka hati. Aplikasi yang menulis pada basis data tidak harus menyadari bahwa terdapat CDC. Berbeda dengan \textit{event sourcing}, logika aplikasi dibangun secara eksplisit di atas asumsi \textit{immutable event} yang ditulis pada \textit{event log}. Pada kasus ini, \textit{event} bersifar \textit{append only} (hanya bisa dibaca dan ditambahkan, tidak bisa diperbarui atau dihapus). Singkatnya, \textit{event sourcing} didesain agar bisa merefleksikan hal yang terjadi pada level aplikasi dan bukan pada \textit{low-level state changes} \parencite{dataIntensiveApplications}.

Sebagai contoh, \textit{event} yang berisi "seorang murid membatalkan keikutsertaannya dalam suatu kelas" menyampaikan maksud dari sebuah aksi (\textit{event sourcing}), sedangkan \textit{event} yang berisi "sebuah entri dihapus dari tabel kepesertaan mata kuliah dan satu alasan pembatalan ditambahkan pada tabel umpan balik mahasiswa" tidak memberikan informasi yang jelas atas hal apa yang terjadi (CDC).

\section{Pemrosesan \textit{Stream}}

Secara umum, \textit{stream} merujuk pada data yang tersedia secara inkremental dari waktu ke waktu. Konsep ini dipakai di berbagai tempat, seperti stdin dan stdout Unix, koneksi TCP, dan lain-lain \parencite{dataIntensiveApplications}.

\textit{Stream processing} merupakan paradigma pemrograman yang memandang \textit{stream} atau urutan \textit{event} sebagai objek masukan dan luaran utama dari komputasi. \cite{streaming101} menyatakan bahwa \textit{stream processing} merupakan tipe mersin pemrosesan data yang didesain dengan mempertimbangkan data yang tidak terbatas.

\cite{streamProcessingComparison} menyatakan bahwa terdapat karakteristik penting pada \textit{stream processing}, yaitu:

\begin{enumerate}
    \item \textit{Delivery guarantees}. Setiap informasi yang masuk harus dijamin akan diproses oleh \textit{streaming engine}.
    \item Toleransi kegagalan (\textit{fault tolerance}). Ketika terjadi kegagalan, \textit{streaming engine} harus mampu melakukan pemulihan dan memulai ulang dari titik yang ditinggalkan.
    \item \textit{State management}. \textit{Streaming engine} harus memiliki mekanisme untuk menyimpan dan memperbarui informasi \textit{state}.
    \item Memiliki kinerja yang baik dari sisi latensi, \textit{throughput}, dan skalabilitas.
    \item Memiliki fitur yang lebih canggih, seperti \textit{event time processing}, \textit{watermarks}, \textit{windowing}, dan lain-lain.
\end{enumerate}

Selain itu, karakteristik \textit{stream processing} juga bisa dibagi menjadi dua jenis, yaitu:

\begin{enumerate}
    \item \textit{Native streaming}. \textit{Stream processing} jenis ini akan langsung memproses data yang diterima secepat mungkin. Contoh \textit{stream processing} tipe ini adalah Apache Storm, Apache Flink, Apache Kafka Streams, dan Apache Samza.
    \item \textit{Micro-batching}. \textit{Stream processing} jenis ini memproses data setiap beberapa detik atau milidetik sekali sehingga data diproses dalam setiap kelompok kecil dengan sedikit keterlambatan. Contoh \textit{stream processing} tipe ini adalah Apache Spark Streaming dan Apache Storm-Trident.
\end{enumerate}

\subsection{RisingWave}

RisingWave merupakan \textit{cloud-native streaming database}. Setelah menghubungkan sumber \textit{stream}, pengguna dapat membuat kueri analisis dengan mendefinisikan \textit{materialized view}, yang diperbarui secara inkremental pada RisingWave \textit{streaming engine} \parencite{risingwave}.

Berikut adalah keuntungan RisingWave:

\begin{enumerate}
    \item Mudah dipelajari karena merupakan ekstensi dari sintaks PostgreSQL.
    \item Mudah dioperasikan dan memiliki kebutuhan sumber daya yang lebih rendah karena ditulis dalam bahasa sistem Rust.
    \item Mendukung berbagai sumber data dan mampu mengirimkan (\textit{sink}) data ke dalam berbagai sumber, seperti mengambil data dari Apache Kafka lalu hasilnya dikirim ke ClickHouse. RisingWave mendukung integrasi dengan PostgreSQL CDC dan Apache Kafka sebagai sumber (\textit{source}) dan tujuan data (\textit{sink}).
    \item Menjamin konsistensi pada \textit{materialized view} dengan menggunakan \textit{snapshot}.
\end{enumerate}

\begin{figure}[ht]
    \centering
    \includegraphics[width=0.8\textwidth]{resources/chapter-2/risingwave.png}
    \caption{Arsitektur RisingWave \parencite{risingwave}}
    \label{fig:risingwave-architecture}
\end{figure}

Node komputasi pada RisingWave terdiri atas \textit{batch engine} dan \textit{streaming engine}. \textit{Batch engine} meliputi \textit{query execution engine} dan \textit{exchange service} untuk menukar data antar node komputasi. \textit{Streaming engine} dibangun atas model aktor pada pemrograman konkuren. Mesin ini berinteraksi langsung dengan \textit{frontend} dan melayani \textit{stream data}. Selain itu, terdapat \textit{meta service} yang berperan sebagai layanan sentral untuk menyimpan metadata seperti keadaan kluster, katalog sistem, keanggotaan kluster, dan lain-lain \parencite{risingwave}.

\chapter{Analisis Masalah dan Rancangan Solusi}

\section{Analisis Sistem}
\label{apx:analisis-kebutuhan}

Sebelum membahas pengoptimalan sistem tiket, perlu dibahas sistem tiket seperti apa yang ingin dioptimalkan pada penelitian ini. Oleh karena itu, berikut adalah spesifikasi sistem tiket yang dimaksud. Sistem ini dibuat semirip mungkin sehingga menyerupai sistem tiket yang ada.

\section{Kebutuhan Fungsional dan Non-Fungsional Sistem Tiket}

ID mengikuti format \((T|P|U)(F|N)-XX\). Huruf pertama menunjukkan kebutuhan fungsional atau non-fungsional untuk layanan tersebut. T berarti tiket, P berarti pembayaran, dan U berarti pengguna. Huruf kedua menunjukkan kebutuhan fungsional atau non-fungsional. F berarti fungsional dan N berarti non-fungsional. Contoh: TF-01 berarti kebutuhan fungsional nomor 01 untuk layanan tiket.

\subsection{Kebutuhan Fungsional}

Berikut adalah kebutuhan fungsional dan non-fungsional sistem tiket yang menjadi cakupan pada tugas akhir ini:

\begingroup
\footnotesize
\begin{longtable}{|l|p{0.4\textwidth}|p{0.4\textwidth}|}
    \caption{Kebutuhan Fungsional Sistem Tiket}                                                                                                                                                                                                                                                                                                                                                                                                                                          \\
    \hline
    \textbf{ID} & \textbf{Kebutuhan}                                                                                             & \textbf{Deskripsi}                                                                                                                                                                                                                                                                                                                                    \\
    \hline
    \endfirsthead

    \multicolumn{3}{|l|}{\tablename\ \thetable\ -- \textit{Lanjutan dari halaman sebelumnya}}                                                                                                                                                                                                                                                                                                                                                                                            \\
    \hline
    \textbf{ID} & \textbf{Kebutuhan}                                                                                             & \textbf{Deskripsi}                                                                                                                                                                                                                                                                                                                                    \\
    \hline
    \endhead

    \hline
    \multicolumn{3}{|r|}{\textit{Dilanjutkan ke halaman berikutnya}}                                                                                                                                                                                                                                                                                                                                                                                                                     \\
    \endfoot

    \hline
    \endlastfoot

    \hline
    TF-01       & Sistem dapat melayani permintaan ketersediaan acara.                                                           &                                                                                                                                                                                                                                                                                                                                                       \\
    \hline
    \hline
    TF-02       & Sistem dapat melayani permintaan ketersediaan tiket untuk suatu acara .                                        & Ketersediaan tiket dibagi berdasarkan kategori. Data ketersediaan bisa lebih granular (menampilkan ketersediaan per kursi) atau hanya menampilkan jumlah ketersediaan untuk suatu kategori. Perilaku ini dapat diatur per kategori tiket.                                                                                                             \\
    \hline
    \hline
    TF-03       & Sistem dapat melayani permintaan pemesanan tiket untuk suatu acara.                                            & Seorang pengguna dapat memesan hingga empat tiket dalam kategori yang sama sekaligus. Pengguna dapat memesan berdasarkan kursi atau area tiket, bergantung pada pengaturan kategori tiket. Pemesanan akan dibatalkan secara otomatis ketika sudah melewati tenggat waktu pembayaran.                                                                  \\
    \hline
    \hline
    TF-04       & Pengguna dapat melihat status tiket yang pernah dipesan.                                                       &                                                                                                                                                                                                                                                                                                                                                       \\
    \hline
    TF-05       & Sistem dapat menangani penjualan tiket untuk lebih dari satu acara dalam satu waktu.                           &                                                                                                                                                                                                                                                                                                                                                       \\
    \hline
    \hline
    UF-01       & Sistem dapat melayani registrasi pengguna.                                                                     &                                                                                                                                                                                                                                                                                                                                                       \\
    \hline
    \hline
    UF-02       & Sistem menyediakan mekanisme \textit{login} bagi pengguna.                                                     &                                                                                                                                                                                                                                                                                                                                                       \\
    \hline
    \hline
    UF-03       & Sistem menyediakan \textit{endpoint} bagi layanan lain untuk memperoleh informasi pengguna yang terotentikasi. &                                                                                                                                                                                                                                                                                                                                                       \\
    \hline
    \hline
    PF-01       & Sistem dapat membuat tagihan pembayaran dan pranala pembayaran.                                                & \textit{Mock service}. Pembayaran dilakukan dengan mengirimkan \textit{request} kepada pranala yang diberikan dengan parameter sukses atau gagal. Selain itu, terdapat tenggat waktu pembayaran yang ditentukan berdasarkan parameter pada saat pembuatan tagihan. Pembayaran akan otomatis gagal apabila tidak dipenuhi hingga batas waktu tersebut. \\
    \hline
    \hline
    PF-02       & Sistem memanggil \textit{webhook} yang telah ditentukan ketika terdapat pembayaran yang berhasil atau gagal.   & \textit{Mock service}. Pembayaran dilakukan dengan mengirimkan \textit{request} kepada pranala yang diberikan dengan parameter sukses atau gagal.                                                                                                                                                                                                     \\
    \hline
    \hline
    PF-03       & Sistem menyediakan \textit{endpoint} untuk menampilkan detail tagihan pembayaran.                              &                                                                                                                                                                                                                                                                                                                                                       \\
    \hline
\end{longtable}
\endgroup

Pemesanan tiket dibagi menjadi dua tahap, yaitu fase \textit{hold seating} dan fase pembayaran. Fase \textit{hold seating} akan me-\textit{reserve} tiket sampai batas waktu tertentu hingga pengguna menyelesaikan fase pembayaran. Saat pembayaran berhasil, pembelian tiket baru akan dianggap sukses. Daftar acara dan ketersediaan awal tiket merupakan data yang diisi dari awal, sehingga fitur manajemen acara dan tiket tidak diimplementasikan.

\subsection{Kebutuhan Non-Fungsional}

Berikut adalah kebutuhan non-fungsional yang menjadi cakupan pada tugas akhir ini:

\begingroup
\footnotesize
\begin{longtable}{|l|p{0.4\textwidth}|p{0.4\textwidth}|}
    \caption{Kebutuhan Non-Fungsional Sistem Tiket}                                                                                                                                                               \\
    \hline
    \textbf{ID} & \textbf{Parameter}      & \textbf{Kebutuhan}                                                                                                                                                    \\
    \hline
    \endfirsthead

    \multicolumn{3}{|l|}{\tablename\ \thetable\ -- \textit{Lanjutan dari halaman sebelumnya}}                                                                                                                     \\
    \hline
    \textbf{ID} & \textbf{Parameter}      & \textbf{Kebutuhan}                                                                                                                                                    \\
    \hline
    \endhead

    \hline
    \multicolumn{3}{|r|}{\textit{Dilanjutkan ke halaman berikutnya}}                                                                                                                                              \\
    \endfoot

    \hline
    \endlastfoot

    \hline
    TN-01       & Konsistensi             & Sistem harus memastikan tidak terjadi \textit{double booking} pada saat pemesanan tiket.                                                                              \\
    \hline
    \hline
    TN-02       & \textit{Data Freshness} & Sistem harus selalu mengembalikan data paling terbaru.                                                                                                                \\
    \hline
    \hline
    UN-01       & Otentisitas Pengguna    & Layanan lain dapat memverifikasi otentikasi token pengguna tanpa harus memanggil layanan pengguna, seperti dengan penggunaan token JWT dengan \textit{shared secret}. \\
    \hline
\end{longtable}
\endgroup


\section{Komponen Sistem Tiket}

Berdasarkan studi yang sudah dibahas sebelumnya dan berdasarkan fokus yang ingin dibahas pada penelitian ini, berikut adalah komponen sistem yang menjadi bahasan dari penelitian ini:

\begin{enumerate}
    \item Layanan \textit{backend} utama yang memproses setiap permintaan yang berkaitan dengan pemesanan tiket. Layanan ini dapat dipecah menjadi beberapa layanan bergantung pada desain setiap arsitektur solusi.
    \item Layanan otentikasi. Layanan ini menggunakan solusi otentikasi yang sudah \textit{established}, seperti Ory Kratos. Otentikasi akan menggunakan JWT token sehingga validasi pengguna tidak harus bergantung pada layanan ini.
    \item Layanan gerbang pembayaran. Implementasi dari layanan ini akan berupa \textit{mock service}. Layanan ini akan disimulasikan sebagai gerbang pembayaran eksternal.
    \item Terdapat satu basis data utama sebagai sumber kebenaran utama. Basis data relasional yang akan digunakan adalah PostgreSQL.
\end{enumerate}

Komponen sistem yang lebih spesifik akan bergantung pada desain setiap arsitektur solusi.

\section{\textit{Entity Relationship Diagram}}

Berikut adalah gambaran kasar dari diagram relasi entitas pada sistem tiket yang menjadi bahasan pada penelitian ini:

\begin{figure}[htbp]
    \centering
    \includegraphics[width=1\textwidth]{resources/appendix/erd.png}
    \caption{ERD Sistem Tiket}
    \label{fig:ticket-system-erd-proposal}
\end{figure}

Berikut adalah pembagian entitas berdasarkan layanan:

\begingroup
\begin{longtable}{|p{0.3\textwidth}|p{0.7\textwidth}|}
    \caption{Kebutuhan Non-Fungsional Sistem Tiket}                                                                                \\
    \hline
    \textbf{Layanan} & \textbf{Entitas}                                                                                            \\
    \hline
    \endfirsthead

    \multicolumn{2}{|l|}{\tablename\ \thetable\ -- \textit{Lanjutan dari halaman sebelumnya}}                                      \\
    \hline
    \textbf{Layanan} & \textbf{Entitas}                                                                                            \\
    \hline
    \endhead

    \hline
    \multicolumn{2}{|r|}{\textit{Dilanjutkan ke halaman berikutnya}}                                                               \\
    \endfoot

    \hline
    \endlastfoot

    \hline
    Tiket            & Events, TicketCategory, Areas, Seats, TicketSale, \linebreak TicketPackage, Orders, OrderItem, IssuedTicket \\
    \hline
    \hline
    Pengguna         & Users                                                                                                       \\
    \hline
    \hline
    Pembayaran       & Invoice                                                                                                     \\
    \hline
\end{longtable}
\endgroup


\section{Analisis Masalah}

Penjualan tiket acara dengan tingkat peminat serupa dengan penjualan tiket Taylor Swift dan Coldplay tentu akan terjadi lagi di masa mendatang. Meskipun begitu, tidak semua penyedia layanan terbiasa menangani beban pada skala ini. Kebanyakan penyedia layanan menggunakan solusi antrean virtual yang membatasi jumlah pengguna yang mengakses situs secara bersamaan. Pendekatan ini memang membantu meringankan beban sistem dan menjaga stabilitas. Meskipun begitu, banyak pengguna yang harus menunggu lama untuk bisa mengakses platform.

Di sisi lain, belum banyak studi yang membahas skalabilitas sistem dengan kasus seperti ini. \cite{microservicesEventDriven} membahas desain arsitektur yang tahan kegagalan dan tidak membahas aspek skalabilitas. \cite{backendForTicketing} juga tidak membahas arsitektur sistem dari sisi skalabilitas. Berdasarkan pertimbangan di atas, diperlukan desain arsitektur yang optimal dan mampu menangani beban seperti ini. Solusi ini tidak serta merta mengganti solusi antrean virtual. Dengan adanya arsitektur yang optimal, jumlah pengguna yang bisa dilayani dalam satu waktu dapat meningkat dan proses penjualan menjadi lebih cepat tanpa kendala.

Tantangan utama dalam sistem ini dapat dipecah menjadi hal-hal berikut:

\begin{enumerate}
    \item Laju pemrosesan transaksi (pemrosesan pemesanan tiket). Pada kasus tiket, akan ada banyak pengguna yang ingin memesan tiket secara bersamaan. Oleh karena itu, laju pemrosesan yang dapat diberikan oleh basis data harus ditingkatkan. Basis data relasional saat ini memang memungkinkan \textit{scaling out} dengan menambah jumlah instans, tetapi instans yang dapat menulis data tetap ada satu sehingga batas laju penulisan hanya dapat ditingkatkan dengan melakukan penskalaan vertikal. Di sisi lain, basis data relasional terus berkembang dan sudah ada berbagai solusi yang memungkinkan basis data relasional mendukung banyak penulis. Peningkatan penskalaan ini memungkinkan peningkatan jumlah pengguna yang dapat dilayani dalam satu waktu.
    \item Pemrosesan kueri baca. Terdapat beberapa kueri baca yang dipanggil dengan jumlah yang banyak dengan data yang selalu berubah. Kueri tersebut adalah kueri permintaan baca ketersediaan tiket, baik secara agregat atau pun satuan. Data ini selalu berubah karena berkaitan dengan ketersediaan tiket yang terus berubah seiring dengan keberjalanan proses penjualan. Oleh karena itu, perlu pengoptimalan khusus yang tidak dapat diselesaikan dengan metode tembolok pada umumnya.
    \item Integritas data dan kondisi pacu. Sifat penjualan tiket yang "siapa cepat, dia dapat" menciptakan perebutan sumber daya yang ekstrem. Beberapa pengguna mungkin mencoba memesan kategori tiket yang sama pada saat yang sama. Oleh karena itu, sistem harus dapat menangani kondisi ini dengan menjamin bahwa tidak akan terjadi pemesanan ganda untuk satu unit tiket yang sama.
    \item Pengendalian aliran untuk stabilitas sistem. Tanpa mekanisme perlindungan, lonjakan permintaan dapat menggangu keberjalanan pemrosesan pemesanan. Beberapa permintaan pada akhirnya akan gagal karena tiket sudah dipesan oleh pengguna lain. Oleh karena itu, diperlukan skema pengendalian aliran diperlukan untuk menjaga keberjalanan dan kestabilan pemrosesan pesanan.
\end{enumerate}


\section{Analisis Solusi}

\subsection{Peningkatan Laju Pemrosesan Pemesanan Tiket}
Tantangan fundamental dalam sistem tiket berskala besar adalah keterbatasan laju pemrosesan transaksi tulis (\textit{write throughput}). Basis data relasional tradisional, seperti PostgreSQL dalam konfigurasi \textit{primary-replica}, menghadapi \textit{bottleneck} inheren: semua operasi tulis harus diproses oleh satu \textit{node primary}. Peningkatan kinerjanya terbatas pada penskalaan vertikal (menambah CPU, memori, dan media penyimpanan pada satu mesin), sebuah pendekatan yang mahal dan memiliki batas fisik. Oleh karena itu, arsitektur ini secara fundamental membatasi jumlah pesanan tiket yang dapat diproses secara bersamaan.

Meskipun basis data NoSQL seperti MongoDB dan Cassandra menawarkan skalabilitas tulis horizontal yang superior, solusi ini umumnya mengorbankan jaminan konsistensi transaksional kuat (ACID) yang esensial untuk sistem tiket. Dalam skenario perebutan tiket, kemampuan untuk mengeksekusi operasi secara atomik—seperti mengunci dan memesan sebuah kursi—adalah kebutuhan esensial untuk mencegah \textit{double-booking}.

Oleh karena itu, fokus solusi adalah pada ranah basis data relasional yang mampu diskalakan secara horizontal (\textit{scale-out}), atau yang dikenal sebagai Distributed SQL. Pendekatan ini menjanjikan skalabilitas tulis dari dunia NoSQL sambil mempertahankan jaminan transaksional dari dunia relasional. Dalam praktiknya, terdapat dua ekosistem utama yang dominan, yaitu ekosistem MySQL (dengan solusi seperti Vitess dan CockroachDB) dan ekosistem PostgreSQL (dengan CitusData dan YugabyteDB).

\subsubsection{Perbandingan Ekosistem PostgreSQL dan MySQL}

Kedua ekosistem menawarkan solusi yang matang dan valid untuk skalabilitas. Vitess menyediakan skalabilitas horizontal untuk MySQL, serupa dengan peran CitusData untuk PostgreSQL \parencite{vitess}. CockroachDB, yang kompatibel dengan antarmuka MySQL, merupakan basis data terdistribusi \textit{native}, sejajar dengan YugabyteDB di dunia PostgreSQL \parencite{cockroachDB}.

Meskipun demikian, ekosistem PostgreSQL dipilih untuk tugas akhir ini karena beberapa alasan strategis yang berfokus pada validitas perbandingan dan efisiensi pengembangan:

\begin{enumerate}
    \item \textbf{Konsistensi Lingkungan Pengujian:} Dengan memilih basis data yang semuanya menggunakan antarmuka PostgreSQL (PostgreSQL, CitusData, dan YugabyteDB), perbandingan kinerja menjadi lebih adil. Semua varian sistem menggunakan dialek SQL, \textit{driver}, dan \textit{tooling} yang sama. Hal ini meminimalkan variabel pembaur (\textit{confounding variables}) sehingga perbedaan kinerja yang teramati dapat lebih akurat diatribusikan pada perbedaan arsitektur fundamental basis data, bukan pada perbedaan implementasi \textit{query planner} atau protokol klien.
    \item \textbf{Kekuatan Ekstensi dan Komunitas:} PostgreSQL dikenal dengan arsitekturnya yang sangat ekstensibel, yang memungkinkan inovasi seperti CitusData untuk berkembang secara organik. Reputasinya yang kuat dalam hal kepatuhan terhadap standar SQL dan fitur-fitur canggih menjadikannya fondasi yang solid untuk eksplorasi arsitektur.
    \item \textbf{Efisiensi Pengembangan:} Keakraban penulis dengan ekosistem PostgreSQL memungkinkan fokus yang lebih mendalam pada perancangan arsitektur dan analisis hasil pengujian, alih-alih menghabiskan waktu signifikan untuk mempelajari dan beradaptasi dengan dialek SQL dan perilaku operasional ekosistem yang berbeda.
\end{enumerate}

\subsubsection{Strategi Arsitektur yang Diuji}

\paragraph{Tolok Ukur: Arsitektur Monolitik Tradisional (PostgreSQL)}

Pendekatan pertama dan paling fundamental adalah menggunakan kluster PostgreSQL standar dengan konfigurasi \textit{primary-replica}. Arsitektur ini berfungsi sebagai tolok ukur (\textit{baseline}) yang esensial dalam penelitian ini. Kinerjanya merepresentasikan solusi yang matang, andal, dan umum digunakan di industri. Dengan menetapkannya sebagai dasar perbandingan, efektivitas dari dua strategi lainnya dalam mengatasi \textit{bottleneck} penulisan dapat diukur secara kuantitatif.

\paragraph{Strategi 1: Ekstensi Terdistribusi pada Basis Data Monolitik}

Pendekatan ini memanfaatkan basis data yang matang dan teruji seperti PostgreSQL dan menambahkan kemampuan distribusi di atasnya. CitusData dipilih untuk merepresentasikan strategi ini. Sebagai ekstensi, CitusData mengubah kluster PostgreSQL menjadi basis data terdistribusi dengan arsitektur \textit{coordinator-worker} \parencite{citus}.

Kemampuan CitusData untuk melakukan pemartisian (\textit{sharding}) tabel secara horizontal di seluruh \textit{node worker} sangat relevan. Hipotesisnya adalah dengan mempartisi tabel krusial seperti TicketSeats dan Orders berdasarkan kolom seperti ticket\_area\_i, beban transaksi pemesanan dapat didistribusikan. Perebutan tiket untuk "Area A" dapat diproses oleh \textit{Worker 1}, sementara perebutan untuk "Area B" diproses secara paralel oleh \textit{Worker 2}. Secara teoretis, pendekatan ini memungkinkan sistem memproses pesanan dari berbagai kategori tiket secara bersamaan, sehingga secara signifikan meningkatkan laju pemrosesan pesanan secara keseluruhan dibandingkan dengan satu \textit{node primary} tunggal.

\paragraph{Strategi 2: Basis Data Relasional Terdistribusi Secara Natif}

Pendekatan kedua adalah menggunakan basis data yang dirancang dari awal (\textit{from the ground up}) sebagai sistem terdistribusi. Basis data ini tidak memiliki ketergantungan pada arsitektur monolitik dan dibangun dengan konsensus terdistribusi sebagai fondasinya. YugabyteDB dipilih sebagai representasi dari arsitektur ini. YugabyteDB menggunakan konsensus Raft dan arsitektur \textit{multi-writer}, di mana setiap \textit{node} dalam kluster mampu memproses operasi tulis untuk data yang disimpannya \parencite{yugabyte}.

Arsitektur YugabyteDB secara teoretis menawarkan skalabilitas tulis dan ketahanan terhadap kegagalan (\textit{fault tolerance}) yang lebih superior karena tidak adanya \textit{single point of contention} seperti \textit{node coordinator} pada CitusData. Dalam skenario perebutan tiket yang ekstrem, di mana puluhan ribu pengguna mencoba memesan tiket secara bersamaan, kemampuan untuk mendistribusikan beban tulis ke semua \textit{node} secara merata adalah keuntungan yang sangat besar. Hipotesisnya adalah YugabyteDB akan menunjukkan laju pemrosesan transaksi tertinggi di bawah beban puncak. Namun, ini datang dengan pertukaran: protokol konsensus Raft yang menjamin konsistensi data di seluruh kluster dapat memperkenalkan latensi tambahan pada setiap operasi tulis.

\subsubsection{Pemilihan Final dan Alternatif yang Dikesampingkan}

Selain kedua pendekatan di atas, pemartisian pada level aplikasi atau \textit{connection pooler} (misalnya PgCat) juga merupakan alternatif yang valid. Namun, pendekatan ini tidak dipilih karena akan memindahkan kompleksitas logika distribusi ke dalam kode aplikasi, sehingga menyulitkan perbandingan yang adil antar-teknologi basis data itu sendiri \parencite{pgcat}.

Dengan demikian, untuk menjawab rumusan masalah secara komprehensif, tiga varian basis data dipilih untuk diuji:
\begin{enumerate}
    \item Kluster PostgreSQL, sebagai tolok ukur (\textit{baseline}) yang merepresentasikan arsitektur relasional tradisional yang matang dan andal.
    \item CitusData, untuk mengevaluasi efektivitas pendekatan "menambahkan distribusi" pada basis data yang sudah ada.
    \item YugabyteDB, untuk mengevaluasi kinerja arsitektur "terdistribusi secara \textit{native}" yang dirancang untuk skalabilitas \textit{cloud-native}.
\end{enumerate}

Pemilihan basis data yang seluruhnya kompatibel dengan antarmuka PostgreSQL memungkinkan pengembangan aplikasi yang sama untuk digunakan di ketiga varian, sehingga memastikan perbandingan kinerja yang lebih adil dan terfokus pada kapabilitas arsitektur basis data masing-masing.

\subsection{Pengoptimalan Operasi Baca Ketersediaan Tiket}

Berdasarkan referensi pemodelan sistem tiket, terdapat dua operasi baca ketersediaan tiket, yaitu operasi baca agregat ketersediaan berdasarkan area dan operasi baca ketersediaan kursi pada area.

Operasi pembacaan ketersediaan berdasarkan area merupakan operasi agregat terhadap data yang selalu diperbarui dan merupakan operasi yang paling banyak dipanggil mengikuti banyaknya pengguna yang ada. Keterbaruan data merupakan aspek penting, sehingga penggunaan tembolok tidak serta merta dapat digunakan.

Penggunaan denormalisasi pada level basis data dapat digunakan, seperti menambahkan kolom sisa kursi tersedia. Meskipun begitu, pendekatan ini tidak dapat digunakan karena dapat menimbulkan \textit{contention} apabila sistem ingin meningkatkan laju penulisan sebanyak-banyaknya. Oleh karena itu, kolom hasil agregat ini tidak disimpan pada basis data, tetapi disimpan pada Redis yang dapat menangani hal ini dengan lebih baik. Sama seperti sebelumnya, tantangan pendekatan ini adalah bagaimana caranya memastikan data yang ada pada basis data dengan Redis tetap sinkron, terutama pada skenario yang memiliki banyak kegagalan.

Operasi berikutnya adalah operasi baca ketersediaan kursi pada suatu area. Operasi ini membutuhkan data berdasarkan baris. Operasi ini memang lebih baik apabila langsung memanggil basis data. Meskipun begitu, pada lonjakan pengunjung yang tinggi, operasi tersebut dapat membebani basis data. Dari sisi keterbaruan data, data yang 150 milisekon lebih lama tidak akan jauh berbeda dengan data paling terbaru. Terlebih lagi, sistem ini bukan sistem yang sangat sensitif sehingga perbedaan sekian milisekon akan mempengaruhi hasil secara signifikan. Pertukaran kinerja yang ditawarkan dengan pendekatan ini membuat penggunaan tembolok secara mikro jauh lebih bermanfaat untuk menjaga stabilitas sistem. Penyimpanan tembolok dapat diimplementasikan dengan menggunakan hashmap di memori untuk latensi yang lebih baik. Waktu hidup tembolok sebesar 150 milisekon membuat penggunaan Redis tidak cocok karena perbedaan latensi yang bisa menjadi sangat signifikan.

\subsection{Integritas data dan Pencegahan Pesanan Ganda}

Pemesanan tiket dibagi menjadi dua tahap, yaitu fase \textit{hold seating} dan fase penyelesaian pesanan. Fase \textit{hold seating} akan mencadangkan tiket sampai batas waktu tertentu hingga pengguna menyelesaikan pembayaran. Saat pembayaran berhasil, pembelian tiket baru akan dianggap sukses.

Penggunaan basis data relasional seperti PostgreSQL memungkinkan implementasi pemesanan tiket yang lebih mudah dan terjamin konsistensinya. Untuk mencegah pemesanan ganda, basis data dapat melakukan penguncian pada level baris di dalam transaksi, sehingga setiap tiket hanya terjual sebanyak satu kali.

Pada pemesanan berdasarkan nomor kursi secara langsung, hal ini dapat diimplementasikan dengan memulai transaksi, lalu melakukan kueri "SELECT WHERE available FOR UPDATE" untuk baris data yang ingin dikunci. Kueri tersebut akan gagal apabila terdapat transaksi lain yang sedang mengakses data yang sama.

Pemesanan berdasarkan area tertentu (tidak memilih nomor kursi) memiliki pendekatan yang serupa dengan menggunakan kueri "SELECT WHERE available FOR UPDATE SKIP LOCKED". Kueri ini akan mengembalikan data yang tersedia untuk dikunci dan bertindak sebagai antrean untuk alokasi kursi virtual pada tiket \textit{free standing}.

\subsection{Penggunaan Skema Pengendalian Aliran Pada Pemesanan Tiket}

Dengan banyaknya pengguna yang ingin memesan tiket pada satu waktu, basis data akan cukup sibuk menangani transaksi gagal dan menangani konflik selama proses transaksional. Sebagaimana dibahas sebelumnya, penggunaan pengendalian aliran menarik untuk dieksplorasi lebih lanjut pengaruhnya terhadap kinerja sistem tiket. Setidaknya terdapat dua strategi pengendalian aliran yang dapat digunakan pada operasi pemesanan tiket, yaitu penggunaan antrian dan penolakan permintaan.

\subsubsection{Penolakan Permintaan Lebih Awal}

Tentunya penolakan permintaan lebih awal harus dilakukan secara strategis agar dapat mengurangi beban tanpa menghalangi pembelian tiket yang masih tersedia. Ide dasar yang dapat digunakan adalah dengan membuang permintaan terhadap kursi yang sudah terjual atau akan terjual (terdapat permintaan lain yang memesan kursi yang sama, tetapi belum \textit{commited} atau pesanannya masih diproses oleh sistem). Proses ini dapat dilimpahkan pada basis data yang memiliki latensi rendah seperti Redis. Tantangan pendekatan ini berada pada bagaimana cara memastikan data ketersediaan pada basis data dengan data ketersediaan pada Redis tetap tersinkronisasi terutama saat banyak terjadi kegagalan pada sistem.

Hal yang harus diperhatikan dari penggunaan Redis adalah aspek persistensi. Redis memiliki dua jenis persistensi, yaitu \textit{snapshot} dan \textit{Append-Only File}. Tentunya pada kasus ini persistensi penting untuk memitigasi kegagalan. Pengaturan yang paling direkomendasikan adalah penggunaan \textit{snapshot} dengan \textit{append-only file} (AOF). Pengaturan AOF sendiri memiliki dua pengaturan, yaitu \textit{everysec} dan \textit{always}. Opsi \textit{everysec} dipilih karena merupakan pengaturan yang menyeimbangkan kinerja dengan persistensi. Sekalipun terjadi kegagalan, data yang terdapat pada Redis dapat dibuat ulang dan permintaan yang tidak dapat ditolak pada tahap ini tetap akan ditolak pada tahap berikutnya.

\subsubsection{Penggunaan Antrean Pemrosesan Pesanan}

Penggunaan antrean didasari pada ide bahwa basis data akan kesulitan menangani banyak permintaan secara bersamaan. Banyaknya permintaan yang menumpuk mengakibatkan latensi yang tinggi dan seringkali basis data tidak dapat pulih dari kondisi tersebut kecuali bebannya dikurangi secara signifikan. Oleh karena itu, penting agar menjaga basis data pada utilisasi yang optimal sehingga basis data dapat beroperasi dengan baik. Oleh karena itu, proses pemrosesan pemesanan tiket seharusnya disesuaikan berdasarkan kapasitas basis data untuk menjaga stabilitas sistem.

Pada tugas akhir ini, pemrosesan pesanan yang sesuai dengan kapasitas sistem diimplementasikan dengan cara yang sederhana, yaitu membatasi banyaknya pesanan yang diproses pada satu waktu hingga jumlah tertentu. Penggunaan algoritma yang lebih canggih dapat digunakan, tetapi hal tersebut bukan merupakan fokus dari tugas akhir ini.

Apakah pendekatan ini berarti menjadikan proses pemesanan tiket menjadi asinkron? belum tentu. Operasi ini tetap diimplementasikan secara sinkron yang berupa pemanggilan HTTP lalu menunggu respons dari server. Hanya saja, permintaan pengguna dipindahkan ke antrean terlebih dahulu. Pertukaran dari implementasi ini adalah latensi yang lebih tinggi yang akan dirasakan dari sisi pengguna. Klasifikasi yang tepat adalah pemrosesan permintaan secara sinkron dengan pemrosesan dari sisi \textit{backend} asinkron.

Pendekatan tersebut memunculkan masalah baru terutama perihal penanganan kegagalan atau permintaan yang melewati batas waktu tertentu. Untuk menangani hal tersebut, implementasi proses pemesanan tiket dapat dibuat idempoten dengan menyertakan kunci idempotensi. Dengan begitu, apabila sebuah pemesanan tiket melewati batas waktu dari sisi klien, pesanan tetap diproses dengan sukses oleh server sebagaimana mestinya tanpa harus membuat pesanan baru.

Pertimbangan berikutnya adalah pemilihan platform antrean. Dari sisi kinerja, penggunaan platform penyiaran \textit{event} seperti Kafka atau Redpanda menawarkan latensi yang jauh lebih baik dengan penskalaan yang jauh lebih baik \parencite{comparingKafkaAlternatives}. meskipun begitu, beban implementasi untuk menjadikan platform tersebut sebagai platform antrean tradisional membutuhkan investasi yang tinggi, sehingga tidak dapat diimplementasikan dalam waktu yang singkat. Oleh karena itu, platform antrean tradisional seperti RabbitMQ dipilih karena memiliki beban implementasi yang lebih rendah dan \textit{deployment} yang lebih mudah. Solusi ini merupakan solusi yang sudah terbukti dan memilki fitur yang lengkap. Selain itu, meskipun beban yang diterima sangat tinggi, pada akhirnya jumlah permintaan yang masuk untuk proses pemesanan tiket akan jauh lebih sedikit sehingga keterbatasan penskalaan pada RabbitMQ tidak akan menjadi masalah.


\section{Rancangan Layanan Tiket}

\subsection{Sistem Tiket}

Komponen sistem tiket dapat dibagi menjadi beberapa bagian, yaitu ticket \textit{backend}, basis data relasional, dan kluster Redis. Komponen basis data relasional dapat dibagi menjadi tiga jenis, yaitu kluster PostgreSQL dengan \textit{read replica}, kluster CitusData, dan kluster YugabyteDB. Komponen ini yang akan menjadi \textit{source of truth} dari sistem ini. Selain itu, kluster Redis digunakan untuk menyimpan data agregat ketersediaan berdasarkan area.

\begin{figure}[htbp]
    \centering
    \includegraphics[width=0.8\textwidth]{resources/chapter-3/ticket-nofc.png}
    \caption{Diagram Arsitektur Sistem Tiket Tanpa \textit{Flow Control}}
    \label{fig:ticket-nofc}
\end{figure}

\pagebreak

Selain itu, berikut adalah variasi konfigurasi RDBMS yang mungkin terjadi.

\begin{figure}[htbp]
    \centering
    \includegraphics[width=0.5\textwidth]{resources/chapter-3/rdbms.png}
    \caption{Variasi RDBMS}
    \label{fig:rdbms-variation}
\end{figure}

Pada konfigurasi kluster PostgreSQL, klien terhubung dengan semua \textit{instance}. Pada konfigurasi CitusData, klien hanya terhubung dengan koordinator dan koordinator yang akan meneruskan permintaan kepada \textit{worker}. Pada konfigurasi YugabyteDB, klien terhubung dengan semua Master yang masing-masing terhubung dengan TServer. Klien sebenarnya dapat terhubung dengan salah satu master saja, tetapi konfigurasi seperti ini membuat koneksi klien ke YugabyteDB menjadi lebih \textit{fault tolerant} dan juga dapat mengurangi beban agar tidak terpusat pada satu \textit{instance} saja.

\begin{figure}[htbp]
    \centering
    \includegraphics[width=0.8\textwidth]{resources/chapter-3/ticket-fc.png}
    \caption{Diagram Arsitektur Sistem Tiket dengan \textit{Flow Control}}
    \label{fig:ticket-fc}
\end{figure}

Pada sistem tiket dengan \textit{flow control}, terdapat dua komponen baru yaitu RabbitMQ dan \textit{booking processor}. RabbitMQ bertugas untuk menyimpan \textit{queue} permintaan pemesanan tiket dan \textit{booking processor} bertugas untuk memproses pemesanan tiket. Selain itu, kluster Redis memiliki tanggung jawab tambahan untuk menyimpan data yang digunakan untuk \textit{early dropping} permintaan pesanan yang masuk.

\section{Rencana Pengujian}

Bagian ini membahas rancangan awal rencana pengujian penelitian ini.

\subsection{Lingkungan Pengujian}

Penelitian ini rencananya akan diuji pada \textit{cloud provider} dengan \textit{dedicated hardware}. Penelitian ini akan menggunakan dua lingkungan yang berbeda, yaitu:

\begin{enumerate}
    \item \textit{Virtual machine} untuk menjalankan basis data PostgreSQL. Basis data relasional akan lebih optimal apabila dijalankan pada lingkungan sendiri, bukan pada kontainer. Meskipun begitu, perlu mekanisme khusus agar \textit{redeployment} dan konfigurasi dapat diotomasi untuk memudahkan proses pengujian.
    \item \textit{Kubernetes} untuk melakukan orkestrasi kontainer layanan lainnya.
    \item \textit{Virtual machine} untuk menjalankan agen yang bertindak sebagai pengguna ketika proses pengujian berlangsung.
\end{enumerate}

Mengingat penelitian ini fokus pada peningkatan \textit{throughput}, setiap node yang dijalankan akan berada pada satu \textit{data center}/ \textit{availability zone} yang sama agar latensi dapat dikurangi. Selain itu, setiap node akan menggunakan server dengan media penyimpanan bertipe SSD NVME untuk memaksimalkan \textit{throughput} perangkat keras.

\section{Mekanisme Pengawasan}

Pengawasan merupakan aspek yang krusial untuk mengukur kinerja dari setiap solusi yang dibahas pada penelitian ini. Hal ini meliputi \textit{health check}, log, dan metrik. Selama pengujian, penelitian ini akan menggunakan kakas yang umum digunakan untuk mendukung fungsi pengawasan yaitu Grafana Stack yang meliputi dasbor Grafana, Grafana Loki untuk pengumpulan log, Grafana Agent untuk pengiriman log, dan Prometheus untuk pengumpulan metrik.

Metrik yang akan dikumpulkan merupakan metrik dasar yang umum digunakan untuk melakukan pengawasan. Setiap metrik yang diukur merupakan metrik yang bersifat \textit{time series}. Metrik yang dikumpulkan meliputi tetapi tidak terbatas pada metrik-metrik berikut:

\begin{enumerate}
    \item Metrik mesin, seperti penggunaan RAM, CPU, media penyimpanan, \textit{network in/out}.
    \item Metrik layanan HTTP, seperti jumlah permintaan yang dibagi berdasarkan kode respons, distribusi latensi, banyaknya galat, dan lain-lain.
    \item Metrik PostgreSQL, seperti IOPS, QPS, penggunaan CPU, jumlah koneksi, \textit{locking} dan \textit{contention}, \textit{replication lag}, dan lain-lain.
    \item Metrik Redpanda, seperti \textit{throughput}, latensi, penggunaan media penyimpanan, jumlah partisi, dan lain-lain.
    \item Metrik Redis, seperti \textit{throughput}, latensi, penggunaan memori, metrik \textit{persistence}, metrik replikasi, jumlah galat, dan lain-lain.
\end{enumerate}

Pada dasarnya, setiap metrik yang relevan dalam proses pengujian akan dikumpulkan terlebih dahulu. Setelah proses pengujian selesai, data baru akan diolah untuk memperoleh \textit{insights} dan perbandingan dari setiap solusi yang diimplementasikan.

\section{Agen Penguji dan Alur Pengujian}

Selama pengujian, akan ada banyak agen yang dijalankan untuk menyimulasikan perilaku pengguna mulai dari \textit{login} hingga pemesanan berhasil atau gagal. Agen ini juga akan mencatat latensi dan waktu yang dibutuhkan untuk menyelesaikan setiap tahapan saat pemesanan. Berikut adalah gambaran diagram alur agen penguji:

Pertama-tama, agen akan melakukan aksi \textit{login} lalu menunggu hingga penjualan dibuka. Setiap agen sudah diatur untuk memiliki kategori tiket tertentu yang ingin dibeli beserta jumlahnya. Saat sudah dibuka, agen akan mengambil data ketersediaan tiket, memilih \textit{seat} dan jumlah tiket yang ingin dipesan, lalu melakukan proses pembelian. Apabila pesanan gagal dibuat, agen akan memanggil ulang data ketersediaan tiket, lalu mencoba memesan lagi sampai tidak ada tiket tersedia yang sesuai dengan kebutuhan agen. Apabila pesanan berhasil dibuat, agen akan melakukan pembayaran. Pembayaran yang dilakukan agen dibuat agar memiliki kemungkinan kecil untuk gagal. Setelah pembayaran terverifikasi, agen akan memanggil \textit{endpoint} untuk memeriksa status pesanan yang sudah dibuat. Apabila pesanan sudah terkonfirmasi, alur pembelian untuk agen tersebut dianggap selesai.

\begin{figure}[htbp]
    \centering
    \includegraphics[width=0.4\textwidth]{resources/appendix/flow-agent.png}
    \caption{Alur Perilaku Agen Penguji}
    \label{fig:agent-flow}
\end{figure}

\subsection{Skenario Pengujian}

Terdapat dua jenis skenario pengujian yang akan diuji pada penelitian ini, yaitu \textit{load test} dan \textit{scaled-down simulation test}. Keduanya memiliki pendekatan dan tujuan yang berbeda.

\subsubsection{Load Test}

Skenario ini berfokus pada jumlah concurrent virtual user. Pada skenario ini, sistem akan dibebankan sejumlah concurrent virtual user dalam waktu hingga 15 menit. Kemudian, analisis dilakukan untuk diketauhi kinerjanya. Meskipun begitu, skenario ini tidak menunjukkan keadaan contention sebagiamana terjadi pada kehidupan nyata.

\subsubsection{Scaled-Down Simulation Test}

Skenario ini menguji beberapa hal, seperti \textit{spike testing} dan perilaku sistem pada saat terjadi contention. Tes ini terinspirasi dari rasio antara pengguna dengan tiket yang dijual. Rasio ini sangat tinggi karena ada banyak pengguna yang menginginkan satu tiket.

Skenario ini berfokus pada jumlah arrival rate alih-alih concurrent virtual user. Skenario ini dibuat dengan cara berikut:

\begin{enumerate}
    \item Estimasikan jumlah pengguna dan jumlah tiket yang dijual. Misalkan terdapat 40 ribu pengguna (iterasi alur pengguna) dan 16 ribu tiket yang dijual.
    \item Distribusikan arrival pengguna berdasarkan distribusi tertentu. Penelitian ini menggunakan distribusi lognormal untuk menyimulasikan spike. Total area di bawah kurva akan sama dengan jumlah pengguna. Buat range/ histogram.
    \item Hasil grafik distribusi digunakan untuk menentukan jumlah arrival pengguna pada waktu tertentu.
\end{enumerate}


\section{Tantangan Pengujian}

Tantangan utama pengujian penelitian ini adalah estimasi biaya yang harus dikeluarkan. Untuk mengoptimalkan biaya yang dikeluarkan, proses pengujian harus dilakukan secepat mungkin dan menggunakan layanan komputasi awan yang lebih murah dibandingkan dengan yang lain atau menggunakan kredit gratis dari penyedia layanan apabila masih tersedia. Berikut adalah rencana yang akan dilakukan untuk mengoptimalkan biaya pengujian:

\begin{enumerate}
    \item Salah satu layanan komputasi awan Hetzner menyediakan layanan yang cukup terjangkau, terutama untuk server yang berada di Jerman. Meskipun begitu, layanan ini tidak menyediakan \textit{managed} Kubernetes. Akan tetapi, terdapat proyek Terraform Kube Hetzner yang bisa memudahkan konfigurasi lingkungan Kubernetes.
    \item Proses pengembangan dilakukan secara lokal. Selain itu, otomasi dan \textit{script} perlu digunakan agar lingkungan yang dikembangkan secara lokal dapat dengan mudah di-\textit{reproduce} di lingkungan pengujian. Otomasi ini juga berguna untuk mempercepat proses \textit{deployment} dan proses pengujian.
\end{enumerate}

% \chapter{Rencana Pelaksanaan}

\chapter{Rencana Pelaksanaan}

\chapter{Rencana Pelaksanaan}

\input{chapters/schedule/schedule.tex}

\input{chapters/schedule/risk.tex}


\section{Risiko Pelaksanaan}

Berikut adalah risiko yang diperkirakan dapat menghambat pelaksanaan tugas akhir:

\begin{enumerate}
    \item Karena tingkat kesulitan yang belum tentu dapat diestimasikan dengan baik, ketiga arsitektur belum tentu dapat diimplementasikan. Apabila terdapat banyak hambatan, hanya dua dari tiga arsitektur yang akan diimplementasikan, yaitu arsitektur acuan dan arsitektur yang mengoptimalkan PostgreSQL.
    \item Mekanisme pengujian dan uji kasus harus dirancang dengan baik agar dapat merepresentasikan beban di dunia nyata. Untuk menangani hal ini, mekanisme pengujian dan uji kasus harus menjadi aspek yang mendapat perhatian khusus. Dalam perancangan pengujian, sebaiknya mempertimbangkan berbagai referensi mekanisme \textit{benchmark} yang khusus dibuat untuk basis data transaksional atau sistem dengan kasus serupa.
    \item Karena keterbatasan biaya, jumlah sumber daya yang akan dialokasikan untuk pengujian tidak akan sepenuhnya sama dengan apa yang terjadi di \textit{production}. Untuk menangani hal ini, berbagai opsi lingkungan pengujian harus dipertimbangkan, seperti penyedia layanan yang lebih murah seperti Hetzner atau DigitalOcean. Lingkugnan pengujian tidak harus dilakukan pada layanan Google Cloud atau AWS.
    \item Perbedaan kinerja antar arsitektur dapat disebabkan karena pemilihan bahasa, \textit{hardware}, \textit{tools}, atau pun hal lain yang tidak diduga. Oleh karena itu, variasi lingkungan pengujian harus dibuat seminimal mungkin, sehingga perbedaan pada hasil pengujian benar-benar disebabkan karena perbedaan arsitektur. Untuk meminimalkan perbedaan lingkungan, setiap arsitektur harus diuji pada lingkungan yang sama dengan \textit{hardware dedicated}. Selain itu, pemilihan bahasa dan kakas harus sama untuk setiap arsitektur.
    \item Apabila skala lingkungan yang diuji merupakan miniatur dari lingkungan sesungguhnya, kinerja pada lingkungan sesungguhnya belum tentu \textit{scale} dengan cara yang sama sebagaimana yang terjadi pada lingkungan pengujian. Agar \textit{scale behavior} juga dapat diprediksi, pengujian sebaiknya dilakukan dengan jumlah beban dan jumlah sumber daya yang berbeda-beda.
\end{enumerate}


\section{Risiko Pelaksanaan}

Berikut adalah risiko yang diperkirakan dapat menghambat pelaksanaan tugas akhir:

\begin{enumerate}
    \item Karena tingkat kesulitan yang belum tentu dapat diestimasikan dengan baik, ketiga arsitektur belum tentu dapat diimplementasikan. Apabila terdapat banyak hambatan, hanya dua dari tiga arsitektur yang akan diimplementasikan, yaitu arsitektur acuan dan arsitektur yang mengoptimalkan PostgreSQL.
    \item Mekanisme pengujian dan uji kasus harus dirancang dengan baik agar dapat merepresentasikan beban di dunia nyata. Untuk menangani hal ini, mekanisme pengujian dan uji kasus harus menjadi aspek yang mendapat perhatian khusus. Dalam perancangan pengujian, sebaiknya mempertimbangkan berbagai referensi mekanisme \textit{benchmark} yang khusus dibuat untuk basis data transaksional atau sistem dengan kasus serupa.
    \item Karena keterbatasan biaya, jumlah sumber daya yang akan dialokasikan untuk pengujian tidak akan sepenuhnya sama dengan apa yang terjadi di \textit{production}. Untuk menangani hal ini, berbagai opsi lingkungan pengujian harus dipertimbangkan, seperti penyedia layanan yang lebih murah seperti Hetzner atau DigitalOcean. Lingkugnan pengujian tidak harus dilakukan pada layanan Google Cloud atau AWS.
    \item Perbedaan kinerja antar arsitektur dapat disebabkan karena pemilihan bahasa, \textit{hardware}, \textit{tools}, atau pun hal lain yang tidak diduga. Oleh karena itu, variasi lingkungan pengujian harus dibuat seminimal mungkin, sehingga perbedaan pada hasil pengujian benar-benar disebabkan karena perbedaan arsitektur. Untuk meminimalkan perbedaan lingkungan, setiap arsitektur harus diuji pada lingkungan yang sama dengan \textit{hardware dedicated}. Selain itu, pemilihan bahasa dan kakas harus sama untuk setiap arsitektur.
    \item Apabila skala lingkungan yang diuji merupakan miniatur dari lingkungan sesungguhnya, kinerja pada lingkungan sesungguhnya belum tentu \textit{scale} dengan cara yang sama sebagaimana yang terjadi pada lingkungan pengujian. Agar \textit{scale behavior} juga dapat diprediksi, pengujian sebaiknya dilakukan dengan jumlah beban dan jumlah sumber daya yang berbeda-beda.
\end{enumerate}


\chapter{Implementasi dan Pengujian}

\section{Implementasi}

\subsection{Batasan Implementasi}

Berikut adalah batasan yang ditetapkan dalam melakukan implementasi sistem ini.

\begin{enumerate}
  \item Semua batasan masalah dan konfigurasi yang telah dibahas pada bagian \ref{sec:batasan-masalah}.
  \item Layanan otentikasi/ pengguna tidak diimplementasikan sebagaimana dibatas pada bagian rancangan solusi.
  \item Layanan pembayaran merupakan \textit{mock service} yang bertujuan untuk menyimulasikan sistem eksternal yang berfungsi sebagai gerbang pembayaran.
  \item Setiap layanan yang diimplementasikan hanya berupa REST API Endpoint yang akan dipanggil secara langsung oleh \textit{virtual user}. Antarmuka sistem tidak diimplementasikan karena dinilai tidak relevan dengan fokus pengujian.
  \item Terdapat dasbor Grafana dan Prometheus server untuk menampilkan metrik dan status setiap sistem.
\end{enumerate}

\subsection{Lingkungan Pengujian}

Penelitian ini rencananya akan diuji pada \textit{cloud provider} dengan \textit{dedicated hardware}. Penelitian ini akan menggunakan dua lingkungan yang berbeda, yaitu:

\begin{enumerate}
    \item \textit{Virtual machine} untuk menjalankan basis data PostgreSQL. Basis data relasional akan lebih optimal apabila dijalankan pada lingkungan sendiri, bukan pada kontainer. Meskipun begitu, perlu mekanisme khusus agar \textit{redeployment} dan konfigurasi dapat diotomasi untuk memudahkan proses pengujian.
    \item \textit{Kubernetes} untuk melakukan orkestrasi kontainer layanan lainnya.
    \item \textit{Virtual machine} untuk menjalankan agen yang bertindak sebagai pengguna ketika proses pengujian berlangsung.
\end{enumerate}

Mengingat penelitian ini fokus pada peningkatan \textit{throughput}, setiap node yang dijalankan akan berada pada satu \textit{data center}/ \textit{availability zone} yang sama agar latensi dapat dikurangi. Selain itu, setiap node akan menggunakan server dengan media penyimpanan bertipe SSD NVME untuk memaksimalkan \textit{throughput} perangkat keras.

\subsection{Kakas yang Digunakan}

\subsubsection{Layanan Pembayaran}

Node.JS 20
TypeScript
Hono sebagai HTTP Framework
BullMQ sebagai kakas Message Queue di atas Redis

\subsection{Layanan Tiket}

Golang 1.24.0
Golang Echo sebagai HTTP Framework
TestContainers untuk integration testing
Go Concurrency Limits
PGX Postgres Driver Library
BigCache untuk in-memory cache
UberFX untuk dependency injection

\subsection{Lainnya}

Grafana K6 untuk library load testing
TypeSpec untuk membuat dokumentasi OpenAPI
Integrasi Prometheus untuk metrics collection



\subsection{Docker Image Container dan Helm Chart}

Terdapat berbagai Docker image container dan Helm Chart yang digunakan untuk proses implementasi dan \textit{deployment}, seperti:

\begin{enumerate}
    \item Custom image layanan pembayaran dengan base image node:20.19.1-bookworm-slim. Container ini menyimpan entrypoint untuk payment server dan payment notifier.
    \item Custom image layanan tiket dengan base image golang:1.24.0-bullseye. Container ini menyimpan berkas hasil kompilasi untuk server dengan varian \textit{flow control} dan tanpa \textit{flow control}, seeder, serta \textit{order worker} dengan \textit{flow control}.
    \item Custom image k6 dengan base image grafana/k6:1.0.0. Container ini berisi \textit{custom built} berkas eksekusi K6 dengan tambahan ekstensi grafana/xk6-faker.
    \item Custom image PostgreSQL untuk setup kluster primary-replica dengan base image postgres:16.8. Image ini diadaptasi dari contoh konfigurasi kluster PostgreSQL dengan Patroni.
    \item Custom image CitusData dengan base image postgres:16.8. Image ini diadaptasi dari contoh konfigurasi kluster CitusData dengan Patroni.
    \item Helm chart grafana/grafana versi 8.13.1.
    \item Helm chart prometheus versi 27.11.0.
    \item Helm chart grafana/loki versi 6.29.0.
    \item Helm chart grafana/alloy versi 1.0.2.
    \item Helm chart jetstack/cert-manager versi 1.17.2.
    \item Helm chart bitnami/nginx-ingress-controller versi 11.6.16.
    \item Helm chart bitnami/redis-cluster versi 11.5.3.
    \item Helm chart bitnami/rabbitmq versi 16.0.1.
    \item Helm chart yugabytedb/yugabyte versi 2024.2.2.
    \item Helm chart icoretech/pgbouncer versi 2.7.0.
    \item Operator grafana/k6-operator versi 3.13.0.
\end{enumerate}

\subsection{Konfigurasi Deployment Kluster Sistem Backend}

\subsubsection{Sistem Pengawasan}

\begin{figure}[htbp]
    \centering
    \includegraphics[width=1\textwidth]{resources/chapter-4/monitoring.png}
    \caption{Deployment Sistem Monitoring}
    \label{fig:deployment-monitoring}
\end{figure}

Sebagaimana dibahas pada bagian sebelumnya, komponen utama dari sistem pengawasan ini terdiri atas Grafana Alloy, Grafana Loki, Grafana Dashbaord, dan Prometheus. Meskipun begitu, terdapat beberapa service dan container tambahan yang merupakan konfigurasi default dari chart yang digunakan.

\pagebreak

\subsubsection{Nginx Ingress Controller}

\begin{figure}[htbp]
    \centering
    \includegraphics[width=1\textwidth]{resources/chapter-4/nginx.png}
    \caption{Deployment Nginx Ingress}
    \label{fig:deployment-nginx}
\end{figure}

Sistem ini terdiri atas cert-manager untuk menangani sertifikat SSL domain secara otomatis, deployment dan service Nginx Ingress Controller, serta beberapa ingress yang mengekspos layanan internal kubernetes ke dunia luar. Pada kluster ini terdapat empat layanan yang diekspos, yaitu layanan tiket, layanan pembayaran, dasbor prometheus, dan dasbor Grafna.

\pagebreak

\subsubsection{Layanan Pembayaran}

\begin{figure}[htbp]
    \centering
    \includegraphics[width=0.8\textwidth]{resources/chapter-4/payment.png}
    \caption{Deployment Layanan Pembayaran}
    \label{fig:deployment-payment}
\end{figure}

Layanan pembayaran secara umum terdiri atas deployment payment, service payment, dan kluster Redis. Deployment payment terdiri atas dua container, yaitu payment service dan payment notifier.

\pagebreak

\subsubsection{Layanan Tiket varian tanpa \textit{Flow Control}}

\begin{figure}[htbp]
    \centering
    \includegraphics[width=1\textwidth]{resources/chapter-4/ticket-nofc.png}
    \caption{Deployment Layanan Ticket tanpa \textit{Flow Control}}
    \label{fig:deployment-ticket-nofc}
\end{figure}

Layanan tiket tanpa \textit{flow control} terdiri atas deployment ticket server, kluster redis, dan sanity check. Sistem ini juga terhubung dengan salah satu kluster basis data relasional yang ada melalui Pgcat.

\pagebreak

\subsubsection{Layanan Tiket varian dengan \textit{Flow Control}}

\begin{figure}[htbp]
    \centering
    \includegraphics[width=1\textwidth]{resources/chapter-4/ticket-fc.png}
    \caption{Deployment Layanan Ticket dengan \textit{Flow Control}}
    \label{fig:deployment-ticket-fc}
\end{figure}

Layanan tiket dengan \textit{flow control} terdiri atas deployment ticket server, ticket-worker, kluster redis, rabbitmq, dan sanity check. Sistem ini juga terhubung dengan salah satu kluster basis data relasional yang ada melalui Pgcat.

\pagebreak

\subsubsection{Kluster PostgreSQL}

\begin{figure}[htbp]
    \centering
    \includegraphics[width=1\textwidth]{resources/chapter-4/postgres.png}
    \caption{Deployment kluster PostgreSQL}
    \label{fig:deployment-postgres}
\end{figure}

Kluster PostgreSQL terdiri atas stateful set PostgreSQL yang akan terdiri atas satu primary dan satu replica. Selain itu, terdapat Pgcat yang bertindak sebagai pooler dan load balancer di level query. Pgcat akan membaca kueri dan meneruskan query kepada primary atau replica berdasarkan jenis kueri dan beban setiap instance.

\pagebreak

\subsubsection{Kluster CitusData}

\begin{figure}[htbp]
    \centering
    \includegraphics[width=1\textwidth]{resources/chapter-4/citusdata.png}
    \caption{Deployment kluster CitusData}
    \label{fig:deployment-citusdata}
\end{figure}

Kluster CitusData terdiri atas tiga stateful set CitusCluster. Cluster pertama merupakan koordinator dan sisanya merupakan worker. Terdapat Pgcat yang terhubung dengan koordinator. Sistem tiket terhubung langsung dengan Pgcat. Koneksi dengan worker dilakukan melalui koordinator.

\pagebreak

\subsubsection{Kluster YugabyteDB}

\begin{figure}[htbp]
    \centering
    \includegraphics[width=1\textwidth]{resources/chapter-4/yugabyte.png}
    \caption{Deployment kluster YugabyteDB}
    \label{fig:deployment-yugabyte}
\end{figure}

Kluster YugabyteDB terdiri atas stateful set yugabyte master, stateful set yugabyte tablet server, dan Pgcat. Pgcat terhubung dengan semua instance yugabyte master. Sistem tiket terhubung langsung dengan Pgcat. Koneksi dengan yugabyte master di-\textit{load balance} oleh Pgcat.

\pagebreak

\subsection{Konfigurasi Deployment Kluster Agen Penguji}

\subsubsection{Sistem Pengawasan}

\begin{figure}[htbp]
    \centering
    \includegraphics[width=1\textwidth]{resources/chapter-4/agent-monitoring.png}
    \caption{Deployment Sistem Monitoring}
    \label{fig:deployment-monitoring-agent}
\end{figure}

Sistem pengawasan pada agen penguji lebih sederhana dengan komponen hanya terdiri atas Grafana Dashboard dan Prometheus.

\pagebreak

\subsubsection{Nginx Ingress Controller}

\begin{figure}[htbp]
    \centering
    \includegraphics[width=1\textwidth]{resources/chapter-4/agent-nginx.png}
    \caption{Deployment Nginx Ingress}
    \label{fig:deployment-nginx-agent}
\end{figure}

Sistem ini terdiri atas cert-manager untuk menangani sertifikat SSL domain secara otomatis, deployment dan service Nginx Ingress Controller, serta beberapa ingress yang mengekspos layanan internal kubernetes ke dunia luar. Pada kluster ini terdapat dua layanan yang diekspos, yaitu dasbor prometheus dan dasbor Grafna.

\pagebreak

\subsubsection{K6 Operator}

\begin{figure}[htbp]
    \centering
    \includegraphics[width=1\textwidth]{resources/chapter-4/k6-operator.png}
    \caption{Deployment K6 Operator}
    \label{fig:deployment-k6-operator}
\end{figure}

K6 Operator merupakan komponen utama kluster penguji. Komponen ini mengatur inisiasi dan runtime script pengujian k6.

\pagebreak

% \input{chapters/chapter-4/implementasi/04-dashboard.tex}

% \input{chapters/chapter-4/implementasi/05-service.tex}

\pagebreak

\subsection{Pengujian Implementasi}

Terdapat dua pengujian jenis yang dilakukan untuk memverifikasi kebenaran implementasi dan konfigurasi \textit{deployment}, yaitu \textit{unit test} dan \textit{integration test} serta pengujian pada kluster lokal Kubernetes dengan K3d. Detail prosedur dan hasil pengujian dibahas pada bagian Lampiran \ref{apx:implementation-test}.


\pagebreak

\section{Analisis Hasil Pengujian}

\subsection{Siklus Penjualan Tiket}

Kedua skenario pengujian ini memiliki siklus/fase pola permintaan yang berbeda. Contoh pada skenario beban berkelanjutan diambil dari pengujian f1t2, sedangkan contoh pada skenario perebutan tiket diambil dari pengujian f1t4. Siklus ini juga terjadi pada skenario serupa ketika proses penjualan berjalan dengan lancar.

\subsubsection{Skenario Beban Berkelanjutan}

Pada skenario ini, sistem memiliki ketersediaan tiket yang banyak sehingga penjualan tiket yang berhasil berjalan cukup lama. Setelah tiket habis terjual, jumlah permintaan ketersediaan tiket jauh meningkat karena penguji mencoba mencari tiket dari berbagai area dengan lebih banyak. Pola ini ditunjukkan pada Gambar \ref{fig:pattern-stress-traffic}.

\begin{figure}[htbp]
    \centering
    \includegraphics[width=0.5\textwidth]{resources/chapter-4/pattern-stress-traffic.png}
    \caption{Pola Permintaan pada Beban Berkelanjutan}
    \label{fig:pattern-stress-traffic}
\end{figure}

Status respons pemesanan tiket menunjukkan bahwa terjadi konflik (kode 409) selama pemrosesan pesanan sebanyak 3-10\% dari respons secara keseluruhan. Hal ini ditunjukkan pada Gambar \ref{fig:pattern-stress-order}.

\begin{figure}[htbp]
    \centering
    \includegraphics[width=0.5\textwidth]{resources/chapter-4/pattern-stress-order.png}
    \caption{Status Respons pada Beban Berkelanjutan}
    \label{fig:pattern-stress-order}
\end{figure}

\pagebreak

\subsubsection{Skenario Perebutan Tiket}

Pada skenario ini, jumlah permintaan ketersediaan jauh lebih banyak dibandingkan permintaan pemesanan tiket. Hal ini wajar karena terdapat lebih banyak peminat dibandingkan dengan tiket yang tersedia. Berdasarkan Gambar \ref{fig:pattern-sim-traffic}, tiket habis terjual dalam kurang lebih 2 menit penjualan.


\begin{figure}[htbp]
    \centering
    \includegraphics[width=0.6\textwidth]{resources/chapter-4/pattern-sim-traffic.png}
    \caption{Pola Permintaan pada Perebutan Tiket}
    \label{fig:pattern-sim-traffic}
\end{figure}

Status respons pemesanan tiket menunjukkan bahwa terjadi konflik (kode 409) yang cukup banyak bahkan sejak proses penjualan dimulai. Pengguna yang baru masuk setelahnya pada akhirnya akan gagal memperoleh tiket, sebagaimana digambarkan pada keadaan akhir penguji pada Gambar \ref{fig:pattern-sim-order}.

\begin{figure}[htbp]
    \centering
    \includegraphics[width=0.6\textwidth]{resources/chapter-4/pattern-sim-order.png}
    \caption{Status Respons pada Perebutan Tiket}
    \label{fig:pattern-sim-order}
\end{figure}


\subsection{Kinerja Selama Pengujian}

Gambaran kinerja setiap varian basis data ditunjukkan pada Tabel \ref{table:kinerja-pemrosesan-pesanan}. Secara garis besar, varian PostgreSQL mampu menangani beban yang diberikan dengan sangat baik. Varian CitusData masih memiliki kinerja yang baik, meski memiliki latensi yang lebih tinggi dibandingkan dengan PostgreSQL. YugabyteDB mengalami lebih banyak kegagalan dan kinerja yang buruk sehingga tidak dapat menyelesaikan proses penjualan dengan baik. Sampel pengujian yang dipilih pada adalah skenario stress-2 tanpa pengendalian aliran. Skenario ini merupakan skenario pengujian dengan kinerja YugabyteDB yang dapat diterima dan mampu menyelesaikan proses penjualan. Dengan beban yang lebih tinggi, YugabyteDB mengalami banyak kegagalan.

\begin{table}[h]
    \centering
    \caption{Gambaran Kinerja Permosesan Pesanan}
    \label{table:kinerja-pemrosesan-pesanan}
    \begin{tabular}{|l|l|l|l|}
        \hline
        \textbf{Metrik}              & \textbf{PostgreSQL} & \textbf{CitusData} & \textbf{YugabyteDB} \\
        \hline
        \textit{Throughput} Maksimum & 466 rps             & 410 rps            & 216 rps             \\
        \hline
        Penggunaan CPU (Puncak)      & 8 vCPU              & 10 vCPU            & 19 vCPU             \\
        \hline
        Penggunaan Memori (Puncak)   & 3.4 GB              & 5 GB               & 36 GB               \\
        \hline
        Latensi Pemrosesan (P50)     & 192-382 ms          & 496-650 ms         & 854-10000 ms        \\
        \hline
    \end{tabular}
\end{table}

Latensi dan \textit{throughput} pemrosesan diukur dari sisi Ticket Server, bukan dari level basis data. Latensi pemrosesan yang diambil merupakan latensi terbaik saat beban sudah cukup tinggi dan laju pemrosesan mulai stabil. Penggunaan CPU dan penggunaan memori merupakan agregat puncak jumlah penggunaan sumber daya setiap instans basis data yang berkaitan. Apabila terdapat dua nilai maksimal \textit{throughput} yang mendekati, nilai yang dipilih adalah nilai yang memiliki rasio sukses (kode respons 200) paling tinggi.

Berdasarkan Tabel \ref{table:kinerja-pemrosesan-pesanan}, PostgreSQL unggul dari segala aspek, mulai dari laju pemrosesan, penggunaan sumber daya, serta latensi. CitusData memiliki laju pemrosesan yang mendekati PostgreSQL dengan latensi 2x lebih lambat dan penggunaan sumber daya yang sedikit lebih tinggi. Di sisi lain, YugabyteDB memiliki laju pemrosesan kurang dari setengah PostgreSQL dengan latensi setidaknya 4x lebih lambat, penggunaan CPU 2.4x lebih banyak, dan penggunaan memori 10x lebih banyak.

\subsection{Kinerja Basis Data Terdistribusi}

Sebagaimana digambarkan pada Tabel \ref{table:kinerja-pemrosesan-pesanan}, kinerja PostgreSQL lebih baik dibandingkan CitusData dan YugabyteDB. Berikut adalah penelusuran lebih mendalam untuk kinerja setiap basis data, terutama pada skenario lain.

\subsubsection{Rata-Rata Waktu Eksekusi Kueri}

Tabel \ref{table:latensi-kueri} menunjukkan data rata-rata waktu eksekusi kueri yang diambil dari tabel pg\_stat\_statements sesaat setelah pengujian dilaksanakan. Data PostgreSQL diambil dari pengujian f1t2, CitusData diambil dari pengujian f2t3, dan YugabyteDB diambil dari f3t1. Beban antara PostgreSQL dan CitusData sama karena pengujian dijalankan pada skenario stress-0, sedangkan beban pada YugabyteDB lebih ringan karena dijalankan pada skenario stress-2. Data PostgreSQL merupakan gabungan antara statistik pada instans \textit{primary} dan replika. Data mentah merupakan data 10 kueri dengan total eksekusi paling banyak. Kueri yang kosong berarti total eksekusinya tidak cukup signifikan, sehingga dapat diasumsikan latensinya cukup rendah untuk diabaikan dibandingkan dengan yang lain.

\begin{table}[h]
    \centering
    \caption{Latensi Eksekusi Kueri pada Basis Data dalam Milisekon}
    \label{table:latensi-kueri}
    \begin{tabular}{|l|l|l|l|}
        \hline
        \textbf{Kueri}        & \textbf{PostgreSQL} & \textbf{CitusData} & \textbf{YugabyteDB} \\
        \hline
        LockFreeStandingSeats & 8.5                 & 9.8                & 1855                \\
        \hline
        LockNumberedSeats     & -                   & 8.25               & 43                  \\
        \hline
        InsertOrder           & 0.2                 & 4.4                & 77                  \\
        \hline
        UpdateOrder           & -                   & 3.9                & 33                  \\
        \hline
        InsertOrderItem       & 0.24                & -                  & 40                  \\
        \hline
        InsertIssuedTiket     & 0.28                & -                  & 74                  \\
        \hline
        InsertInvoice         & 0.12                & 4.16               & 31                  \\
        \hline
        UpdateInvoice         & 0.12                & 7.7                & -                   \\
        \hline
        UpdateSeatStatus      & 0.06                & 4.34               & 47                  \\
        \hline
        GetAllSeats           & 10                  & 16                 & -                   \\
        \hline
        GetOrderDetail        & 3.5                 & 6.7                & 163                 \\
        \hline
    \end{tabular}
\end{table}

Latensi dan laju pemrosesan yang baik pada PostgreSQL didukung dengan latensi kueri yang rendah. Kueri baca pada CitusData beberapa milisekon lebih lambat dibandingkan dengan PostgreSQL. Perbedaan signifikan terletak pada latensi penulisan yang berada pada ordo milisekon. Bila dibandingkan, latensi penulisan pada CitusData puluhan kali lebih lambat daripada latensi penulisan pada PostgreSQL. Di sisi lain, latensi kueri pada YugabyteDB lebih lambat hingga ratusan kali lipat dibandingkan dengan PostgreSQL baik itu pada kueri baca dan penulisan. Terlebih lagi, YugabyteDB kesulitan menangani kueri LockFreeStandingSeats dengan latensi hingga 1800 milisekon. Perbedaan latensi ini yang membuat laju penulisan CitusData dan YugabyteDB lebih lambat dibandingkan dengan PostgreSQL.

\subsubsection{Pembahasan Kinerja PostgreSQL}

Kinerja PostgreSQL yang baik berasal dari arsitekturnya yang monolitik. PostgreSQL menghindari \textit{overhead} latensi jaringan dan beban koordinasi antar \textit{node}. Beban ini ada pada basis data terdistribusi seperti CitusData dan YugabyteDB. Hal ini terbukti dengan latensi tulis dan baca yang secara konsisten jauh lebih rendah dibandingkan dengan yang lain. Arsitektur inilah yang memungkinkan laju pemrosesan yang tinggi dengan efisiensi penggunaan sumber daya yang baik.

\subsubsection{Pembahasan Kinerja CitusData}

Meskipun secara teori ide pemartisian pada PostgreSQL dengan CitusData terdengar baik, terdapat \textit{tradeoff} dari sisi koordinator yang ternyata tidak dapat diabaikan. Pertukaran ini semakin tidak dapat diabaikan pada kasus penjualan tiket ketika pola kueri adalah banyak kueri ringan yang dijalankan secara berulang-ulang. Perhatikan perbedaan penggunaan sumber daya antara koordinator dan \textit{worker} CitusData pada Gambar \ref{fig:citusdata-usage}.

\begin{figure}[htbp]
    \centering
    \includegraphics[width=1\textwidth]{resources/chapter-4/citusdata-usage.png}
    \caption{Penggunaan CPU CitusData (f2t3)}
    \label{fig:citusdata-usage}
\end{figure}

\textit{Node} cituscluster-0 merupakan koordinator dengan sisanya adalah \textit{worker}. Pada kasus ini, beban pada koordinator cukup tinggi hingga dua kali beban pada satu instans \textit{worker}. Berdasarkan hal tersebut, diketahui bahwa \textit{overhead} pada koordinator memiliki dampak yang signifikan pada kinerja CitusData.

Setelah ditelusuri, terdapat sebuah diskusi pada repositori CitusData yang membahas kinerja CitusData yang buruk pada \textit{benchmark} TPC-C. Terdapat sebuah komentar yang menjawab mengapa hal ini memang wajar. Selain karena CitusData membutuhkan beberapa \textit{network round-trip}, beban TPC-C lebih banyak pada \textit{query planning} dibandingkan dengan eksekusinya \parencite{Slot2020}.

Beban TPC-C dan beban penjualan tiket memiliki kemiripan, sehingga hal yang sama juga berlaku pada kasus ini. Pada penjualan tiket, beban basis data terdiri atas banyak kueri baca dan kueri tulis yang sebenarnya cukup ringan, sehingga koordinator mengalami beban yang tinggi karena harus melakukan tambahan \textit{query planning} untuk koordinasi.

Terdapat alternatif \textit{benchmark} TPC-C yang juga dibahas pada komentar tersebut, yaitu \textit{benchmark} HammerDB. HammerDB menyimpan logika pengujian pada \textit{stored procedure} PostgreSQL. Citus dapat mendelegasikan pemanggilan prosedur tersebut dengan \textit{overhead} yang jauh lebih kecil dan dengan jumlah \textit{network round-trip} yang lebih sedikit \parencite{Slot2020}. Hal ini dibahas lebih lanjut pada artikel yang membahas penggunaan \textit{distributed function} untuk latensi yang lebih baik pada CitusData versi 9 \parencite{Slot2020faster}.

Pendekatan dengan \textit{stored procedure} merupakan alternatif yang dapat dipelajari lebih lanjut pada penelitian lain untuk pengoptimalan yang lebih optimal. Meskipun begitu, permasalahan dari pendekatan ini adalah pemindahan logika bisnis dari kode aplikasi ke basis data. Selain karena keterbatasan sintaks dan dukungan integrasi, logika bisnis yang dijalankan pada level aplikasi belum tentu dapat diterjemahkan secara persis menjadi \textit{stored procedure} PostgreSQL.

\subsubsection{Pembahasan Kinerja YugabyteDB}

Sebagaimana ditunjukkan pada Tabel \ref{table:latensi-kueri}, latensi YugabyteDB memang dapat diprediksi karena mekanisme di baliknya yang menggunakan konsensus berbasiskan Raft. Setiap penulisan harus menunggu kluster mencapai konsensus terlebih dahulu, sehingga latensinya lebih tinggi. Meskipun begitu, latensi YugabyteDB yang lebih lambat hingga 100-200x dibandingkan dengan PostgreSQL merupakan hal yang tidak diprediksi.

Pada kasus yang lebih sederhana, seorang pengguna melaporkan penurunan kinerja hingga 300x lebih lambat dibandingkan dengan PostgreSQL. Meski setelah beberapa tahun berlalu telah banyak dilakukan pengoptimalan pada YugabyteDB, referensi hasil pengujian ini tetap dapat dianggap relevan sebagai validasi bahwa penurunan kinerja seperti ini memang juga terjadi pada kasus lain \parencite{yugabyteIssuePerformance}.

Permasalahan latensi ini dapat diterima apabila kinerja YugabyteDB juga diiringi dengan keunggulan dari aspek lain, seperti kemampuan menangani keadaan perebutan tiket dengan baik dan kestabilan kluster yang baik. Meskipun begitu, hal ini juga tidak dapat ditunjukkan oleh YugabyteDB. Gambar \ref{fig:metrics-f3t1} menunjukkan gambaran kinerja YugabyteDB saat berjalan dengan baik.

\begin{figure}[htbp]
    \centering
    \includegraphics[width=0.6\textwidth]{resources/chapter-4/latensi-yugabyte-success.png}
    \caption{Metrik Pemrosesan Pesanan (f3t2)}
    \label{fig:metrics-f3t1}
\end{figure}

Latensi pemrosesan diawali dengan tingkat kegagalan yang tinggi, lalu pemrosesannya mulai stabil. Setelah itu, latensi pemrosesan mulai meningkat hingga tiba-tiba turun lagi. Hasil ini merupakan hasil paling baik yang dapat diperoleh dari pengujian YugabyteDB. \textit{Tuning} lebih lanjut seperti mengubah batas koneksi dan penambahan beban membuat YugabyteDB mengalami kegagalan penuh sebagaimana terjadi pada hasil pengujian lain yang ditunjukkan pada Gambar \ref{fig:metrics-f3t2}.

\begin{figure}[htbp]
    \centering
    \includegraphics[width=0.6\textwidth]{resources/chapter-4/latensi-yugabyte-fail.png}
    \caption{Metrik Pemrosesan Pesanan (f3t2)}
    \label{fig:metrics-f3t2}
\end{figure}

Selama pengujian, terdapat beberapa kasus ketika salah satu \textit{node} YugabyteDB mati, sehingga menyebabkan kluster basis data mati dan tidak dapat menangani permintaan selama beberapa waktu. Kestabilan kluster ini juga merupakan hal yang menjadi permasalahan selama pengujian.

Setelah ditelusuri, laju pemrosesan kueri saat YugabyteDB mengalami kegagalan sebagaimana ditunjukkan pada Gambar \ref{fig:metrics-f3t2} mengalami penurunan yang cukup tajam dibandingkan saat YugabyteDB berjalan cukup baik sebagiamana ditunjukkan pada Gambar \ref{fig:metrics-f3t1}. Penyebab pasti hal ini belum dapat diidentifikasi hingga pengujian selesai dilaksanakan. Akan tetapi, dapat diasumsikan bahwa setelah melewati beban tertentu, YugabyteDB mulai kesulitan menangani beban yang diberikan, sehingga kegagalan mulai sering terjadi dan laju pemrosesan kueri menjadi lebih lambat.

Masalah lain yang berhasil diidentifikasi pada YugabyteDB ini adalah masalah manajemen koneksi. Setelah beberapa kali pengujian, tidak ditemukan kombinasi pengaturan koneksi yang memberikan kinerja yang baik, baik itu jumlah koneksi dari sisi klien ke PGCat atau pun jumlah koneksi dari PGCat ke TServer. \textit{Pooler} bawaan YugabyteDB, yaitu YSQL Connection Manager juga tidak memberikan perubahan kinerja. Selain itu, pengujian tanpa \textit{pooler} (menggunakan \textit{direct connection}) juga sudah dicoba dan tetap tidak memperoleh hasil yang baik.

Dari sisi \textit{throughput}, terdapat kasus ketika YugabyteDB memiliki hasil yang cukup baik sebagaimana ditunjukkan pada Gambar \ref{fig:yugabytedb-throughput}. Meskipun begitu, hasil ini tidak dapat direplikasi pada pengujian lain.

\begin{figure}[htbp]
    \centering
    \includegraphics[width=1\textwidth]{resources/chapter-4/yugabyte-ops.png}
    \caption{\textit{Throughput} Kueri Pada YugabyteDB}
    \label{fig:yugabytedb-throughput}
\end{figure}

Singkat cerita, YugabyteDB kesulitan menangani beban koneksi yang ada pada pengujian dan memiliki masalah pada kestabilan kluster untuk dapat digunakan secara andal. Hal ini mungkin saja berubah apabila YugabyteDB diberi lebih banyak sumber daya atau lebih banyak \textit{node}.

\subsection{Pengoptimalan Kueri Baca}

\subsubsection{Kueri Ketersediaan Area}

Operasi baca ketersediaan area merupakan \textit{endpoint} yang cukup banyak dipanggil, terutama ketika ketersediaan tiket mulai menipis dan pengguna mencoba mengeksplorasi lebih banyak area. Hal ini ditunjukkan pada Gambar \ref{fig:latency-get-area}. Pada pengujian f5t2, \textit{endpoint} ini memiliki puncak pemanggilan sebanyak 1700 rps. Sepanjang pengujian, latensi rata-rata operasi tersebut berkisar antara 2.5-4.5 milisekon. Hal ini menunjukkan bahwa pengoptimalan kueri ketersediaan area dengan menggunakan Redis merupakan pendekatan yang sangat efektif, terutama dalam meringankan beban pada basis data.

\begin{figure}[htbp]
    \centering
    \includegraphics[width=1\textwidth]{resources/chapter-4/latency-area-availability.png}
    \caption{Metrik Baca Ketersediaan Area (f5t2)}
    \label{fig:latency-get-area}
\end{figure}

\subsubsection{Kueri Ketersediaan Kursi}

Operasi baca ketersediaan kursi tidak dipanggil sebanyak operasi sebelumnya. Hal ini karena tidak semua tiket merupakan tiket dengan nomor kursi dan pengguna tidak akan memanggil operasi ini apabila tidak menemukan area yang memenuhi kriterianya. Meskipun begitu, pengoptimalan kueri ini tidak berjalan dengan cukup baik sebagaimana ditunjukkan pada Gambar \ref{fig:latency-get-seat}.

\pagebreak

\begin{figure}[htbp]
    \centering
    \includegraphics[width=1\textwidth]{resources/chapter-4/latency-seat-availability.png}
    \caption{Metrik Baca Ketersediaan Kursi (f5t2)}
    \label{fig:latency-get-seat}
\end{figure}

Rata-rata latensi (P50) memiliki nilai mulai dari 40 milisekon hingga 1.75 sekon (apabila mengesampingkan latensi di awal saat jumlah koneksi masih sedikit). Di sisi lain, latensi P95 cukup mengkhawatirkan dengan nilai mulai dari 1.6 sekon hingga 7 sekon. Hal ini menunjukkan bahwa penggunaan tembolok singkat (waktu hidup 150 milisekon) tidak begitu efektif. Hal ini terjadi karena beberapa hal berikut:

\begin{enumerate}
    \item Tembolok yang bersifat lokal pada level aplikasi.
    \item Sebaran pengguna yang membaca ID area yang berbeda-beda.
    \item Waktu hidup yang terlalu singkat.
\end{enumerate}

Kombinasi hal tersebut membuat \textit{cache hit} operasi ini sangat rendah sehingga tidak efektif. Selain itu, latensi ini menunjukkan bahwa operasi ini merupakan operasi yang cukup berat yang ditandai dengan latensi yang ada dan merupakan hal yang dapat menjadi fokus pengoptimalan di masa mendatang. Sebagaimana ditunjukkan pada pengoptimalan sebelumnya, pengoptimalan penggunaan Redis bisa menjadi alternatif yang dapat dieksplorasi lebih lanjut.

\subsection{Penggunaan Pengendalian Aliran}

Gambar \ref{fig:latency-fc} dan Gambar \ref{fig:latency-nofc} menunjukkan perbedaan latensi pemrosesan antara sistem dengan pendendalian aliran dan tanpa pengendalian aliran. Saat berjalan dengan baik, latensi pemrosesan dengan pengendalian aliran dapat turun hingga ratusan milisekon.

\begin{figure}[htbp]
    \centering
    \includegraphics[width=0.8\textwidth]{resources/chapter-4/latency-fc-pg-stress-0.png}
    \caption{Latensi Pemrosesan dengan Pengendalian Aliran}
    \label{fig:latency-fc}
\end{figure}

\begin{figure}[htbp]
    \centering
    \includegraphics[width=0.8\textwidth]{resources/chapter-4/latency-nofc-pg-stress-0.png}
    \caption{Latensi Pemrosesan tanpa Pengendalian Aliran}
    \label{fig:latency-nofc}
\end{figure}

Selanjutnya, perhatikan laju pemrosesan berdasarkan kode respons sebagaimana ditunjukkan pada Gambar \ref{fig:rps-fc-pg-stress-0} dan Gambar \ref{fig:rps-nofc-pg-stress-0}. Meskipun memiliki latensi yang lebih tinggi, laju pemrosesan puncak varian ini sama dengan laju pemrosesan puncak pada sistem referensi. Oleh karena itu, latensi yang tinggi tidak serta merta membuat laju pemrosesan lebih lambat.

\pagebreak

\begin{figure}[htbp]
    \centering
    \includegraphics[width=0.8\textwidth]{resources/chapter-4/rps-fc-pg-stress-0.png}
    \caption{Laju Pemrosesan dengan Pengendalian Aliran}
    \label{fig:rps-fc-pg-stress-0}
\end{figure}

\begin{figure}[htbp]
    \centering
    \includegraphics[width=0.8\textwidth]{resources/chapter-4/rps-nofc-pg-stress-0.png}
    \caption{Laju Pemrosesan tanpa Pengendalian Aliran}
    \label{fig:rps-nofc-pg-stress-0}
\end{figure}

Latensi yang tinggi pada sistem ini merupakan konsekuensi yang tidak diinginkan karena penggunaan RabbitMQ dan pemisahan pemrosesan dengan \textit{worker} (pengendalian aliran dengan sistem antrean). Dengan menggunakan antrean melalui RabbitMQ, sebuah pemesanan harus dikirimkan melalui RabbitMQ, menunggu diproses di \textit{worker}, diproses, dikirim kembali ke Ticket Server melalui RabbitMQ, dan Ticket Server menunggu hingga respons pemesanan diterima. Penambahan \textit{network round trip} ini yang membuat lantesi menjadi sangat tinggi. Meskipun begitu, sistem tetap memproses pesanan sebagaimana mestinya.

Kode respons 409 yang tinggi pada varian pengendalian aliran juga merupakan dampak dari implementasi sistem antrean yang kurang optimal. Dengan latensi yang tinggi, terdapat lebih banyak \textit{time window} ketika sebuah kursi seharusnya sudah dipesan tetapi pemrosesan belum sepenuhnya berhasil. Oleh karena itu, kemungkinan pengguna virtual memesan tiket yang sudah dipesan menjadi lebih tinggi. Hal ini juga yang menyebabkan laju pemrosesan yang berhasil terkadang turun. Pada sistem antrean dengan latensi yang lebih rendah, kode respons 409 seharusnya menjadi lebih berkurang.

Meskipun implementasi sistem antrean kurang optimal, terdapat beberapa hal positif yang dapat dibahas. Perhatikan Tabel \ref{table:latensi-kueri-fc-nofc} yang membandingkan latensi kueri pada varian dengan pengendalian aliran (FC) dan tanpa pengendalian aliran (No FC):

\begin{table}[h]
    \centering
    \caption{Latensi Eksekusi Kueri pada Pengendalian Aliran dalam Milisekon}
    \label{table:latensi-kueri-fc-nofc}
    \begin{tabular}{|l|l|l|}
        \hline
        \textbf{Kueri}        & \textbf{No FC} & \textbf{FC} \\
        \hline
        LockFreeStandingSeats & 8.5            & 0.98        \\
        \hline
        LockNumberedSeats     & -              & -           \\
        \hline
        InsertOrder           & 0.2            & 0.2         \\
        \hline
        InsertIssuedTiket     & 0.28           & 0.3         \\
        \hline
        InsertInvoice         & 0.12           & 0.12        \\
        \hline
    \end{tabular}
\end{table}

Tidak ada perbedaan latensi pada operasi tulis. Meskipun begitu, latensi untuk \textit{lock} kursi pada varian dengan pengendalian aliran jauh lebih kecil. Selanjutnya, mari bandingkan latensi pemrosesan pesanan saat kode respons 409 berdasarkan Gambar \ref{fig:latency-by-code-nofc} dan Gambar \ref{fig:latency-by-code-fc}.

\begin{figure}[htbp]
    \centering
    \includegraphics[width=0.7\textwidth]{resources/chapter-4/latency-by-code-nofc-pg-stress-0.png}
    \caption{Latensi Berdasarkan Status Kode tanpa Pengendalian Aliran}
    \label{fig:latency-by-code-nofc}
\end{figure}

\begin{figure}[htbp]
    \centering
    \includegraphics[width=0.7\textwidth]{resources/chapter-4/latency-by-code-fc-pg-stress-0.png}
    \caption{Latensi Berdasarkan Status Kode dengan Pengendalian Aliran}
    \label{fig:latency-by-code-fc}
\end{figure}

Latensi rata-rata pada varian dengan pengendalian aliran berkisar antara 50-100 milisekon, sedangkan latensi rata-rata pada varian tanpa pengendalian aliran berkisar antara 1000-2000 milisekon. Hal ini menunjukkan perbedaan yang signifikan karena pesanan yang pada akhirnya akan ditolak menjadi ditolak lebih awal. Latensi tanpa \textit{early dropper} lebih tinggi karena pesanan baru akan ditolak saat sudah memasuki basis data. Selain memiliki latensi yang lebih tinggi, kasus ini juga dapat membebani basis data lebih jauh.

Secara umum, pendekatan pengendalian aliran dengan menolak pesanan lebih awal cukup baik dalam mengurangi beban basis data. Di sisi lain, pada pengujian kali ini pendekatan pengendalian aliran dengan sistem antrean tidak terlalu optimal karena latensi yang tinggi. Selain itu, beban pesanan yang masuk tidak cukup tinggi sehingga tidak cukup mengganggu keberjalanan pemrosesan pesanan. Meskipun begitu, pengendalian aliran pesanan yang masuk menunjukkan dampak positif dalam menurunkan \textit{contention} yang ditunjukkan dengan penurunan latensi salah satu kueri \textit{locking}.

Implementasi antrean yang lebih baik dan beban yang lebih tinggi seharusnya dapat menunjukkan perbedaan kinerja kedua varian lebih jauh. Agar latensi lebih baik, opsi untuk tidak menggunakan RabbitMQ juga dapat digunakan. Alih-alih memisahkan komponen Ticket Server dan Ticket Worker, antrean dapat dibuat lokal per Ticket Server. Dengan begitu, tidak perlu ada tambahan \textit{network round trip} yang menambah latensi secara signifikan. Meskipun begitu, pendekatan ini harus memikirkan bagaimana penanganan \textit{idempotency} dan penanganan ketika pengguna mengirimkan ulang pesanan yang sama.

\subsection{Integritas Tiket Selama Perebutan Tiket}

Pengoptimalan baca dengan menggunakan Redis memungkinkan terjadinya ketidakcocokan data antara basis data dengan hasil pengoptimalan. Di sisi lain, sistem juga harus memastikan bahwa tidak ada kursi yang terjual lebih dari satu kali. Oleh karena itu, terdapat tiga hal yang diperiksa untuk memastikan integritas sistem tiket:

\begin{enumerate}
    \item Selisih agregat ketersediaan antara basis data dengan Redis pada akhirnya harus nol untuk pengoptimalan operasi baca ketersediaan area.
    \item Selisih ketersediaan antara basis data dengan Redis pada akhirnya harus nol untuk pengoptimalan pengendalian aliran bagian penolakan permintaan lebih awal (hanya berlaku pada varian pengendalian aliran).
    \item Tidak ada kursi yang terjual lebih dari satu kali.
\end{enumerate}

Gambar \ref{fig:sanity-f2t3}, Gambar \ref{fig:sanity-f3t1}, dan Gambar \ref{fig:sanity-f5t4} menunjukkan sampel hasil \textit{sanity check} selama pelaksanaan pengujian.

\begin{figure}[htbp]
    \centering
    \includegraphics[width=1\textwidth]{resources/chapter-4/sanity-f2t3.png}
    \caption{Hasil \textit{Sanity Check} (f2t3)}
    \label{fig:sanity-f2t3}
\end{figure}

\begin{figure}[htbp]
    \centering
    \includegraphics[width=1\textwidth]{resources/chapter-4/sanity-f3t1.png}
    \caption{Hasil \textit{Sanity Check} (f3t1)}
    \label{fig:sanity-f3t1}
\end{figure}

\begin{figure}[htbp]
    \centering
    \includegraphics[width=1\textwidth]{resources/chapter-4/sanity-f5t4.png}
    \caption{Hasil \textit{Sanity Check} (f5t4)}
    \label{fig:sanity-f5t4}
\end{figure}

Berdasarkan sampel di atas, terdapat beberapa waktu terjadinya ketidaksesuaian data yang pada akhirnya konsisten. Setelah ditelusuri, hal tersebut terjadi karena masalah pada pengumpulan data yang mengalami kegagalan memperoleh data yang lengkap. Hal ini didukung dengan fakta bahwa selisih nilainya memang sangat ekstrem. Oleh karena itu, dapat disimpulkan bahwa implementasi yang ada sekarang telah menjamin kesesuaian data dan berhasil menjamin bahwa tidak ada kursi yang terjual lebih dari satu kali.


\chapter{Penutup}

\section{Kesimpulan}

apa guys kesimpulannya?

postgres op.



\section{Saran}
Adapun banyak kekurangan dan kelemahan yang ditemukan dalam penelitian tugas akhir ini. Berikut adalah beberapa saran yang dapat dilakukan untuk pengembangan atau penelitian selanjutnya agar dapat meningkatkan kegunaan PERISAI terutama bagian proses \textit{remote deployment}.

apa aja guys sarannya?

lebih baik fokus dulu ke gimana cara biar primary fokus handle write dan read didelegate ke replica atau apalah.

setup citusdata yugabyte buat koordinator dipakein replika juga biar primary koodinator fokus ke write dan sisanya handle read.

skala pengujian yang lebih representatif.

optimization spesifik yang enable basis data itu ketimbang implementasi sama semua. biar bisa perform at their best
%---------------------------------------------------------------%

% Daftar pustaka
\printbibliography

% Setting judul lampiran
\titlespacing*{\chapter}{0pt}{0pt}{0pt}
\titlespacing*{\section}{0pt}{0pt}{*1}

% Setting judul anak lampiran
\titleformat*{\section}{\bfseries}

\appendix

\chapter{Skema Basis Data Layanan Tiket}
\label{apx:ticket-schema}

Tabel \ref{table:ticket-event-schema} menunjukkan skema setiap entitas tiket dan acara.

\begingroup
\footnotesize
\begin{longtable}{|l|p{0.2\textwidth}|p{0.5\textwidth}|}
	\caption{Skema Entitas Event dan Tiket}        
	\label{table:ticket-event-schema}                                                                                     \\
	\hline
	\textbf{Atribut}     & \textbf{Tipe Data}    & \textbf{Deskripsi}                                                                   \\
	\endfirsthead

	\multicolumn{3}{|l|}{\tablename\ \thetable\ -- \textit{Lanjutan dari halaman sebelumnya}}                                           \\
	\hline
	\textbf{Atribut}     & \textbf{Tipe Data}    & \textbf{Deskripsi}                                                                   \\
	\endhead

	\hline
	\multicolumn{3}{|r|}{\textit{Dilanjutkan ke halaman berikutnya}}                                                                    \\
	\endfoot

	\hline
	\endlastfoot

	\hline
	\multicolumn{3}{|l|}{\textbf{Event}}                                                                                                \\
	\hline
	id                   & \texttt{bigint}       & ID unik untuk setiap event. (PK)                                                     \\
	\hline
	name                 & \texttt{text}         & Nama event.                                                                          \\
	\hline
	location             & \texttt{text}         & Lokasi diselenggarakannya event.                                                     \\
	\hline
	description          & \texttt{text}         & Deskripsi detail dari event.                                                         \\
	\hline
	created\_at          & \texttt{timestamptz}  & Waktu record event dibuat.                                                           \\
	\hline
	updated\_at          & \texttt{timestamptz}  & Waktu record event terakhir diperbarui.                                              \\
	\hline
	\multicolumn{3}{|l|}{\textbf{TicketCategory}}                                                                                       \\
	\hline
	id                   & \texttt{bigint}       & ID unik untuk setiap kategori tiket. (PK)                                            \\
	\hline
	name                 & \texttt{text}         & Nama kategori tiket (misalnya, "VIP", "Reguler").                                    \\
	\hline
	event\_id            & \texttt{bigint}       & Foreign Key ke \texttt{Event.id}.                                                    \\
	\hline
	created\_at          & \texttt{timestamptz}  & Waktu record kategori tiket dibuat.                                                  \\
	\hline
	updated\_at          & \texttt{timestamptz}  & Waktu record kategori tiket terakhir diperbarui.                                     \\
	\hline
	\multicolumn{3}{|l|}{\textbf{TicketSale}}                                                                                           \\
	\hline
	id                   & \texttt{bigint}       & ID unik untuk setiap periode penjualan tiket. (PK)                                   \\
	\hline
	name                 & \texttt{text}         & Nama periode penjualan (misalnya, "Presale 1").                                      \\
	\hline
	sale\_begin\_at      & \texttt{timestamptz}  & Waktu dimulainya periode penjualan tiket.                                            \\
	\hline
	sale\_end\_at        & \texttt{timestamptz}  & Waktu berakhirnya periode penjualan tiket.                                           \\
	\hline
	event\_id            & \texttt{bigint}       & Foreign Key ke \texttt{Event.id}.                                                    \\
	\hline
	created\_at          & \texttt{timestamptz}  & Waktu record penjualan tiket dibuat.                                                 \\
	\hline
	updated\_at          & \texttt{timestamptz}  & Waktu record penjualan tiket terakhir diperbarui.                                    \\
	\hline
	\multicolumn{3}{|l|}{\textbf{TicketPackage}}                                                                                        \\
	\hline
	id                   & \texttt{bigint}       & ID unik untuk setiap paket tiket. (PK)                                               \\
	\hline
	price                & \texttt{int}          & Harga dari paket tiket.                                                              \\
	\hline
	ticket\_category\_id & \texttt{bigint}       & Foreign Key ke \texttt{TicketCategory.id}.                                           \\
	\hline
	ticket\_sale\_id     & \texttt{bigint}       & Foreign Key ke \texttt{TicketSale.id}.                                               \\
	\hline
	created\_at          & \texttt{timestamptz}  & Waktu record paket tiket dibuat.                                                     \\
	\hline
	updated\_at          & \texttt{timestamptz}  & Waktu record paket tiket terakhir diperbarui.                                        \\
	\hline
	\multicolumn{3}{|l|}{\textbf{TicketArea}}                                                                                           \\
	\hline
	id                   & \texttt{bigint}       & ID unik untuk setiap area tiket. (PK)                                                \\
	\hline
	area\_type           & \texttt{type}         & Tipe area: 'numbered-seating' (duduk bernomor) atau 'free-standing' (berdiri bebas). \\
	\hline
	ticket\_package\_id  & \texttt{bigint}       & Foreign Key ke \texttt{TicketPackage.id}.                                            \\
	\hline
	created\_at          & \texttt{timestamptz}  & Waktu record area tiket dibuat.                                                      \\
	\hline
	updated\_at          & \texttt{timestamptz}  & Waktu record area tiket terakhir diperbarui.                                         \\
	\hline
	\multicolumn{3}{|l|}{\textbf{TicketSeat}}                                                                                           \\
	\hline
	id                   & \texttt{bigint}       & ID unik untuk kursi (Bagian dari Composite PK).                                      \\
	\hline
	seat\_number         & \texttt{text}         & Nomor kursi.                                                                         \\
	\hline
	status               & \texttt{seat\_status} & Status kursi: 'available' (tersedia), 'on-hold' (ditahan), 'sold' (terjual).         \\
	\hline
	ticket\_area\_id     & \texttt{bigint}       & Foreign Key ke \texttt{TicketArea.id} (Bagian dari Composite PK).                    \\
	\hline
	created\_at          & \texttt{timestamptz}  & Waktu record kursi dibuat.                                                           \\
	\hline
	updated\_at          & \texttt{timestamptz}  & Waktu record kursi terakhir diperbarui.                                              \\
\end{longtable}
\endgroup

Tabel \ref{table:order-schema} menunjukkan skema setiap entitas pemesanan.

\begingroup
\footnotesize
\begin{longtable}{|l|p{0.2\textwidth}|p{0.5\textwidth}|}
	\caption{Skema Entitas Pemesanan}     
	\label{table:order-schema}                                                                                \\
	\hline
	\textbf{Atribut}     & \textbf{Tipe Data}       & \textbf{Deskripsi}                                                  \\
	\endfirsthead

	\multicolumn{3}{|l|}{\tablename\ \thetable\ -- \textit{Lanjutan dari halaman sebelumnya}}                             \\
	\hline
	\textbf{Atribut}     & \textbf{Tipe Data}       & \textbf{Deskripsi}                                                  \\
	\endhead

	\hline
	\multicolumn{3}{|r|}{\textit{Dilanjutkan ke halaman berikutnya}}                                                      \\
	\endfoot

	\hline
	\endlastfoot

	\hline
	\multicolumn{3}{|l|}{\textbf{Order}}                                                                                  \\
	\hline
	id                   & \texttt{bigint}          & ID unik untuk setiap pesanan (Bagian dari Composite PK).            \\
	\hline
	status               & \texttt{order\_status}   & Status pesanan: 'waiting-for-payment', 'failed', 'success'.         \\
	\hline
	fail\_reason         & \texttt{text}            & Alasan jika status pesanan 'failed'.                                \\
	\hline
	event\_id            & \texttt{bigint}          & Foreign Key ke \texttt{Event.id}.                                   \\
	\hline
	ticket\_sale\_id     & \texttt{bigint}          & Foreign Key ke \texttt{TicketSale.id}.                              \\
	\hline
	ticket\_area\_id     & \texttt{bigint}          & Foreign Key ke \texttt{TicketArea.id} (Bagian dari Composite PK).   \\
	\hline
	external\_user\_id   & \texttt{text}            & Foreign Key ke \texttt{User.external\_user\_id}.                    \\
	\hline
	created\_at          & \texttt{timestamptz}     & Waktu record pesanan dibuat.                                        \\
	\hline
	updated\_at          & \texttt{timestamptz}     & Waktu record pesanan terakhir diperbarui.                           \\
	\hline
	\multicolumn{3}{|l|}{\textbf{OrderItem}}                                                                              \\
	\hline
	id                   & \texttt{bigint}          & ID unik untuk setiap item dalam pesanan (Bagian dari Composite PK). \\
	\hline
	customer\_name       & \texttt{text}            & Nama pelanggan untuk item tiket ini.                                \\
	\hline
	customer\_email      & \texttt{text}            & Alamat surel pelanggan untuk item tiket ini.                        \\
	\hline
	price                & \texttt{int}             & Harga yang dibayarkan untuk item ini.                               \\
	\hline
	order\_id            & \texttt{bigint}          & Foreign Key ke \texttt{Order.id}.                                   \\
	\hline
	ticket\_category\_id & \texttt{bigint}          & Foreign Key ke \texttt{TicketCategory.id}.                          \\
	\hline
	ticket\_seat\_id     & \texttt{bigint}          & Foreign Key ke \texttt{TicketSeat.id}.                              \\
	\hline
	ticket\_area\_id     & \texttt{bigint}          & Foreign Key ke \texttt{TicketArea.id} (Bagian dari Composite PK).   \\
	\hline
	created\_at          & \texttt{timestamptz}     & Waktu record item pesanan dibuat.                                   \\
	\hline
	updated\_at          & \texttt{timestamptz}     & Waktu record item pesanan terakhir diperbarui.                      \\
	\hline
	\multicolumn{3}{|l|}{\textbf{Invoice}}                                                                                \\
	\hline
	id                   & \texttt{bigint}          & ID unik untuk setiap invoice (Bagian dari Composite PK).            \\
	\hline
	status               & \texttt{invoice\_status} & Status invoice: 'pending', 'expired', 'failed', 'paid'.             \\
	\hline
	amount               & \texttt{int}             & Total jumlah yang harus dibayar pada invoice.                       \\
	\hline
	external\_id         & \texttt{text}            & ID eksternal dari penyedia layanan pembayaran.                      \\
	\hline
	order\_id            & \texttt{bigint}          & Foreign Key ke \texttt{Order.id}.                                   \\
	\hline
	ticket\_area\_id     & \texttt{bigint}          & Foreign Key ke \texttt{TicketArea.id} (Bagian dari Composite PK).   \\
	\hline
	created\_at          & \texttt{timestamptz}     & Waktu record invoice dibuat.                                        \\
	\hline
	updated\_at          & \texttt{timestamptz}     & Waktu record invoice terakhir diperbarui.                           \\
	\hline
	\multicolumn{3}{|l|}{\textbf{IssuedTicket}}                                                                           \\
	\hline
	id                   & \texttt{bigint}          & ID unik untuk tiket yang diterbitkan (Bagian dari Composite PK).    \\
	\hline
	serial\_number       & \texttt{text}            & Nomor seri unik pada tiket yang diterbitkan.                        \\
	\hline
	holder\_name         & \texttt{text}            & Nama pemegang tiket.                                                \\
	\hline
	name                 & \texttt{text}            & Nama tiket (misalnya, nama kategori tiket).                         \\
	\hline
	description          & \texttt{text}            & Deskripsi pada tiket.                                               \\
	\hline
	ticket\_seat\_id     & \texttt{bigint}          & Foreign Key ke \texttt{TicketSeat.id}.                              \\
	\hline
	order\_id            & \texttt{bigint}          & Foreign Key ke \texttt{Order.id}.                                   \\
	\hline
	order\_item\_id      & \texttt{bigint}          & Foreign Key ke \texttt{OrderItem.id}.                               \\
	\hline
	ticket\_area\_id     & \texttt{bigint}          & Foreign Key ke \texttt{TicketArea.id} (Bagian dari Composite PK).   \\
	\hline
	created\_at          & \texttt{timestamptz}     & Waktu record tiket diterbitkan.                                     \\
	\hline
	updated\_at          & \texttt{timestamptz}     & Waktu record tiket terakhir diperbarui.                             \\
\end{longtable}
\endgroup

\chapter{Rancangan Layanan Pembayaran}
\label{apx:payment-service}

Layanan pembayaran berpotensi menjadi sumber \textit{bottleneck} saat proses pembuatan tagihan, terlebih lagi karena komunikasi pembuatan tagihan harus dilakukan secara sinkron. Untuk memastikan laju pemrosesan yang tinggi, layanan ini akan menggunakan \textit{in-memory database} dengan persistensi seperti Redis. Redis akan dikonfigurasikan dalam mode kluster untuk kebutuhan pemartisian dan beberapa penulis. Persistensi dan \textit{snapshot} masih akan dikonfigurasikan, meski tetap akan ada periode waktu yang bisa mengakibatkan terjadinya hilangnya (kurang dari satu detik).

Ketika pembayaran berhasil atau kedaluwarsa, layanan ini akan memanggil \textit{webhook} pada layanan tiket. Untuk memastikan pemberitahuan terkirim, layanan ini akan mengulangi pemberitahuan ketika terjadi kegagalan saat pemanggilan \textit{webhook}. Komponen pengatur waktu untuk kedaluwarsa dan antrean untuk pemanggilan \textit{webhook} juga akan menggunakan Redis.

\section{Komponen Layanan}

Komponen pada layanan ini akan dibagi menjadi dua, yaitu pemroses pembayaran dan komponen notifikasi. Pemroses pembayaran merupakan komponen yang melayani pembuatan tagihan dan pembayaran tagihan. Komponen notifikasi merupakan komponen yang menangani pengatur waktu ketika terdapat pembayaran yang kedaluwarsa dan  memanggil \textit{webhook} ketika pembayaran berhasil atau kedaluwarsa. Penskalaan setiap komponen dapat dilakukan secara horizontal. Arsitektur layanan pembayaran ditunjukkan pada Gambar \ref{fig:payment-service-deployment}.

\pagebreak

\begin{figure}[htbp]
    \centering
    \includegraphics[width=0.65\textwidth]{resources/chapter-3/payment-service.png}
    \caption{Arsitektur Layanan Pembayaran}
    \label{fig:payment-service-deployment}
\end{figure}

\section{Alur Layanan}

Gambar \ref{fig:payment-flow1} menunjukkan alur sistem pembayaran untuk fitur pembuatan tagihan, pembayaran tagihan, dan baca data tagihan.

\begin{figure}[htbp]
    \centering
    \includegraphics[width=0.75\textwidth]{resources/chapter-3/payment-flow1.png}
    \caption{Diagram Alur Pembuatan, Pembayaran dan Baca Tagihan}
    \label{fig:payment-flow1}
\end{figure}

\pagebreak

Selanjutnya, Gambar \ref{fig:payment-flow2} menunjukkan alur pemrosesan \textit{webhook} untuk pembayaran pemesanan yang berhasil atau pun gagal. Skema ini menggunakan strategi \textit{exponential backoff} untuk menangani kasus pemberitahuan yang gagal.

\begin{figure}[htbp]
    \centering
    \includegraphics[width=0.8\textwidth]{resources/chapter-3/payment-flow2.png}
    \caption{Diagram Alur Pemrosesan \textit{Webhook}}
    \label{fig:payment-flow2}
\end{figure}

\chapter{Distribusi Profil Lengkap}


\begingroup
\footnotesize
\begin{longtable}{|l|l|l|l|l|}
    \caption{Distribusi Profil Lengkap}                                                                                   \\
    \hline
    \textbf{Preferensi Hari} & \textbf{Tingkatan Kursi} & \textbf{Kuantitas} & \textbf{Persistence} & \textbf{Persentase} \\
    \hline
    \endfirsthead

    \multicolumn{5}{|l|}{\tablename\ \thetable\ -- \textit{Lanjutan dari halaman sebelumnya}}                             \\
    \hline
    \textbf{Preferensi Hari} & \textbf{Tingkatan Kursi} & \textbf{Kuantitas} & \textbf{Persistence} & \textbf{Persentase} \\
    \hline
    \endhead

    \hline
    \multicolumn{5}{|r|}{\textit{Dilanjutkan ke halaman berikutnya}}                                                      \\
    \endfoot

    \hline
    \endlastfoot

    Specific                 & Seated-Low               & Solo               & Medium               & 1.00\%              \\
    \hline
    Specific                 & Seated-Low               & Solo               & High                 & 2.00\%              \\
    \hline
    Specific                 & Seated-Low               & Couple             & Low                  & 2.00\%              \\
    \hline
    Specific                 & Seated-Low               & Couple             & Medium               & 3.00\%              \\
    \hline
    Specific                 & Seated-Low               & Couple             & High                 & 2.00\%              \\
    \hline
    Specific                 & Seated-Low               & Group              & Low                  & 2.00\%              \\
    \hline
    Specific                 & Seated-Low               & Group              & Medium               & 2.00\%              \\
    \hline
    Specific                 & Seated-Mid               & Solo               & Medium               & 1.00\%              \\
    \hline
    Specific                 & Seated-Mid               & Solo               & High                 & 2.00\%              \\
    \hline
    Specific                 & Seated-Mid               & Couple             & Low                  & 2.00\%              \\
    \hline
    Specific                 & Seated-Mid               & Couple             & Medium               & 2.00\%              \\
    \hline
    Specific                 & Seated-Mid               & Couple             & High                 & 2.00\%              \\
    \hline
    Specific                 & Seated-Mid               & Group              & Low                  & 3.00\%              \\
    \hline
    Specific                 & Seated-Mid               & Group              & Medium               & 2.00\%              \\
    \hline
    Specific                 & Seated-High              & Solo               & Medium               & 2.00\%              \\
    \hline
    Specific                 & Seated-High              & Solo               & High                 & 2.00\%              \\
    \hline
    Specific                 & Seated-High              & Couple             & Medium               & 2.00\%              \\
    \hline
    Specific                 & Seated-High              & Couple             & High                 & 2.00\%              \\
    \hline
    Specific                 & Seated-High              & Group              & Medium               & 2.00\%              \\
    \hline
    Specific                 & Area-Mid                 & Solo               & Low                  & 1.00\%              \\
    \hline
    Specific                 & Area-Mid                 & Solo               & Medium               & 2.00\%              \\
    \hline
    Specific                 & Area-Mid                 & Couple             & Low                  & 1.00\%              \\
    \hline
    Specific                 & Area-Mid                 & Couple             & Medium               & 2.00\%              \\
    \hline
    Specific                 & Area-Mid                 & Couple             & High                 & 1.00\%              \\
    \hline
    Specific                 & Area-Mid                 & Group              & Medium               & 2.00\%              \\
    \hline
    Specific                 & Area-High                & Solo               & Medium               & 2.00\%              \\
    \hline
    Specific                 & Area-High                & Solo               & High                 & 1.00\%              \\
    \hline
    Specific                 & Area-High                & Couple             & Medium               & 1.00\%              \\
    \hline
    Specific                 & Area-High                & Couple             & High                 & 1.00\%              \\
    \hline
    Any Day                  & Seated-Low               & Solo               & Medium               & 2.00\%              \\
    \hline
    Any Day                  & Seated-Low               & Couple             & Low                  & 3.00\%              \\
    \hline
    Any Day                  & Seated-Low               & Couple             & Medium               & 3.00\%              \\
    \hline
    Any Day                  & Seated-Low               & Couple             & High                 & 3.00\%              \\
    \hline
    Any Day                  & Seated-Low               & Group              & Medium               & 2.00\%              \\
    \hline
    Any Day                  & Seated-Mid               & Solo               & Medium               & 2.00\%              \\
    \hline
    Any Day                  & Seated-Mid               & Couple             & Low                  & 3.00\%              \\
    \hline
    Any Day                  & Seated-Mid               & Couple             & Medium               & 1.00\%              \\
    \hline
    Any Day                  & Seated-Mid               & Couple             & High                 & 2.00\%              \\
    \hline
    Any Day                  & Seated-Mid               & Group              & Medium               & 4.00\%              \\
    \hline
    Any Day                  & Seated-High              & Solo               & Medium               & 2.00\%              \\
    \hline
    Any Day                  & Seated-High              & Solo               & High                 & 2.00\%              \\
    \hline
    Any Day                  & Seated-High              & Group              & Medium               & 3.00\%              \\
    \hline
    Any Day                  & Seated-High              & Couple             & High                 & 2.00\%              \\
    \hline
    Any Day                  & Area-Mid                 & Solo               & Medium               & 2.00\%              \\
    \hline
    Any Day                  & Area-Mid                 & Couple             & Low                  & 2.00\%              \\
    \hline
    Any Day                  & Area-Mid                 & Couple             & High                 & 2.00\%              \\
    \hline
    Any Day                  & Area-Mid                 & Group              & Medium               & 2.00\%              \\
    \hline
    Any Day                  & Area-High                & Solo               & High                 & 2.00\%              \\
    \hline
    Any Day                  & Area-High                & Couple             & Medium               & 2.00\%              \\
    \hline
    Any Day                  & Area-High                & Couple             & High                 & 2.00\%              \\
    \hline
\end{longtable}
\endgroup

\chapter{Kakas Implementasi}
\label{apx:tools-used}

Tabel \ref{tab:kakas-implementasi-grouped} menampilkan daftar kakas esensial yang digunakan selama implementasi. Tabel ini hanya melampirkan kakas yang relevan, sehingga tidak semua kakas yang digunakan masuk ke daftar.

\begingroup
\footnotesize
% The table now has 3 columns: Kakas, Versi, Deskripsi
\begin{longtable}{|l|l|p{0.5\textwidth}|}
    \caption{Daftar Kakas Implementasi Sistem Tiket}
    \label{tab:kakas-implementasi-grouped} \\
    \hline
    % Header for the first page
    \textbf{Kakas (Alat)} & \textbf{Versi} & \textbf{Deskripsi} \\
    \hline
    \endfirsthead

    % Header for continuation pages
    \multicolumn{3}{|l|}{\tablename\ \thetable\ -- \textit{Lanjutan dari halaman sebelumnya}} \\
    \hline
    \textbf{Kakas (Alat)} & \textbf{Versi} & \textbf{Deskripsi} \\
    \hline
    \endhead

    % Footer for pages with continuation
    \hline
    \multicolumn{3}{|r|}{\textit{Dilanjutkan ke halaman berikutnya}} \\
    \endfoot

    % Final footer of the table
    \hline
    \endlastfoot

    % Category: Layanan Pembayaran
    \hline
    \multicolumn{3}{|l|}{\textbf{Layanan Pembayaran}} \\
    \hline
    NodeJS & 20.19.1 & \textit{Runtime} JavaScript. \\
    TypeScript & 5.8.2 & Bahasa pemrograman. \\
    Hono & 4.7.4 & Kerangka kerja HTTP. \\
    BullMQ & 5.41.8 & \textit{Wrapper} untuk MQ di atas Redis. \\
    \hline

    % Category: Layanan Tiket
    \multicolumn{3}{|l|}{\textbf{Layanan Tiket}} \\
    \hline
    Golang & 1.24.0 & Bahasa pemrograman. \\
    Echo & 4.13.3 & Kerangka kerja HTTP. \\
    Echo-Prometheus & 0.17.4 & Untuk integrasi Echo dengan Prometheus. \\
    TestContainers & 0.36.0 & Untuk pengujian terintegrasi. \\
    Go Concurrency Limits & 0.8.0 & Algoritma pengendalian aliran (adaptasi dari Netflix untuk kakas yang sama). \\
    Jackc PGX & 5.7.2 & Kakas \textit{driver} untuk PostgreSQL. \\
    go-cache & 2.1.0 & Untuk tembolok di memori. \\
    UberFX & 1.23.0 & Untuk \textit{dependency injection}. \\
    TypeSpec & 0.67.2 & Untuk membuat dokumentasi OpenAPI. \\
    \hline

    % Category: Kakas Pengawasan
    \multicolumn{3}{|l|}{\textbf{Kakas Pengawasan}} \\
    \hline
    Grafana Loki & - & Untuk agregasi \textit{log}. \\
    Grafana Alloy & - & Untuk koleksi \textit{log}. \\
    Grafana Dashboard & - & Untuk dasbor visualisasi. \\
    Prometheus & - & Untuk koleksi metrik. \\
    Prometheus exporter & - & \textit{Exporter} spesifik (contoh: PostgreSQL, Node, Pod). \\
    \hline

    % Category: Agen Penguji
    \multicolumn{3}{|l|}{\textbf{Agen Penguji}} \\
    \hline
    Grafana K6 & - & Untuk pengujian beban. \\
    Grafana K6 Operator & - & Pengujian terdistribusi di kluster Kubernetes. \\
    \hline

    % Category: Lainnya
    \multicolumn{3}{|l|}{\textbf{Lainnya}} \\
    \hline
    Kubectl & - & CLI API untuk kluster Kubernetes. \\
    Kubectx & - & Beralih antar konfigurasi kluster Kubernetes. \\
    Helm Package Manager & - & Untuk \textit{deployment} Kubernetes Helm Chart. \\
    Helmfile & - & Konfigurasi Helm Chart secara deklaratif. \\

\end{longtable}
\endgroup


\chapter{Implementasi Layanan Pembayaran (Tambahan)}
\label{apx:payment-implementation}

Layanan ini dibuat dengan bahasa pemrograman TypeScript yang dijalankan pada NodeJS versi 20. \textit{Runtime} layanan ini dibagi menjadi dua, yakni HTTP Server dan Notifier. Selain itu, BullMQ dan Redis digunakan untuk menyimpan data dan melakukan proses antrean notifikasi webhook kepada sistem tiket.

Layanan ini merupakan \textit{mock service} sebuah gerbang pembayaran, sehingga hanya fungsionalitas dasar saja yang diimplementasikan, seperti membuat tagihan, membayar tagihan, dan melihat tagihan. Oleh karena itu, pada layanan ini digunakan Redis sebagai basis data untuk menyimpan data tagihan karena implementasi dan konfigurasi yang lebih mudah.

Agar sistem menyerupai gerbang pembayaran, layanan ini juga menggunakan HMAC dengan rahasia bersama pada isi data webhook yang dikirimkan pada sistem tiket. Kemudian, sistem tiket akan memverifikasi nilai yang diterima dan membandingkannya dengan isi data yang diterima.

Pemisahan server dan Notifier ini dilakukan untuk memisahkan beban dan mengimplementasikan mekanisme perulangan ketika webhook gagal. Selain itu, penggunaan Notifier memungkinkan pengedaluwarsaan tagihan yang melewati batas kedaluwarsa, sehingga sistem dapat melepaskan status pemesanan pada tiket yang tidak jadi terjual.

Terdapat \textit{endpoint} \textit{healthcheck} yang memeriksa koneksi sistem dengan kluster Redis. Integrasi Hono dengan Prometheus juga dilakukan untuk proses monitoring berbagai metrik HTTP Server dan jumlah antrean pada Notifier.

\textit{Webhook} akan dipanggil saat pembayaran berhasil, gagal, dan kedaluwarsa. Layanan ini akan memanggil \textit{endpoint} yang sudah ditentukan sebelumnya untuk memberi notifikasi \textit{webhook} yang berisi data tagihan paling terbaru.

\end{document}
