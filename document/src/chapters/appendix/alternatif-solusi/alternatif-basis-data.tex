\section{Alternatif Basis Data Terdistribusi}

Selain Citus, terdapat berbagai alternatif basis data terdistribusi lain, seperti Vitess, TiDB, YugabyteDB, dan CockroachDB. Berikut adalah perbandingan masing-masing solusi \parencite{citus,vitess,tiDB,yugabyte,cockroachDB}:

\begingroup
\footnotesize
\begin{longtable}{|p{0.14\textwidth}|p{0.14\textwidth}|p{0.14\textwidth}|p{0.14\textwidth}|p{0.14\textwidth}|p{0.14\textwidth}|}
    \caption{Perbandingan Antara Citus, Vitess, TiDB, YugabyteDB, dan CockroachDB}                                                                                                                                                                                                                    \\
    \hline
    \textbf{Aspek}            & \textbf{Citus}                                               & \textbf{Vitess}                                     & \textbf{TiDB}                                  & \textbf{YugabyteDB}                            & \textbf{CockroachDB}                           \\
    \hline
    \endfirsthead

    \multicolumn{6}{|c|}{\tablename\ \thetable\ -- \textit{Lanjutan dari halaman sebelumnya}}                                                                                                                                                                                                         \\
    \hline
    \textbf{Aspek}            & \textbf{Citus}                                               & \textbf{Vitess}                                     & \textbf{TiDB}                                  & \textbf{YugabyteDB}                            & \textbf{CockroachDB}                           \\
    \hline
    \endhead

    \hline
    \multicolumn{6}{|r|}{\textit{Dilanjutkan ke halaman berikutnya}}                                                                                                                                                                                                                                  \\
    \endfoot

    \hline
    \endlastfoot

    \hline
    Basis data yang mendasari & PostgreSQL                                                   & MySQL                                               & Dibuat dari awal                               & Dibuat dari awal                               & Dibuat dari awal                               \\
    \hline
    \hline
    Arsitektur                & \textit{Sharded Multi-Master with a Coordinator}             & \textit{Sharded Multi-Master with a Coordinator}    & \textit{Multi-Master with Shared Nothing}      & \textit{Multi-Master with Shared Nothing}      & \textit{Multi-Master with Shared Nothing}      \\
    \hline
    \hline
    Tipe                      & Ekstensi PostgreSQL                                          & Ekstensi MySQL                                      & Basis data terdistribusi dengan konsensus Raft & Basis data terdistribusi dengan konsensus Raft & Basis data terdistribusi dengan konsensus Raft \\
    \hline
    \hline
    Kompatibilitas SQL        & PostgreSQL                                                   & MySQL                                               & Kompatibel dengan MySQL 8.0                    & Kompatibel dengan PostgreSQL                   & Kompatibel dengan PostgreSQL                   \\
    \hline
    \hline
    Konsistensi               & Sama seperti PostgreSQL (konsisten dalam satu \textit{node}) & \textit{Eventual consistent} untuk operasi tertentu & ACID                                           & ACID                                           & ACID                                           \\
    \hline
    \hline
    Dukungan CDC              & Ada                                                          & Ada                                                 & Ada                                            & Ada                                            & Ada                                            \\
    \hline
    \hline
    Distribusi Data           & \textit{shard}                                               & \textit{shard}                                      & \textit{native distributed}                    & \textit{native distributed}                    & \textit{native distributed}                    \\
    \hline
\end{longtable}
\endgroup

Distribusi data yang \textit{natively distributed} masih sama-sama berupa \textit{sharding} data. Meskipun begitu, proses \textit{sharding} ini dilakukan secara otomatis dan tidak ditentukan oleh pengguna. Selain itu, suatu \textit{shard} dapat dipegang oleh beberapa \textit{node} sekaligus untuk mencapai \textit{redundancy} dan \textit{fault tolerance}.

Pada basis data terdistribusi seperti YugabyteDB, CockroachDB, dan TiDB, terdapat \textit{overhead} dalam koordinasi antarnode untuk operasi tulis, terlebih lagi setiap operasi tulis harus mencapai konsensus terlebih dahulu. Berbeda dengan basis data terdistribusi yang memakai koordinator seperti Citus dan Vitess, \textit{overhead} lebih kecil dan berada pada koordinator saja. Setelah koordinator menentukan node yang bertanggung jawab atas operasi tersebut, operasi langsung diarahkan kepada node yang berkaitan. Pendekatan konsensus memang memberikan konsistensi dan \textit{failover} yang lebih baik, tetapi terdapat \textit{tradeoff} berupa latensi yang lebih tinggi. Penggunaan basis data terdistribusi dengan konsensus akan lebih \textit{desirable} apabila melayani pengguna dalam beberapa \textit{region} yang berbeda. Meskipun begitu, penelitian ini berfokus pada pengoptimalan untuk satu \textit{region} yang sama.

Berdasarkan pembahasan di atas, terdapat dua pilihan yang tersisa, yaitu Vitess dan Citus. Citus dipilih agar basis \textit{database} yang digunakan sama, sehingga perbandingan antar arsitektur lebih disebabkan karena perbedaan arsitektur dan bukan karena perbedaan pengoptimalan pada basis data yang berbeda. Selain itu, basis data PostgreSQL merupakan basis data yang paling familiar dengan penulis dibandingan dengan MySQL.
