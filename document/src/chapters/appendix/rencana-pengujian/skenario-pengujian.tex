\section{Skenario Pengujian}

Terdapat beberapa variasi skenario yang dapat dijadikan skenario pengujian, sebagaimana dijelaskan pada tabel berikut:


\begingroup
\footnotesize
\begin{longtable}{|p{0.3\textwidth}|p{0.6\textwidth}|}
  \caption{Variasi Skenario Pengujian}                                                                                                                                                                                                                                                                                       \\
  \hline
  \textbf{Parameter}                 & \textbf{Variasi}                                                                                                                                                                                                                                                                      \\
  \hline
  \endfirsthead

  \multicolumn{2}{|l|}{\tablename\ \thetable\ -- \textit{Lanjutan dari halaman sebelumnya}}                                                                                                                                                                                                                                  \\
  \hline
  \textbf{Parameter}                 & \textbf{Variasi}                                                                                                                                                                                                                                                                      \\
  \hline
  \endhead

  \hline
  \multicolumn{2}{|r|}{\textit{Dilanjutkan ke halaman berikutnya}}                                                                                                                                                                                                                                                           \\
  \endfoot

  \hline
  \endlastfoot

  \hline
  Tipe Tiket yang Dijual             & Penjualan tiket dengan pemilihan \textit{seat} secara langsung dan pemilihan berdasarkan area.                                                                                                                                                                                        \\
  \hline
  \hline
  Arsitektur PGP                     & Pada arsitektur PGP, pengujian dilakukan dengan berbagai variasi pengoptimalan, seperti arsitektur PGP sepenuhnya, tanpa Citus, tanpa sistem antrean, serta tanpa RisingWave.                                                                                                         \\
  \hline
  \hline
  Sumber daya sistem                 & Pengujian dilakukan dengan variasi sumber daya sistem, seperti satu kali lipat, dua kali lipat, hingga beberapa kali lipat banyaknya dari satuan sumber daya tertentu.                                                                                                                \\
  \hline
  \hline
  Banyaknya agen dan banyaknya tiket & Pengujian dilakukan dengan variasi jumlah agen dan tiket yang dijual, seperti sepuluh ribu agen, seratus ribu agen, lima puluh ribu tiket, dan seratus ribu tiket.                                                                                                                    \\
  \hline
  \hline
  Rasio pengguna dan banyaknya tiket & Setiap penjualan tiket memiliki rasio tertentu antara banyaknya tiket yang dijual dengan banyaknya peminat.                                                                                                                                                                           \\
  \hline
  \hline
  Durasi Penjualan                   & Pada kenyataannya, seluruh tiket tidak akan dipesan dalam kurun waktu beberapa menit, tetapi beban tersebar dalam durasi waktu tertentu. Setiap skenario pengujian akan memiliki target waktu selesainya proses penjualan, seperti penjualan seratus ribu tiket dalam waktu satu jam. \\
  \hline
\end{longtable}
\endgroup


Acara dengan peminat tinggi besar kemungkinan dilaksanakan selama beberapa hari. Untuk itu, setiap skenario acara yang diuji akan menggunakan asumsi acara dilaksanakan selama tiga hari berturut-turut.

Berikut adalah referensi besarnya peminat dan jumlah tiket yang dijual pada acara Taylor Swift The Eras Tour yang dilaksanakan pada tahun 2023 hingga 2024:

\begin{enumerate}
  \item Di Amerika Serikat, terdapat 14 juta peminat dengan 625 ribu tiket yang akan dijual \parencite{USTaylorSwift}.
  \item Di Singapura, terdapat 22 juta peminat dengan 300 ribu tiket yang akan dijual \parencite{asiaTaylorSwift}.
\end{enumerate}

Penjelasan variasi pengujian dan skenario di atas merupakan gambaran pengujian yang akan dilakukan ketika pelaksanaan. Saat ini, belum ada skenario pengujian dengan jumlah beban dan variasi sumber daya yang eksak.