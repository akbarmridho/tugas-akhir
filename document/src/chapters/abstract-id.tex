\clearpage
\chapter*{ABSTRAK}
\addcontentsline{toc}{chapter}{ABSTRAK}
\begin{center}
  \center
  \begin{singlespace}
    \large\bfseries\MakeUppercase{\thetitle}

    \normalfont\normalsize
    Oleh:

    \bfseries \theauthor
  \end{singlespace}
\end{center}

\begin{singlespace}
  \small
  Proses penjualan tiket acara seperti konser yang tinggi peminat tidak selalu berjalan mulus. Karakteristik unik dari masalah ini adalah bahwa penjualan untuk satu kursi sangat diperebutkan. Selain itu, sistem harus melayani kueri permintaan ketersediaan tiket dalam jumlah yang sangat banyak. Di sisi lain, stabilitas sistem juga merupakan aspek yang cukup penting untuk menjaga keberlangsungan penjualan tiket. Oleh karena itu, penelitian ini bertujuan untuk mengeksplorasi pengoptimalan apa saja yang cocok untuk sistem ini. Pengoptimalan pertama adalah peningkatan penskalaan laju transaksi dengan menggunakan basis data relasional terdistribusi, seperti CitusData dan YugabyteDB. Dasar acuan sistem ini adalah kluster PostgreSQL. Selanjutnya, pengendalian aliran pemrosesan pesanan dilakukan dengan menolak permintaan yang memenuhi kriteria tertentu. Pesanan kemudian diproses secara asinkron dengan maksimal pemrosesan pemasanan pada satu waktu. Pada pengujian beban dengan 10 ribu hingga 15 ribu pengguna virtual selama 10 hingga 15 menit, kluster PostgreSQL menunjukkan hasil yang sangat baik dengan latensi rendah dan penggunaan sumber daya yang efisien. CitusData memiliki hasil yang dapat diterima dengan latensi dan penggunaan sumber daya yang lebih tinggi. YugabyteDB memiliki hasil yang buruk dengan penggunaan sumber daya yang jauh lebih tinggi dan tingkat kegagalan yang tidak dapat diterima. Di sisi lain, skema pengendalian aliran varian kluster PostgreSQL berjalan dengan baik pada beban yang sama. Penolakan permintaan pemesanan lebih awal mengurangi beban pada basis data, terutama pada kasus pesanan yang ditolak karena sudah dipesan terlebih dahulu oleh pengguna lain. Penggunaan antrean menunjukkan latensi yang lebih rendah untuk kueri \textit{locking}, meski implementasinya membuat latensi jauh lebih tinggi dan beban pengujian tidak cukup tinggi untuk menunjukkan perbedaan yang signifikan.

  \textbf{\textit{Kata kunci: Sistem Tiket, Kluster PostgreSQL, CitusData, YugabyteDB, Pengendalian Aliran}}

\end{singlespace}
\clearpage