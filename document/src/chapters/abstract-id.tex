\clearpage
\chapter*{ABSTRAK}
\addcontentsline{toc}{chapter}{ABSTRAK}
\begin{center}
  \center
  \begin{singlespace}
    \large\bfseries\MakeUppercase{\thetitle}

    \normalfont\normalsize
    Oleh:

    \bfseries \theauthor
  \end{singlespace}
\end{center}

\begin{singlespace}
  \small
  Penjualan tiket untuk acara populer seperti konser sering kali menghadapi tantangan teknis. Karakteristik utamanya adalah jumlah peminat yang jauh melebihi ketersediaan tiket, sehingga sistem harus menangani lonjakan kueri ketersediaan secara masif sambil menjaga stabilitas. Penelitian ini bertujuan untuk menganalisis dan membandingkan strategi arsitektur untuk mengoptimalkan sistem tersebut. Tiga pendekatan utama dieksplorasi: pertama, perbandingan kinerja basis data relasional terdistribusi (CitusData dan YugabyteDB) dengan kluster PostgreSQL sebagai tolok ukur. Kedua, pengoptimalan operasi baca ketersediaan tiket per area dengan menyimpan data agregat pada Redis. Ketiga, evaluasi skema pengendalian aliran yang terdiri dari penolakan permintaan lebih awal dan penggunaan antrean untuk membatasi pemrosesan pesanan secara serentak guna menjaga stabilitas basis data. Pada pengujian beban dengan 15 ribu pengguna virtual, kluster PostgreSQL menunjukkan kinerja paling baik, mencapai laju pemrosesan 466 rps dengan latensi P50 192-382 ms dan penggunaan sumber daya yang efisien. CitusData memberikan hasil yang dapat diterima, tetapi dengan latensi sekitar 2x lebih tinggi dan konsumsi sumber daya yang lebih besar. YugabyteDB menunjukkan kinerja yang tidak memadai, dengan laju pemrosesan setengah dari PostgreSQL, penggunaan CPU 2.4x dan memori 10x lebih banyak, serta tingkat kegagalan yang tinggi. Pengoptimalan operasi baca ketersediaan tiket menggunakan Redis terbukti sangat efektif, mampu melayani permintaan puncak hingga 1.700 rps dengan latensi rata-rata 2.5-4.5 ms, yang secara signifikan mengurangi beban pada basis data. Di sisi lain, skema pengendalian aliran pada varian terbukti bermanfaat. Strategi penolakan permintaan lebih awal berhasil menurunkan latensi untuk pesanan yang ditolak dari $>$1000 ms menjadi $<$100 ms. Implementasi antrean membuat latensi rata-rata mencapai 10 detik karena \textit{overhead} komunikasi dan beban pengujian tidak cukup tinggi untuk menunjukkan perbedaan kinerja.

  \textbf{\textit{Kata kunci: Sistem Tiket, Kluster PostgreSQL, CitusData, YugabyteDB, Pengendalian Aliran}}

\end{singlespace}
\clearpage