\clearpage
\chapter*{ABSTRACT}
\addcontentsline{toc}{chapter}{ABSTRACT}

\begin{center}
    \center
    \begin{singlespace}
        \large\bfseries\MakeUppercase{A Comparative Analysis of Architectural Strategies for Performance Optimization of a Large-Scale Event Ticketing System}

        \normalfont\normalsize
        By:

        \bfseries \theauthor
    \end{singlespace}
\end{center}


\begin{singlespace}
    \small
    Ticket sales for popular events like concerts often face technical challenges. The main characteristic is that the number of interested people far exceeds the availability of tickets, so the system must handle a massive surge in availability queries while maintaining stability. This study aims to analyze and compare architectural strategies to optimize such systems. Three main approaches were explored: first, comparing the performance of distributed relational databases (CitusData and YugabyteDB) with a PostgreSQL cluster as a benchmark. Second, optimizing ticket availability read operations per area by storing aggregate data in Redis. Third, evaluating a flow control scheme consisting of early request rejection and the use of queues to limit concurrent order processing to maintain database stability. In load testing with 15 thousand virtual users, the PostgreSQL cluster showed the best performance, achieving a processing rate of 466 rps with a P50 latency of 192-382 ms and efficient resource usage. CitusData provided acceptable results, but with approximately 2x higher latency and greater resource consumption. YugabyteDB showed inadequate performance, with a processing rate half that of PostgreSQL, 2.4x more CPU and 10x more memory usage, and a high failure rate. Optimizing ticket availability read operations using Redis proved to be very effective, capable of serving peak requests up to 1,700 rps with an average latency of 2.5-4.5 ms, which significantly reduced the load on the database. On the other hand, the flow control scheme on the variants proved beneficial. The early request rejection strategy successfully lowered latency for rejected orders from $>$1000 ms to $<$100 ms. Queue implementation resulted in an average latency of 10 seconds due to communication overhead, and the test load was not high enough to show a performance difference.

    \textbf{\textit{Keywords: Ticket System, PostgreSQL Cluster, CitusData, YugabyteDB, Flow Control}}
\end{singlespace}
\clearpage

\clearpage