\subsection{Layanan Pengguna}

Layanan ini merupakan layanan yang berkaitan dengan otentikasi pengguna, seperti registrasi dan login pengguna. Layanan ini menyimpan data pengguna sistem tiket yang merupakan pemesan.

\subsubsection{Kebutuhan Fungsional dan Non-Fungsional}

Kebutuhan fungsional layanan pengguna dibahas pada Tabel \ref{table:fungsional-pengguna}, sedangkan kebutuhan nonfungsional layanan pengguna dibahas pada Tabel \ref{table:nonfungsional-pengguna}.

\pagebreak

Berikut adalah kebutuhan fungsional layanan pengguna.

\begingroup
\footnotesize
\begin{longtable}{|l|p{0.4\textwidth}|p{0.4\textwidth}|}
    \caption{Kebutuhan Fungsional Layanan Pengguna}
    \label{table:fungsional-pengguna}                                                                                                                                                            \\
    \hline
    \textbf{ID} & \textbf{Kebutuhan}                                                                           & \textbf{Deskripsi}                                                              \\
    \endfirsthead

    \multicolumn{3}{|l|}{\tablename\ \thetable\ -- \textit{Lanjutan dari halaman sebelumnya}}                                                                                                    \\
    \hline
    \textbf{ID} & \textbf{Kebutuhan}                                                                           & \textbf{Deskripsi}                                                              \\
    \endhead

    \hline
    \multicolumn{3}{|r|}{\textit{Dilanjutkan ke halaman berikutnya}}                                                                                                                             \\
    \endfoot

    \hline
    \endlastfoot

    \hline
    UF-01       & Sistem dapat melayani registrasi pengguna.                                                   &                                                                                 \\
    \hline
    UF-02       & Sistem menyediakan mekanisme \textit{login} bagi pengguna.                                   &                                                                                 \\
    \hline
    UF-03       & Sistem menyediakan \textit{endpoint} untuk memperoleh informasi pengguna yang terotentikasi. & \textit{Endpoint} yang dapat digunakan oleh layanan lain seperti layanan tiket. \\
\end{longtable}
\endgroup

Berikut adalah kebutuhan non-fungsional layanan pengguna:

\begingroup
\footnotesize
\begin{longtable}{|l|p{0.4\textwidth}|p{0.4\textwidth}|}
    \caption{Kebutuhan Non-Fungsional Layanan Pengguna}
    \label{table:nonfungsional-pengguna}                                                                                                                                                                       \\
    \hline
    \textbf{ID} & \textbf{Parameter}   & \textbf{Kebutuhan}                                                                                                                                                    \\
    \endfirsthead

    \multicolumn{3}{|l|}{\tablename\ \thetable\ -- \textit{Lanjutan dari halaman sebelumnya}}                                                                                                                  \\
    \hline
    \textbf{ID} & \textbf{Parameter}   & \textbf{Kebutuhan}                                                                                                                                                    \\
    \endhead

    \hline
    \multicolumn{3}{|r|}{\textit{Dilanjutkan ke halaman berikutnya}}                                                                                                                                           \\
    \endfoot

    \hline
    \endlastfoot

    \hline
    UN-01       & Otentisitas Pengguna & Layanan lain dapat memverifikasi otentikasi token pengguna tanpa harus memanggil layanan pengguna, seperti dengan penggunaan token JWT dengan kunci rahasia tertentu. \\
\end{longtable}
\endgroup

\subsubsection{Entitas Layanan}

Entitas yang berkaitan dengan layanan ini adalah entitas pengguna (User). Entitas tersebut memiliki properti sebagaimana ditunjukkan pada Tabel \ref{table:skema-entitas-invoice}.

\begingroup
\footnotesize
\begin{longtable}{|l|p{0.2\textwidth}|p{0.4\textwidth}|}
    \caption{Skema Entitas Invoice}
    \label{table:skema-entitas-invoice}                                                       \\
    \hline
    \textbf{Atribut} & \textbf{Tipe Data} & \textbf{Deskripsi}                                \\
    \endfirsthead

    \multicolumn{3}{|l|}{\tablename\ \thetable\ -- \textit{Lanjutan dari halaman sebelumnya}} \\
    \hline
    \textbf{Atribut} & \textbf{Tipe Data} & \textbf{Deskripsi}                                \\
    \endhead

    \hline
    \multicolumn{3}{|r|}{\textit{Dilanjutkan ke halaman berikutnya}}                          \\
    \endfoot

    \hline
    \endlastfoot

    \hline
    id               & \texttt{string}    & ID unik untuk setiap pengguna.                    \\
    \hline
    name             & \texttt{string}    & Nama pengguna.                                    \\
    \hline
    email            & \texttt{string}    & Alamat surel pengguna.                            \\
    \hline
    password         & \texttt{string}    & Hash kata sandi pengguna                          \\
\end{longtable}
\endgroup
