\subsection{Layanan Pembayaran}

Layanan ini merupakan perupaan sistem gerbang pembayaran eksternal. Layanan ini bukan gerbang pembayaran sesungguhnya, tetapi merupakan \textit{mock service} yang menyimulasikan alur gerbang pembayaran.

\subsubsection{Kebutuhan Fungsional}


\begingroup
\footnotesize
\begin{longtable}{|l|p{0.4\textwidth}|p{0.4\textwidth}|}
    \caption{Kebutuhan Fungsional Sistem Pembayaran}                                                                                                                                                                                                                                                                                                                                                                                                      \\
    \hline
    \textbf{ID} & \textbf{Kebutuhan}                                                                                           & \textbf{Deskripsi}                                                                                                                                                                                                                                                                                                       \\
    \endfirsthead

    \multicolumn{3}{|l|}{\tablename\ \thetable\ -- \textit{Lanjutan dari halaman sebelumnya}}                                                                                                                                                                                                                                                                                                                                                             \\
    \hline
    \textbf{ID} & \textbf{Kebutuhan}                                                                                           & \textbf{Deskripsi}                                                                                                                                                                                                                                                                                                       \\
    \endhead

    \hline
    \multicolumn{3}{|r|}{\textit{Dilanjutkan ke halaman berikutnya}}                                                                                                                                                                                                                                                                                                                                                                                      \\
    \endfoot

    \hline
    \endlastfoot

    \hline
    PF-01       & Sistem dapat membuat tagihan pembayaran dan pranala pembayaran.                                              & Pembayaran dilakukan dengan mengirimkan permintaan kepada pranala yang diberikan dengan parameter sukses atau gagal. Selain itu, terdapat tenggat waktu pembayaran yang ditentukan berdasarkan parameter pada saat pembuatan tagihan. Pembayaran akan otomatis gagal apabila tidak dipenuhi hingga batas waktu tersebut. \\
    \hline
    PF-02       & Sistem memanggil \textit{webhook} yang telah ditentukan ketika terdapat pembayaran yang berhasil atau gagal. &                                                                                                                                                                                                                                                                                                                          \\
    \hline
    PF-03       & Sistem menyediakan \textit{endpoint} untuk menampilkan detail tagihan pembayaran.                            &                                                                                                                                                                                                                                                                                                                          \\
\end{longtable}
\endgroup

\subsubsection{Entitas Layanan}

Entitas yang berkaitan dengan layanan ini adalah entitas tagihan (Invoice). Entitas tersebut memiliki properti sebagai berikut:

\pagebreak

\begingroup
\footnotesize
\begin{longtable}{|l|p{0.4\textwidth}|p{0.4\textwidth}|}
    \caption{Skema Entitas Invoice}                                                                                       \\
    \hline
    \textbf{Atribut} & \textbf{Tipe Data}                     & \textbf{Deskripsi}                                        \\
    \hline
    \endfirsthead

    \multicolumn{3}{|l|}{\tablename\ \thetable\ -- \textit{Lanjutan dari halaman sebelumnya}}                             \\
    \hline
    \textbf{Atribut} & \textbf{Tipe Data}                     & \textbf{Deskripsi}                                        \\
    \hline
    \endhead

    \hline
    \multicolumn{3}{|r|}{\textit{Dilanjutkan ke halaman berikutnya}}                                                      \\
    \endfoot

    \hline
    \endlastfoot

    \hline
    id               & \texttt{string}                        & ID unik untuk setiap tagihan.                             \\
    \hline
    amount           & \texttt{number}                        & Jumlah total tagihan.                                     \\
    \hline
    description      & \texttt{string}                        & Deskripsi dari tagihan.                                   \\
    \hline
    externalId       & \texttt{string}                        & ID dari sistem eksternal yang terkait dengan tagihan ini. \\
    \hline
    createdAt        & \texttt{date}                          & Waktu saat tagihan dibuat.                                \\
    \hline
    expiredAt        & \texttt{date}                          & Waktu saat tagihan akan kedaluwarsa.                      \\
    \hline
    paidAt           & \texttt{date}                          & Waktu saat tagihan dibayar.                               \\
    \hline
    paidAmount       & \texttt{number}                        & Jumlah yang telah dibayarkan.                             \\
    \hline
    status           & "pending", "expired", "paid", "failed" & Status terkini dari tagihan.                              \\
    \hline
\end{longtable}
\endgroup