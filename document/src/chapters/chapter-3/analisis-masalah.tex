
\section{Analisis Masalah}

Penjualan tiket acara dengan tingkat peminat serupa dengan penjualan tiket Taylor Swift dan Coldplay tentu akan terjadi lagi di masa mendatang. Meskipun begitu, tidak semua penyedia layanan terbiasa menangani beban pada skala ini. Kebanyakan penyedia layanan menggunakan solusi antrean virtual yang membatasi jumlah pengguna yang mengakses situs secara bersamaan. Pendekatan ini memang membantu meringankan beban sistem dan menjaga stabilitas. Meskipun begitu, banyak pengguna yang harus menunggu lama untuk bisa mengakses platform.

Di sisi lain, belum banyak studi yang membahas skalabilitas sistem dengan kasus seperti ini. \cite{microservicesEventDriven} membahas desain arsitektur yang \textit{fault-tolerant} dan tidak membahas aspek skalabilitas. \cite{backendForTicketing} juga tidak membahas arsitektur sistem dari sisi skalabilitas. \cite{barua2024enhancingresiliencescalabilitytravel} membahas desain sistem pemesanan tiket pesawat dengan arsitektur \textit{microservice}. Pengoptimalan \textit{fault tolerance} dan \textit{load balancing} memungkinkan penggunaan sumber daya yang lebih optimal, sehingga \textit{throughput} meningkat. Meskipun begitu, penelitian tersebut tidak membahas pengoptimalan pada pola akses basis data. Berdasarkan pertimbangan di atas, diperlukan desain arsitektur yang optimal dan mampu menangani beban seperti ini. Solusi ini tidak serta merta mengganti solusi antrean virtual. Dengan adanya arsitektur yang optimal, jumlah pengguna yang bisa dilayani dalam satu waktu dapat meningkat dan proses penjualan menjadi lebih cepat tanpa kendala.

Beban sistem tiket dapat dibagi menjadi dua: pemrosesan pemesanan tiket dan permintaan baca ketersediaan tiket. Pada kasus tiket, akan ada banyak pengguna yang ingin memesan tiket secara bersamaan. Oleh karena itu, \textit{raw throughput} yang dapat diberikan oleh basis data harus ditingkatkan. Basis data relasional saat ini memang memungkinkan \textit{scaling out} dengan menambah jumlah \textit{instance}, tetapi \textit{instance} yang dapat menulis data tetap ada satu sehingga batas \textit{write throughput} hanya dapat ditingkatkan dengan melakukan penskalaan vertikal. Di sisi lain, basis data relasional terus berkembang dan sudah ada berbagai solusi yang memungkinkan basis data relasional mendukung \textit{multiple-writer}. Peningkatan penskalaan ini memungkinkan peningkatan jumlah pengguna yang dapat dilayani dalam satu waktu.

Pada saat yang bersamaan, terjadi \textit{write contention} saat pemesanan tiket. Skema \textit{locking} diperlukan agar tidak terjadi \textit{double-booking} dan skema \textit{flow control} juga diperlukan agar percobaan pemesanan tiket dapat berkurang/ ditolak lebih cepat sebelum masuk ke basis data. Selain itu, skema ini juga memungkinkan tercapainya stabilitas sistem yang lebih baik.