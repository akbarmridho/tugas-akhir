
\section{Analisis Masalah}

Penjualan tiket acara dengan tingkat peminat serupa dengan penjualan tiket Taylor Swift dan Coldplay tentu akan terjadi lagi di masa mendatang. Meskipun begitu, tidak semua penyedia layanan terbiasa menangani beban pada skala ini. Kebanyakan penyedia layanan menggunakan solusi antrean virtual yang membatasi jumlah pengguna yang mengakses situs secara bersamaan. Pendekatan ini memang membantu meringankan beban sistem dan menjaga stabilitas. Meskipun begitu, banyak pengguna yang harus menunggu lama untuk bisa mengakses platform.

Di sisi lain, belum banyak studi yang membahas skalabilitas sistem dengan kasus seperti ini. \cite{microservicesEventDriven} membahas desain arsitektur yang tahan kegagalan dan tidak membahas aspek skalabilitas. \cite{backendForTicketing} juga tidak membahas arsitektur sistem dari sisi skalabilitas. Berdasarkan pertimbangan di atas, diperlukan desain arsitektur yang optimal dan mampu menangani beban seperti ini. Solusi ini tidak serta merta mengganti solusi antrean virtual. Dengan adanya arsitektur yang optimal, jumlah pengguna yang bisa dilayani dalam satu waktu dapat meningkat dan proses penjualan menjadi lebih cepat tanpa kendala.

Tantangan utama dalam sistem ini dapat dipecah menjadi hal-hal berikut:

\begin{enumerate}
    \item Laju pemrosesan transaksi (pemrosesan pemesanan tiket). Pada kasus tiket, akan ada banyak pengguna yang ingin memesan tiket secara bersamaan. Oleh karena itu, laju pemrosesan yang dapat diberikan oleh basis data harus ditingkatkan. Basis data relasional saat ini memang memungkinkan \textit{scaling out} dengan menambah jumlah instans, tetapi instans yang dapat menulis data tetap ada satu sehingga batas laju penulisan hanya dapat ditingkatkan dengan melakukan penskalaan vertikal. Di sisi lain, basis data relasional terus berkembang dan sudah ada berbagai solusi yang memungkinkan basis data relasional mendukung banyak penulis. Peningkatan penskalaan ini memungkinkan peningkatan jumlah pengguna yang dapat dilayani dalam satu waktu.
    \item Pemrosesan kueri baca. Terdapat beberapa kueri baca yang dipanggil dengan jumlah yang banyak dengan data yang selalu berubah. Kueri tersebut adalah kueri permintaan baca ketersediaan tiket, baik secara agregat atau pun satuan. Data ini selalu berubah karena berkaitan dengan ketersediaan tiket yang terus berubah seiring dengan keberjalanan proses penjualan. Oleh karena itu, perlu pengoptimalan khusus yang tidak dapat diselesaikan dengan metode tembolok pada umumnya.
    \item Integritas data dan kondisi pacu. Sifat penjualan tiket yang "siapa cepat, dia dapat" menciptakan perebutan sumber daya yang ekstrem. Beberapa pengguna mungkin mencoba memesan kategori tiket yang sama pada saat yang sama. Oleh karena itu, sistem harus dapat menangani kondisi ini dengan menjamin bahwa tidak akan terjadi pemesanan ganda untuk satu unit tiket yang sama.
    \item Pengendalian aliran untuk stabilitas sistem. Tanpa mekanisme perlindungan, lonjakan permintaan dapat menggangu keberjalanan pemrosesan pemesanan. Beberapa permintaan pada akhirnya akan gagal karena tiket sudah dipesan oleh pengguna lain. Oleh karena itu, diperlukan skema pengendalian aliran diperlukan untuk menjaga keberjalanan dan kestabilan pemrosesan pesanan.
\end{enumerate}
