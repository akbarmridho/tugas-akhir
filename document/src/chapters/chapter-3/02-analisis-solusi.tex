\section{Ringkasan Solusi}

\subsection{Ide Dasar}

Berikut adalah ide dasar yang dapat dimanfaatkan untuk mengoptimalkan sistem ini:

\begin{enumerate}
  \item Tanggung jawab pelayanan ketersediaan tiket dapat dilimpahkan kepada RisingWave, sehingga beban pada basis data berkurang. \textit{Streaming database} ini dapat memperbarui hasil kueri secara inkremental setiap kali ada perubahan. Pendekatan ini mengurangi proses agregasi yang dilakukan secara berulang-ulang, sehingga kueri lebih efisien. Pendekatan ini merupakan bagian dari \textit{query responsibility segregation} pada pola CQRS.
  \item Menggunakan kluster PostgreSQL dengan ekstensi Citus dan pembagian data berdasarkan baris. Pendekatan ini memungkinkan adanya \textit{multiple writer}, sehingga \textit{throughput} pemesanan tiket dapat meningkat. Di sisi lain, tidak menggunakan basis data relasional juga bisa menjadi opsi. Pendekatan \textit{database inside-out} yang memisahkan komponen \textit{storage} dan \textit{query} memungkinkan setiap komponen \textit{scale} secara independen. Meskipun begitu, perlu perhatian khusus untuk menghindari pemesanan ganda dan menjaga integritas transaksi.
  \item Untuk mengontrol pemesanan tiket, pendekatan penyeimbangan beban berbasiskan antrean dapat digunakan. Pemesanan tiket dapat didesain sebagai \textit{command} pada pola CQRS. Dengan begini, sistem dapat memproses permintaan sesuai dengan kapasitas sistem, sehingga stabilitas sistem dapat terjaga. Selain itu, informasi tiket yang sedang dipesan tetapi belum \textit{commit} (kita sebut \textit{uncommited data}) dapat disimpan pada Redis yang kemudian digunakan untuk menolak permintaan lebih awal dan mengurangi permintaan yang masuk.
\end{enumerate}

\subsection{Arsitektur Solusi}

Setiap arsitektur solusi memiliki layanan tiket, pengguna, pembayaran, dan basis data. Bahasan lengkap setiap arsitektur solusi dan layanan pendukung dibahas pada lampiran \ref{apx:analisis solusi}. Terdapat tiga arsitektur solusi yang akan dibahas, yaitu:

\subsubsection{Arsitektur Dasar Acuan (RADAR)}

Arsitektur dasar acuan, diberi nama RADAR (rancangan dasar), terdiri atas layanan tiket, pengguna, pembayaran, dan basis data PostgreSQL dalam kluster dengan skema \textit{read replica}. Arsitektur ini akan dijadikan sebagai acuan pengujian.

\subsubsection{Arsitektur yang Mengoptimalkan PostgreSQL (PGP)}

Arsitektur ini, diberi nama Postgres Plus (PGP), mengoptimalkan basis data dengan menggunakan ekstensi Citus. Dengan ini, data dapat dipartisi berdasarkan baris dan setiap node basis data dapat menjadi \textit{writer}. Perubahan pemesanan tiket akan dikonsumsi oleh RisingWave melalui CDC. RisingWave akan melayani permintaan baca alih-alih melalui basis data utama. Redis dan Redpanda juga digunakan untuk menerapkan pola penyeimbangan beban berbasis antrean untuk operasi pemesanan tiket.

\subsubsection{Arsitektur \textit{Event-Driven} (EDA)}

Arsitektur ini, diberi nama \textit{event-driven architecture} (EDA), tidak menggunakan basis data PostgreSQL. Setiap perubahan ditulis dalam bentuk perintah ke dalam Redpanda yang bertindak sebagai \textit{log storage}. Untuk melakukan kueri, RisingWave akan mengonsumsi data dari Redpanda lalu membangun hasil kueri dari data tersebut. Pendekatan ini memungkinkan penskalaan independen untuk setiap komponennya. Selain itu, Redis dengan konfigurasi \textit{persistence} digunakan untuk menyimpan \textit{uncommited data}, sehingga kasus \textit{double booking} dapat dihindari.