\section{Analisis Solusi}

\subsection{Peningkatan \textit{Raw Throughput} Pemesanan Tiket}

Penskalaan \textit{throughpupt} penulisan pada basis data relasional tradisional seperti PostgreSQL terbatas pada maksimal CPU, Memori, dan \textit{throughput} media penyimpanan yang dapat digunakan oleh suatu komputer. Penskalaan ini memiliki batas dari sisi teknologi dan peningkatan biayanya tidak sebanding dengan peningkatan kinerja yang dihasilkan. Oleh karena itu, skema PostgreSQL dengan \textit{primary} dan \textit{replica} memiliki keterbatasan dari sisi penulisan.

Penggunaan basis data nonrelasional berbasikan dokumen seperti MongoDB memang menarik karena menawarkan \textit{throughput} yang jauh lebih baik. Meskipun begitu, basis data tersebut tidak memiliki dukungan kueri transaksional sebagaimana yang dimiliki pada basis data relasional. Hal yang sama juga terjadi dengan penggunaan basis data berbasiskan \textit{wide-column}. Ide yang menyatakan bahwa basis data relasional tidak \textit{scalable} perlu dipertanyakan karena keduanya memiliki pemodelan entitas yang jauh berbeda. Oleh karena itu, penelitian ini akan mengeksporasi basis data relasional yang dapat di-\textit{scale-out} sehingga \textit{write throughput} dapat meningkat.

Terdapat berbagai alternatif basis data relasional yang merupakan pengembangan dari basis data tradisional. PostgreSQL dan MySQL merupakan basis data relasional yang sudah umum diketahui. Di sisi lain, terdapat CitusData yang merupakan ekstensi PostgreSQL dan memungkinkan PostgreSQL mendukung \textit{multiple-writer} dengan skema \textit{sharding} baik dari sisi skema atau pun data \parencite{citus}. Hal yang sama juga terjadi pada ekosistem MySQL dengan Vitess \textit{Vitess}. Selain itu, terdapat basis data yang bukan merupakan pengembangan dari basis data yang ada seperti PostgreSQL dan MySQL dan merupakan pengembangan basis data yang dimulai dari nol dengan penggunaan konsensus seperti Raft dan protokol \textit{low-level} seperti RocksDB atau LevelDB. Meskipun begitu, basis data seperti itu tetap mengimplementasikan antarmuka yang sama pada PostgreSQL dan MySQL. CockroachDB merupakan basis data jenis ini yang kompatibel dengan antarmuka MySQL \parencite{cockroachDB}. YugaByteDB merupakan basis data jenis ini yang kompatible dengan antarmuka PostgreSQL \parencite{yugabyte}. Penggunaan basis data tersebut menarik untuk diuji kinerjanya dibandingkan dengan basis data biasa dan menarik untuk diketahui apakah penggunaan teknologi tersebut dapat membantu kinerja sistem tiket yang dibahas pada penelitian ini.

Dari sekian banyak alternatif basis data relasional, perlu dipilih basis data yang masing-masing dapat merepresentasikan pendekatan yang dipakai pada basis data tersebut. Basis data yang mendukung antarmuka PostgreSQL dipilih karena keakraban dengan teknologi tersebut. Selain itu, pemilihan basis data dengan antarmuka yang sama memungkinkan pengembangan yang lebih mudah dan perbandingan kinerja yang lebih adil. Oleh karena itu, PostgreSQL dipilih sebagai basis data yang menjadi dasar acuan, CitusData dipilih sebagai basis data yang mengembangkan PostgreSQL, sedangkan YugaByteDB dipilih sebagai basis data dengan ide berbeda, tetapi tetap mengimplementasikan antarmuka PostgreSQL.

\subsection{Penggunaan Skema \textit{Flow Control} Pada Pemesanan Tiket}

\subsection{Pengoptimalan Operasi Baca Ketersediaan Tiket}
