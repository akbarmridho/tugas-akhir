\subsection{Distribusi Profil Pengguna}

Ketika pengguna berhasil memesan tiket, terdapat 5\% kemungkinan bahwa pembayaran pengguna gagal. Hal ini dilakukan untuk menyimulasikan pengguna yang batal membeli tiket atau mengalami kegagalan pembayaran. Meskipun begitu, tidak ada referensi pasti yang menyebutkan tingkat kegagalan pemesanan untuk penjualan tiket. Sebagai referensi, tingkat kegagalan pemrosesan transaksi pada \textit{e-commerce} berkisar pada 11\% \parencite{paymentFail}. Meskipun begitu, angka yang lebih rendah diambil karena pada kasus ini pengguna yang mendaftar memiliki keinginan yang lebih kuat untuk memperoleh tiket.

Distribusi preferensi kategori tiket, sebagaimana dirinci pada Tabel \ref{table:distribusi-kategori}, didasarkan pada teori diskriminasi harga (\textit{tiered pricing}) \parencite{acei2022}. Pendekatan ini biasanya digunakan oleh penyelenggara untuk memaksimalkan pendapatan dengan mensegmentasi pasar berdasarkan kesediaan untuk membayar (\textit{willingness to pay}). Sebagian besar permintaan (53\%) terkonsentrasi pada tingkatan harga menengah ke bawah (\textit{Seated Low \& Mid}), yang mencerminkan segmen pasar yang lebih besar dan sensitif terhadap harga. Sebaliknya, profil dengan minat pada kategori harga lebih tinggi merepresentasikan segmen pasar yang lebih kecil dan kurang sensitif terhadap harga untuk tiket yang lebih mahal.

\begin{table}[h]
    \centering
    \caption{Distribusi Kategori}
    \label{table:distribusi-kategori}
    \begin{tabular}{|l|l|l|}
        \hline
        \textbf{Tingkatan} & \textbf{Distribusi} & \textbf{Minat Kategori} \\
        \hline
        Seated Low         & 27\%                & Bronze \& Silver        \\
        \hline
        Seated Mid         & 26\%                & Silver, Gold, \& Bronze \\
        \hline
        Seated High        & 19\%                & Platinum \& Gold        \\
        \hline
        Area Mid           & 17\%                & Zona A \& Zona B        \\
        \hline
        Area High          & 11\%                & VIP \& Zona A           \\
        \hline
    \end{tabular}
\end{table}

Preferensi hari acara dibagi menjadi 52\% "Spesifik" dan 48\% "Bebas" (\textit{Any}), seperti pada Tabel \ref{table:distribusi-hari}. Pembagian ini mencerminkan keseimbangan antara penonton dengan jadwal yang kaku dan mereka yang memiliki fleksibilitas. Terdapat preferensi konsumen yang kuat untuk acara rekreasi pada akhir pekan karena berbagai komitmen sosial dan pekerjaan \parencite{rgate2013}. Kelompok "Spesifik" mewakili mereka yang terikat pada hari tertentu, sementara kelompok "Bebas" mewakili penggemar yang lebih fleksibel atau sangat termotivasi untuk mendapatkan tiket.

\begin{table}[h]
    \centering
    \caption{Distribusi Hari}
    \label{table:distribusi-hari}
    \begin{tabular}{|l|l|}
        \hline
        \textbf{Hari} & \textbf{Distribusi} \\
        \hline
        Specific      & 52\%                \\
        \hline
        Any           & 48\%                \\
        \hline
    \end{tabular}
\end{table}

Variasi berikutnya adalah variasi dari seberapa persisten pengguna untuk mendapatkan tiket yang diinginkan. Variasi ini ditentukan dengan berapa kali maksimal pencarian dan pemesanan sebelum pengguna tersebut menyerah. Distribusi persistensi pengguna dibahas pada Tabel \ref{table:distribusi-persistensi}.

\begin{table}[h]
    \centering
    \caption{Distribusi Persistensi}
    \label{table:distribusi-persistensi}
    \begin{tabular}{|l|l|l|}
        \hline
        \textbf{Persistence} & \textbf{Distribusi} & \textbf{Batas}                      \\
        \hline
        Low                  & 19\%                & 9x pencarian atau 3x pemesanan      \\
        \hline
        Medium               & 51\%                & 18x pencarian atau 6x     pemesanan \\
        \hline
        High                 & 30\%                & 27x pencarian atau 9x pemesanan     \\
        \hline
    \end{tabular}
\end{table}

Variasi terakhir adalah variasi dari berapa banyak tiket yang dipesan oleh suatu pengguna dalam satu kali pemesanan. Hal ini untuk menyimulasikan pengguna yang datang sendiri, bersama pasangan, dan datang dalam satu grup. Distribusi ini dibahas pada Tabel \ref{table:distribusi-kuantitas}. Model ini menghasilkan rata-rata tertimbang 2-3 tiket per transaksi, yang sangat selaras dengan data dari industri hiburan lain, seperti penjualan tiket film yang menunjukkan rata-rata 2.3 tiket per transaksi \parencite{vista2025}.

\begin{table}[h]
    \centering
    \caption{Distribusi Kuantitas}
    \label{table:distribusi-kuantitas}
    \begin{tabular}{|l|l|l|}
        \hline
        \textbf{Varian} & \textbf{Distribusi} & \textbf{Kuantitas} \\
        \hline
        Solo            & 28\%                & 1                  \\
        \hline
        Couple          & 48\%                & 2                  \\
        \hline
        Group           & 24\%                & 3-5                \\
        \hline
    \end{tabular}
\end{table}

Distribusi kombinasi setiap variasi di atas disertakan pada Lampiran \ref{apx:distribusi-profil}.