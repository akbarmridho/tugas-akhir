\subsection{Distribusi Profil Pengguna}

Ketika pengguna berhasil memesan tiket, terdapat 5\% kemungkinan bahwa pembayaran pengguna gagal. Hal ini dilakukan untuk menyimulasikan pengguna yang batal membeli tiket atau mengalami kegagalan pembayaran.

Sebuah acara memiliki dua tipe kursi, yaitu kursi bernomor (\textit{numbered seat}) dan festival (\textit{free-standing}). Tiket kursi bernomor memiliki kategori Platinum, Gold, Silver, dan Bronze. Tiket festival memiliki kategori VIP, Zona A, dan Zona B. Setiap pengguna memiliki preferensi kategori kursi yang ingin dibeli, sebagaimana digambarkan pada distribusi berikut.

\begin{table}[h]
    \centering
    \begin{tabular}{|l|l|l|}
        \hline
        \textbf{Tingkatan} & \textbf{Distribusi} & \textbf{Minat Kategori} \\
        \hline
        Seated Low         & 27\%                & Bronze \& Silver        \\
        \hline
        Seated Mid         & 26\%                & Silver, Gold, \& Bronze \\
        \hline
        Seated High        & 19\%                & Platinum \& Gold        \\
        \hline
        Area Mid           & 17\%                & Zona A \& Zona B        \\
        \hline
        Area High          & 11\%                & VIP \& Zona A           \\
        \hline
    \end{tabular}
    \caption{Distribusi Kategori}
\end{table}

Tingkatan merupakan nama profil dengan minat kategori tertentu secara terurut. Selain itu, setiap pengguna memiliki preferensi hari. Pengguna yang memiliki preferensi spesifik akan mengutamakan hari yang sesuai dengan preferensinya (bisa satu atau dua pilihan hari), dibandingkan dengan kesesuaian preferensi kategori. Pengguna yang tidak memiliki preferensi akan mengutamakan kesesuaian preferensi kategori dibandingkan dengan kesesuaian hari.

\begin{table}[h]
    \centering
    \begin{tabular}{|l|l|}
        \hline
        \textbf{Hari} & \textbf{Distribusi} \\
        \hline
        Specific      & 52\%                \\
        \hline
        Any           & 48\%                \\
        \hline
    \end{tabular}
    \caption{Distribusi Hari}
\end{table}

Variasi berikutnya adalah variasi dari seberapa persisten pengguna untuk mendapatkan tiket yang diinginkan. Variasi ini ditentukan dengan berapa kali maksimal pencarian dan pemesanan sebelum pengguna tersebut menyerah.

\begin{table}[h]
    \centering
    \begin{tabular}{|l|l|l|}
        \hline
        \textbf{Persistence} & \textbf{Distribusi} & \textbf{Batas}                      \\
        \hline
        Low                  & 19\%                & 9x pencarian atau 3x pemesanan      \\
        \hline
        Medium               & 51\%                & 18x pencarian atau 6x     pemesanan \\
        \hline
        High                 & 30\%                & 27x pencarian atau 9x pemesanan     \\
        \hline
    \end{tabular}
    \caption{Distribusi Persistensi}
\end{table}

\pagebreak

Variasi terakhir adalah variasi dari berapa banyak tiket yang dipesan oleh suatu pengguna dalam satu kali pemesanan. Hal ini untuk menyimulasikan pengguna yang datang sendiri, bersama pasangan, dan datang dalam satu grup.

\begin{table}[h]
    \centering
    \begin{tabular}{|l|l|l|}
        \hline
        \textbf{Varian} & \textbf{Distribusi} & \textbf{Kuantitas} \\
        \hline
        Solo            & 28\%                & 1                  \\
        \hline
        Couple          & 48\%                & 2                  \\
        \hline
        Group           & 24\%                & 3-5                \\
        \hline
    \end{tabular}
    \caption{Distribusi Kuantitas}
\end{table}

Distribusi kombinasi setiap variasi di atas disertakan pada lampiran bagian \ref{apx:distribusi-profil}.