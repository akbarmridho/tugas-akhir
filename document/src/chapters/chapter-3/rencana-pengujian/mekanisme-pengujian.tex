\subsection{Metodologi Pengujian Sistem}

Untuk mengevaluasi dan mengoptimalkan sistem tiket, metodologi pengujian dirancang untuk menyimulasikan kondisi operasional di dunia nyata secara akurat. Fokus utama pengujian adalah memastikan sistem mampu menangani skenario beban tinggi dengan andal. Metodologi ini mencakup aspek-aspek berikut:

\begin{enumerate}
    \item \textbf{Simulasi beban realistis pada skala besar:} Pengujian dilakukan dengan beban yang melampaui kapasitas satu server tunggal.
    \item \textbf{Simulasi perilaku pengguna:} Pengujian dilakukan dengan penggunaan pengguna virtual yang memiliki alur aksi serta profil pengguna yang beragam. Setiap pengguna virtual diberi profil tertentu untuk meniru keberagaman perilaku pembeli di dunia nyata.
    \item \textbf{Skenario penjualan:} Data acara dan distribusi kuota tiket diambil dari estimasi yang merepresentasikan penjualan tiket populer di dunia nyata. Distribusi kursi setiap kategori tidak akan sama dengan distribusi minat pembeli. Hal ini wajar karena pada kenyataannya memang terdapat kategori tertentu yang lebih ramai peminat dibandingkan dengan kategori yang lain.
    \item \textbf{Skenario Pengujian:} Dua skenario pengujian disiapkan untuk menganalisis kinerja sistem, yaitu pengujian beban berkelanjutan dan pengujian perebutan tiket.
\end{enumerate}
