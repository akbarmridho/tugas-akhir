\subsection{Implementasi Layanan Pembayaran}

Untuk menyimulasikan proses pembayaran, sebuah \textit{mock service} untuk gerbang pembayaran diimplementasikan. Layanan ini bertanggung jawab untuk mengelola siklus hidup tagihan, mulai dari pembuatan, simulasi pembayaran, hingga pemberitahuan status pembayaran kembali ke sistem tiket.

Interaksi antara sistem tiket dan Payment Service dilakukan melalui serangkaian \textit{endpoint} REST API. Layanan ini juga mengirimkan pemberitahuan melalui \textit{webhook} ketika status sebuah tagihan berubah (misalnya, berhasil dibayar atau kedaluwarsa). Untuk menjamin keamanan, setiap isi data \textit{webhook} diamankan menggunakan HMAC. Detail teknis lebih lanjut mengenai arsitektur internal dan teknologi yang digunakan pada layanan ini dapat dilihat pada lampiran \ref{apx:payment-implementation}.

Tabel \ref{table:payment-endpoint} membahas daftar \textit{endpoint} yang tersedia pada Payment Service untuk berinteraksi dengan sistem tiket dan pengguna.

\begin{table}[h!]
\centering
\caption{Daftar \textit{Endpoint} Layanan Pembayaran}
\label{table:payment-endpoint}
\begin{tabular}{|p{0.4\textwidth}|p{0.5\textwidth}|}
\hline
\textbf{Endpoint} & \textbf{Keterangan} \\
\hline
Get Invoice \newline GET /invoices/{id} & Mengembalikan data tagihan berdasarkan ID yang diberikan pada parameter. \\
\hline
Create Invoice \newline POST /invoices & Menerima jumlah tagihan, deskripsi, dan ID eksternal dari layanan tiket untuk membuat tagihan baru. Tagihan akan kedaluwarsa dalam 15 menit. \\
\hline
Pay Invoice \newline POST /invoices/{id}/payment & Menyimulasikan proses pembayaran yang dilakukan pengguna. Endpoint ini menerima data untuk menandai apakah pembayaran berhasil atau gagal. \\
\hline
\end{tabular}
\end{table}
