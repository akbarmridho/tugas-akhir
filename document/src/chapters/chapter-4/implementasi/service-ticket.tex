\subsection{Implementasi Sistem Tiket}

\subsubsection{Arsitektur Sistem}

Penulisan kode layanan ini terinspirasi dari pendekatan \textit{clean architecture} yang memisahkan kode menjadi bagian Infrastructure, Repository, Entity, Service, dan Usecase. Pada layanan ini, grup Repository, Entity, Service, dan Usecase dipisahkan berdasarkan modul, seperti modul Bookings, Events, Orders, dan Payments. Arsitektur ini dipilih karena merupakan pendekatan yang sudah cukup familiar digunakan. Selain itu, pendekatan ini memungkinkan pemecahan bagian kode berdasarkan tanggung jawab dan \textit{concern} tertentu.

Kode Infrastructure berisi modul yang berkaitan dengan driver PostgreSQL, Redis, RabbitMQ, \textit{in-memory cache}, dan konfigurasi aplikasi. Kode Repository berisi kode yang memuat kueri untuk operasi tertentu. Kode Service berisi grup fungsionalitas tambahan seperti integrasi dengan layanan pembayaran. Kode Usecase merupakan kode yang mengandung logika proses bisnis dan umumnya memanggil Repository serta memulai transaksi basis data bila diperlukan.

Sebagian besar kode dipisah menjadi Interface dan Implementasi. Hal ini dilakukan agar kode implementasi bisa diubah sesuai kebutuhan. Pendekatan ini memungkinkan implementasi kode pemrosesan pesanan dibuat berbeda, tetapi tetap memiliki antarmuka yang sama.

Selain itu, terdapat berkas App yang berisi kode spesifik masing-masing aplikasi, seperti Server, Worker, dan Sanity Check. Kode Server berisi Handler, Route, dan definisi server HTTP. Kode Worker berisi integrasi algoritma pengendalian aliran, \textit{handler} pemrosesan pemesanan, dan lain-lain.

Setiap \textit{runtime} memiliki basis kode yang sama agar kode yang sama dapat digunakan berulang-ulang dan proses pengembangan menjadi lebih mudah.

\subsubsection{Endpoint}

Terdapat berbagai \textit{endpoint} selain health check dan metrics yang tersedia pada sistem tiket. Setiap \textit{endpoint} berikut membutuhkan otentikasi JWT yang diverifikasi melalui HTTP header Authorization.

Berikut adalah \textit{endpoint} yang tersedia pada sistem tiket:

\pagebreak

\begin{table}[h!]
\centering
\caption{Daftar \textit{Endpoint} Layanan Tiket}
\begin{tabular}{|p{0.4\textwidth}|p{0.5\textwidth}|}
\hline
\textbf{Endpoint} & \textbf{Keterangan} \\
\hline
GetEvents \newline GET /events & Mengembalikan daftar acara yang tersedia. \\
\hline
GetEventByID \newline GET /events/:id & Mengembalikan informasi detail terkait suatu acara, seperti kategori tiket, penjualan tiket, dan lain-lain. \\
\hline
GetEventAvailability \newline GET /events/availability/:ticketSaleID & Mengembalikan agregasi ketersediaan tiket yang dibagi berdasarkan area untuk penjualan tiket tertentu. \\
\hline
GetSeats \newline GET /events/seats/:ticketAreaID & Mengembalikan daftar ketersediaan kursi dalam suatu area. \\
\hline
Create Order \newline POST /orders & Menangani permintaan pemesanan tiket. Ketika permintaan pesanan tidak dapat diproses karena konflik, \textit{endpoint} ini akan mengembalikan status 409 dan pengguna akan direkomendasikan untuk memesan kursi lain. Selain itu, \textit{endpoint} ini akan mengembalikan status 423 apabila sistem tidak dapat melakukan penguncian dan pengguna disarankan untuk mengirim ulang permintaan. \\
\hline
Get Order \newline GET /orders/:id & Mengembalikan detail order berdasarkan ID. \\
\hline
Get Issued Tickets \newline GET /orders/:id/tickets & Mengembalikan tiket yang sudah diterbitkan untuk suatu pesanan yang sudah berhasil. \\
\hline
Notify Webhook \newline POST /webhook & Menangani notifikasi dari layanan pembayaran terkait dengan perubahan status tagihan. \\
\hline
\end{tabular}
\end{table}

\subsubsection{Pengoptimalan Skema Basis Data Terdistribusi}

Kueri yang digunakan untuk setiap varian basis data secara umum sama. Meskipun begitu, terdapat variasi pada skema basis data. Hal ini diperlukan agar varian basis data CitusData dan YugabyeDB dapat berjalan dengan optimal.

Pengoptimalan skema untuk CitusData meliputi pembuatan \textit{reference table}, yakni tabel yang direplikasi pada setiap instans PostgreSQL. Hal ini dilakukan agar basis data tidak perlu melakukan \textit{cross-shard join}, sehingga meminimalkan latensi. Tabel yang dibuat menjadi \textit{reference table} adalah tabel Events, TicketCategories, TicketSales, TicketPackages, dan TicketAreas.

Selain itu, terdapat \textit{distributed table}. Tabel ini merupakan pemartisian sebuah tabel berdasarkan baris. Untuk melakukan pemartisian diperlukan sebuah kolom yang menjadi \textit{shard key}. Implementasi tugas akhir ini melakukan distribusi pada tabel TicketSeats, Orders, OrderItems, Invoices, dan IssuedTickets berdasarkan kolom ticket\_area\_id. Setiap tabel ini memiliki kolom ticket\_area\_id. Baris data pada entitas tersebut yang memiliki \textit{shard key} yang sama akan berada pada partisi yang sama, sehingga operasi yang berkaitan pada satu order dilakukan pada satu partisi.

Pengoptimalan skema pada YugabyeDB sedikit berbeda dengan CitusData. Pada YugabyeDB, tidak ada \textit{reference table} karena data memang sudah tereplikasi. Selain itu, tabel TicketSeats, Orders, OrderItems, Invoices, dan IssuedTickets tetap didistribusikan berdasarkan satu kunci hash yang sama yaitu kolom ticket\_area\_id, meski dengan sintaks inisiasi yang sedikit berbda.

\subsubsection{Pengendalian Aliran}

Sebagaimana dibahas pada bagian rancangan, terdapat dua strategi yang ditetapkan untuk meringankan beban pemrosesan pesanan, yaitu:

\begin{enumerate}
      \item Penolakan permintaan lebih awal dengan memprediksi kursi sudah akan terjual ketika terdapat pesanan untuk kursi yang sama sedang diproses lebih awal.
      \item Penggunaan penyangga/ antrean untuk menangani lonjakan permintaan pesanan dan menjaga stabilitas pemrosesan.
\end{enumerate}

Alur implementasi pengendalian aliran sesuai dengan bahasan pada rancangan implementasi. Meskipun begitu, terdapat beberapa detail implementasi yang perlu dielaborasi.

Implementasi pengendalian aliran untuk strategi antrean dapat dibagi menjadi beberapa komponen penting, yaitu: \textit{event consumer}, \textit{rate limiter}, dan pemroses pesanan. \textit{Event consumer} bertugas untuk mendengarkan pesan masuk dari RabbitMQ. \textit{Rate limiter} merupakan komponen yang mengatur banyaknya \textit{worker} yang berjalan pada satu waktu.

Algoritma \textit{rate limiter} yang digunakan dalam tugas akhir ini merupakan algoritma yang sederhana, yaitu membatasi jumlah konkurensi dengan nilai tertentu. Pada kasus ini, batas tersebut adalah 5000 konkurensi. Algoritma memberi jaminan bahwa jumlah pesanan yang diproses konsisten tidak melebihi batas tertentu.

Terdapat algoritma yang lebih cangggih dan mampu mengatur batas konkurensi secara dinamis, seperti Gradient2. Algoritma ini mengatur batas konkurensi berdasarkan gradien perubahan RTT dan \textit{long term exponentially smoothed average RTT}. Penggunaan algoritma ini dapat meminimalkan dampak pencilan untuk trafik yang bersifat bursty \parencite{platinummonkey_go_concurrency_limits}. Meskipun begitu, algoritma sebelumnya digunakan karena kemudahan implementasi dan proses \textit{tuning} selama pengujian.

\subsubsection{\textit{Runtime} Aplikasi}

Sistem tiket dapat dibagi menjadi enam program yang berbeda, yaitu:

\begin{enumerate}
      \item Server tanpa Pengendalian Aliran

            Program ini berisi kode eksekusi layanan tiket tanpa pengendalian aliran.

      \item Server dengan Pengendalian Aliran

            Program ini berisi kode eksekusi layanan tiket dengan pengendalian aliran.

      \item Worker (Pemroses Pesanan)

            Program ini berisi kode eksekusi worker pemesanan tiket pada variasi layanan tiket dengan pengendalian aliran.

      \item Sanity Check

            Program ini melakukan kueri terhadap data ketersediaan dan \textit{early dropper} pada basis data dan pada Redis. Hal ini digunakan untuk melakukan validasi terhadap sinkronisasi data antara Redis dengan basis data. Pemeriksaan ini dilakukan setiap 2 menit.

      \item Seeder

            Program ini dijalankan saat sebelum pengujian dimulai. Program ini menjalankan hal-hal yang berkaitan dengan pengaturan basis data dan Redis, seperti membuat skema, penyemaian data, \textit{prewarm} basis data, penyemaian data ketersediaan pada Redis, dan lain-lain.

\end{enumerate}

Setiap program (selain Seeder) memiliki integrasi dengan Prometheus untuk proses pengumpulan metrik, bergantung pada jenis programnya. Program server berisi metrik yang berkaitan dengan HTTP server. Program Worker berkaitan dengan metrik pengendalian aliran dan ukuran antrean yang sedang diproses. Runtime Sanity Check berisi hasil kueri agregat.
