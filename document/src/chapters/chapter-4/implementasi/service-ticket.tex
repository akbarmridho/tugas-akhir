\subsection{Implementasi Ticket Service}

Program ini dibuat dengan bahasa pemrograman Golang. Runtime layanan ini dibagi menjadi 6 bagian, yaitu:

\begin{enumerate}
      \item Server No Flow Control

            Runtime ini berisi kode eksekusi layanan tiket tanpa varian flow control.

      \item Server with Flow Control

            Runtime ini berisi kode eksekusi layanan tiket dengan varian flow control.

      \item Worker Flow Control

            Runtime ini berisi kode eksekusi worker pemesanan tiket pada layanan tiket dengan flow control.

      \item Sanity Check

            Runtime ini melakukan kueri terhadap data ketersediaan dan \textit{early dropper} pada basis data dan pada Redis. Hal ini digunakan untuk melakukan validasi terhadap sinkronisasi data antara Redis dengan basis data. Pemeriksaan ini dilakukan setiap 2 menit.

      \item Seeder

            Runtime ini dijalankan saat sebelum pengujian dimulai. Runtime ini menjalankan hal-hal yang berkaitan dengan setup basis data dan Redis, seperti membuat skema, seeding data, prewarm basis data, trigger seeding availability pada Redis, dan lain-lain.

\end{enumerate}

Setiap runtime (selain seeder) memiliki integrasi dengan Prometheus untuk proses pengumpulan metrik, bergantung pada jenis runtimenya. Runtime server berisi metrik yang berkaitan dengan HTTP server. Runtime worker berkaitan dengan metrik flow control dan ukuran queue yang sedang diproses. Runtime sanity check berisi hasil kueri agregat, sesuai dengan yang dibahas pada bagian sebelumnya.

\subsubsection{Arsitektur}

Arsitektur layanan ini terinspirasi dari pendekatan clean architecture yang memisahkan bagian kode menjadi infrastructure, repository, entity, service, dan usecase. Pada layanan ini, grup repository, entity, service, dan usecase dipisahkan berdasarkan modul, seperti bookings, events, orders, dan payments.

Kode infrastructure berisi modul yang berkaitan dengan driver PostgreSQL, Redis, RabbitMQ, in-memory cache, dan config. Kode repository berisi kode yang memuat kueri untuk operasi tertentu. Kode service berisi grup fungsionalitas tambahan seperti integrasi dengan payment service. Kode usecase merupakan kode yang mengandung business logic dan umumnya memanggil repository serta memulai transaksi basis data bila dipelrukan.

Kebanyakan kode dipisah menjadi interface dan implementasi. Hal ini dilakukan agar kode implementasi bisa diubah sesuai kebutuhan. Pendekatan ini memungkinkan implementasi kode pemrosesan pesanan dibuat berbeda, tetapi tetap memiliki interface yang sama.

Selain itu, terdapat folder app yang berisi kode spesifik masing-masing aplikasi, seperti server, worker, dan sanity check. Kode server berisi handler, route, dan definisi HTTP server. Kode worker berisi integrasi algoritma flow control, rate limiter, dan lain-lain.

Setiap runtime memiliki basis kode yang sama agar kode yang sama dapat digunakan berulang-ulang dan proses pengembangan menjadi lebih mudah.

\subsubsection{Endpoint}

Terdapat berbagai endpoint selain health check dan metrics yang tersedia pada sistem tiket. Setiap endpoint berikut membutuhkan otentikasi JWT yang diverifikasi melalui header Authorization.

Berikut adalah endpoint yang tersedia pada sistem tiket:

\begin{enumerate}
      \item GET /events - GetEvents

            Endpoint ini mengembalikan daftar event yang tersedia.

      \item GET /events/:id - GetEventByID

            Endpoint ini mengembalikan informasi detail terkait suatu event, seperti kategori tiket, penjualan tiket, dan lain-lain.

      \item GET /events/availability/:ticketSaleID - GetEventAvailability

            Endpoint ini mengembalikan agregasi ketersediaan tiket yang dibagi berdasarkan area.

      \item GET /events/seats/:ticketAreaID - GetSeats

            Endpoint ini mengembalikan daftar ketersediaan kursi dalam suatu area.

      \item POST /orders - Create Order

            Endpoint ini menangani permintaan pemesanan tiket. Ketika permintaan pesanan tidak dapat diproses karena konflik, endpoint ini akan mengembalikan status 409 dan pengguna akan direkomendasikan untuk memesan kursi lain.

      \item GET /orders/:id - Get Order

            Endpoint ini mengembalikan detail order berdasarkan ID.

      \item GET /orders/:id/tickets - Get Issued Tickets

            Endpoint ini mengembalikan tiket yang sudah diterbitkan untuk suatu pesanan yang sudah berhasil.

      \item POST /webhook - Notify Webhook

            Endpoint ini menangani notifikasi dari layanan pembayaran terkait dengen perubahan status tagihan.
\end{enumerate}

\subsubsection{Pengoptimalan Skema Basis Data}

Kueri yang digunakan untuk setiap varian basis data secara umum sama. Meskipun begitu, terdapat variasi pada pengaturan skema. Hal ini diperlukan agar varian basis data CitusData dan YugabyeDB dapat berjalan dengan optimal.

Pengoptimalan skema untuk CitusData meliputi pembuatan reference table, yakni tabel yang direplikasi pada setiap instance PostgreSQL. Hal ini dilakukan agar basis data tidak perlu melakukan cross-shard join, sehingga meminimalkan latensi. Tabel yang dibuat menjadi reference table adalah tabel Events, TicketCategories, TicketSales, TicketPackages, dan TicketAreas.

Selain itu, terdapat distributed table. Tabel ini merupakan sharding sebuah tabel berdasarkan baris. Untuk melakukan sharding diperlukan sebuah kolom yang menjadi shard key. Implementasi tugas akhir ini melakukan distribusi pada tabel TicketSeats, Orders, OrderItems, Invoices, dan IssuedTickets berdasarkan kolom ticket\_area\_id. Setiap tabel ini memiliki kolom ticket\_area\_id. Baris data pada entitas tersebut yang memiliki shard key yang sama akan berada pada shard yang sama, sehingga operasi yang berkaitan pada satu order dilakukan pada satu shard.

Pengoptimalan skema pada YugabyeDB sedikit berbeda dengan CitusData. Pada YugabyeDB, tidak ada reference table karena data memang sudah tereplikasi. Selain itu, tabel TicketSeats, Orders, OrderItems, Invoices, dan IssuedTickets tetap didistribusikan berdasarkan satu hash key yang sama yaitu kolom ticket\_area\_id, meski dengan sintaks inisiasi yang sedikit berbda.

\subsubsection{Flow Control}

Implementasi flow control dapat dibagi menjadi beberapa komponen penting, yaitu: event consumer, rate limiter, dan worker. Worker/ goroutine yang memproses pesanan bertugas untuk memproses pesanan yang masuk. Event consumer bertugas untuk listen event dari RabbitMQ. Rate limiter merupakan komponen yang mengatur banyaknya goroutine/ worker yang berjalan pada satu waktu. Jumlah limit ini dinamis dan bergantung pada kinerja sistem pada saat sistem sedang berjalan.

Algoritma yang digunakan untuk rate limiter adalah Gradient2. Algoritma ini mengatur limit concurrency berdasarkan gradien perubahan RTT dan \textit{long term exponentially smoothed average RTT}. Penggunaan algoritma ini dapat meminimalkan dampak outlier untuk trafik yang bersifat bursty \parencite{platinummonkey_go_concurrency_limits}.