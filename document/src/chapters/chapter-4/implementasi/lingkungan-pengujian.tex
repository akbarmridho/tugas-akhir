\subsection{Lingkungan Pengembangan}

\subsubsection{Lingkungan Pengembangan}

Tugas akhir ini dikembangkan pada perangkat laptop Asus TUF A15 FA506IC dengan spesifikasi sebagai berikut:

\begin{enumerate}
    \item Processor: AMD Ryzen 7 4800H 8 core 16 thread.
    \item RAM: 24GB DDR4 3200MHz.
    \item Windows 11 dengan WSL2 Ubuntu 22.04.
    \item Penyimpanan: 500GB + 1TB SSD NVME.
\end{enumerate}

\subsubsection{Lingkungan Pengujian Lokal}

Lingkungan pengujian lokal memiliki dua pengujian:

\begin{enumerate}
    \item \textit{Integration test} dilakukan pada Ubuntu 22.04 WSL2 dengan Docker Engine untuk orkestrasi dependensi.
    \item \textit{End-to-end testing} dilakukan pada dua kluster kubernetes lokal dengan bantuan K3d (sebuah abstraksi untuk menjalankan K3s sebagai Docker container). Versi image k3s yang digunakan adalah v1.30.12-rc1-k3s1.
          \begin{enumerate}
              \item Sistem backend dengan 1 node control plane dan 1 node worker.
              \item Agen penguji dengan 1 node control plane dan tanpa node worker.
          \end{enumerate}
\end{enumerate}

\subsubsection{Lingkungan Pengujian Penuh}

Lingkungan pengujian penuh dilakukan pada platform Hetzner Cloud dengan lokasi datacenter Germany. Setiap node dipastikan berada pada satu datacenter untuk meminimalkan latensi.

Setiap node kubernetes merupakan \textit{shared virtual machine}. Sistem \textit{dedicated} tidak digunakan karena biaya yang harus dikeluarkan jauh lebih mahal serta akun Hetzner membatasi sewa \textit{dedicated} CPU maksimal sebanyak 8 core. Setiap node kubernetes memiliki spesifikasi sebagai berikut:

\begin{enumerate}
    \item CPU Seri AMD EPYC 7002 dengan 16vCPU.
    \item RAM: 32GB.
    \item Storage: 360GB SSD NVME.
\end{enumerate}

Jumlah node yang digunakan adalah 5 node. Jumlah ini merupakan batas maksimal yang diperbolehkan oleh Hetzner untuk akun baru. Jumlah sumber daya yang digunakan pada skala ini dinilai cukup untuk menggambarkan beban di dunia nyata yang di-\textit{scale down}. Selain itu, penggunaan sumber daya yang lebih besar dari ini akan membutuhkan biaya yang sangat besar, sehingga tidak dinilai \textit{feasible} dari sisi biaya.

Terdapat dua kluster kubernetes berbeda yang digunakan:

\begin{enumerate}
    \item Sistem backend dengan 1 node control plane, 1 node worker, dan Hetzner Load Balancer tipe LB11.
    \item Agen penguji dengan 1 node control plane, 2 node worker, dan Hetzner Load Balancer tipe LB11.
\end{enumerate}
