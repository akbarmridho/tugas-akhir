\subsection{Lingkungan Pengembangan}

\subsubsection{Lingkungan Pengembangan}

Tugas akhir ini dikembangkan pada perangkat laptop Asus TUF A15 FA506IC dengan spesifikasi sebagai berikut:

\begin{enumerate}
    \item Processor: AMD Ryzen 7 4800H 8 core 16 thread.
    \item RAM: 24GB DDR4 3200MHz.
    \item Windows 11 dengan WSL2 Ubuntu 22.04.
    \item Penyimpanan: 500GB + 1TB SSD NVME.
\end{enumerate}

\subsubsection{Lingkungan Pengujian Lokal}

Lingkungan pengujian lokal memiliki dua pengujian:

\begin{enumerate}
    \item \textit{Integration test} dilakukan pada Ubuntu 22.04 WSL2 dengan Docker Engine untuk orkestrasi dependensi.
    \item \textit{End-to-end testing} dilakukan pada dua kluster kubernetes lokal dengan bantuan K3d (sebuah abstraksi untuk menjalankan K3s sebagai Docker container). Versi image k3s yang digunakan adalah v1.30.12-rc1-k3s1.
          \begin{enumerate}
              \item Sistem backend dengan 1 node control plane dan 1 node worker.
              \item Agen penguji dengan 1 node control plane dan tanpa node worker.
          \end{enumerate}
\end{enumerate}
