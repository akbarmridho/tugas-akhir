\subsection{Kakas yang Digunakan}
Dalam melakukan implementasi ini diperlukan beberapa kakas, diantaranya adalah sebagai berikut.
\begin{enumerate}
  \item \textit{Docker}, \textit{Docker Desktop} dan \textit{Kind} untuk dipakai sebagai \textit{containerization} dan \textit{cluster} kubernetes lokal.
  \item Implementasi \textit{service} bahasa pemrograman golang dan menggunakan beberapa kakas berikut
        \begin{enumerate}
          \item \textit{Kubernetes Go Client} untuk mengontrol \textit{cluster} kubernetes melalui kode Golang.
          \item \textit{Echo} sebagai \textit{server framework}
          \item \textit{Logrus} sebagai logger dari setiap aksi pada sistem
          \item \textit{Cobra dan Viper} sebagai \textit{CLI command} pada golang untuk memudahkan pemilihan \textit{entrypoint} dan \textit{environment variables}
          \item \textit{PQ, Sqlx, Go-migrate} sebagai kakas yang menghubungkan sistem dengan \textit{database}. Sqlx merupakan ekstensi dari library \textit{database/sql} milik golang. Go migrate digunakan untuk melakukan migrasi dari schema sql yang telah dibuat dan melakukan keep track dari versi schema yang sedang digunakan
          \item \textit{Validator} digunakan untuk memvalidasi \textit{request} yang masuk
        \end{enumerate}

  \item Implementasi \textit{dashboard} menggunakan \textit{vue} dan \textit{typescript} dan \textit{nuxt} sebagai framework.
        \begin{enumerate}
          \item \textit{NuxtUI} Sebagai kakas untuk memudahkan pembuatan \textit{UI}.
          \item \textit{Pinia} Sebagai kakas untuk manajemen data \textit{UI}.
          \item \textit{Zod} Sebagai object schema validator sebelum mengirimkan request.
        \end{enumerate}
\end{enumerate}