\subsection{Lingkungan Pengujian}

Lingkungan pengujian dilakukan pada platform Hetzner Cloud dengan lokasi pusat data Jerman. Setiap node dipastikan berada pada satu pusat data untuk meminimalkan latensi.

Setiap node kubernetes merupakan \textit{shared virtual machine}. Sistem \textit{dedicated} tidak digunakan karena biaya yang harus dikeluarkan jauh lebih mahal serta akun Hetzner membatasi sewa \textit{dedicated} CPU maksimal sebanyak 8 core. Setiap node kubernetes memiliki spesifikasi sebagai berikut:

\begin{enumerate}
    \item CPU Seri AMD EPYC 7002 dengan 16 vCPU.
    \item RAM: 32GB.
    \item Storage: 360GB SSD NVME.
\end{enumerate}

Jumlah node yang digunakan adalah 5 node. Jumlah ini merupakan batas maksimal yang diperbolehkan oleh Hetzner untuk akun baru. Jumlah sumber daya yang digunakan pada skala ini dinilai cukup untuk menggambarkan beban di dunia nyata meski dengan skala yang lebih kecil. Selain itu, penggunaan sumber daya yang lebih besar dari ini akan membutuhkan biaya yang sangat besar, sehingga tidak dinilai layak dikerjakan dari sisi biaya. Terdapat dua kluster kubernetes berbeda yang digunakan, yaitu:

\begin{enumerate}
    \item Sistem tiket dengan 1 \textit{node control plane}, 2 \textit{node} pekerja, dan Hetzner Load Balancer tipe LB31.
    \item Penguji dengan 1 \textit{node control plane}, 1 \textit{node} pekerja, dan Hetzner Load Balancer tipe LB11.
\end{enumerate}

Perbedaan penting antara Hetzner Load Balancer tipe LB11 dan LB31 adalah jumlah koneksi yang diperbolehkan. Pada tipe LB31, maksimal koneksi yang diperbolehkan adalah 40.000 koneksi, sedangkan tipe LB11 memiliki batas koneksi sebanyak 10.000 koneksi. Setiap kluster Kubernetes pada Hetzner dikonfigurasikan dengan bantuan kakas kube-hetzner/terraform-hcloud-kube-hetzner. Kakas ini melanjalankan K3s dengan basis sistem operasi OpenSUSE MicroOS.
