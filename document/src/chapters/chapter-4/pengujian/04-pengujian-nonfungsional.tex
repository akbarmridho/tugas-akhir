\subsection{Pengujian Non Fungsional}
Pada bagian ini, dilakukan pengujian terhadap kebutuhan non-fungsional sistem. Terdapat dua kebutuhan non-fungsional yang akan diuji yaitu \textit{Security dan Portability}. Pengujian akan menggunakan \textit{security testing} dan \textit{compatibility testing} untuk masing masing kebutuhan non-fungsional.

\subsubsection{\textit{Security Testing}}
\textit{Security Testing} bertujuan untuk menguji kebutuhan non-fungsional yaitu \textit{security}. Pengujian ini hanya terbatas pada masalah autentikasi dan authorisasi. Pengujian kebutuhan non-fungsional ini dilakukan dengan cara mencoba mengakses \textit{dashboard} dan \textit{service} tanpa meletakan token yang valid. Pengujian ini mencakup ID pengujian P47 hingga P50

\begin{enumerate}
  \item Mengakses \textit{dashboard} tanpa kredensial

        Ketika mengakses \textit{dashboard} tanpa memiliki kredensial, \textit{dashboard} memiliki \textit{middleware} yang akan mengecek token yang disimpan pada \textit{client}. Jika token tidak valid maka sistem akan langsung melakukan \textit{redirect} ke halaman login.

  \item Mengakses user \textit{service} tanpa kredensial

        Ketika mencoba untuk mengakses \textit{service} pada endpoint apapun, \textit{service} memiliki \textit{middleware}
        yang akan mengecek header authentikasi yang dikirimkan oleh client. Jika kredensial pada header tidak dicantumkan maka akan mengembalikan \textit{401 Unauthorized} seperti pada lampiran \ref{fig:akses-service-user}.

  \item Mengakses admin \textit{service} tanpa kredensial

        Admin endpoint memiliki \textit{middleware} validateAdminJWTKey. Ketika mencoba untuk mengakses \textit{endpoint} admin tanpa memberikan header X-Admin-Api-Key yang sesuai maka \textit{service} akan mengembalikan \textit{401 Unauthorized} seperti pada lampiran \ref{fig:akses-service-admin}.

  \item Mengakses \textit{url kubernetes} tanpa kredensial

        Seluruh cluster kubernetes akan mengexpose endpoint pada port 6443. Pengujian ini akan mengakses url kubernetes yaitu https://34.101.95.240:6443/. Ketika diakses, hasilnya akan menunjukan status \textit{401 Unauthorized} seperti pada lampiran \ref{fig:akses-service-kubernetes}.

\end{enumerate}

Setelah melakukan pengujian, pengujian dengan ID P47 hingga P50 berjalan sesuai dengan ekspektasi yaitu menunjukan kegagalan saat mengakses \textit{resource} yang diminta. Seluruh rekap pengujian dapat dilihat pada tabel \ref{tab:pengujian-nonfungsional-security}.

\subsubsection{Compatibility Testing}
\textit{Compatibility testing} dilakukan untuk menguji kebutuhan non-fungsional yaitu \textit{portability}. Pengujian kebutuhan non-fungsional ini dialkukan dengan tiga skenario yaitu mengakses \textit{dashboard} dari mobile serta mengakses dari berbagai \textit{browser}. Tujuan dari pengujian ini adalah memastikan bahwa \textit{dashboard} dapat dijalankan pada berbagai \textit{platform}. Pengujian ini mencakup pengujian dengan ID P52 hingga P54.

\begin{enumerate}
  \item Mengakses \textit{dashboard} dari perangkat mobile

        \textit{Dashboard} berhasil diakses melalui perangkat mobile dan dapat dilihat hasil dapat dilihat pada lampiran \ref{fig:akses-dashboard-mobile}.

  \item Mengakses \textit{dashboard} dari Chromium based browser

        \textit{Dashboard} berhasil diakses melalui browser \textit{chromium based} yaitu "Arc" dan "Google Chrome" yang dapat dilhat pada lampiran \ref{fig:akses-dashboard-chromium}.

  \item Mengakses \textit{dashboard} dari browser safari

        \textit{Dashboard} berhasil diakses melalui browser safari yang dapat dilhat pada lampiran \ref{fig:akses-dashboard-safari}.

\end{enumerate}

Setelah melakukan pengujian, pengujian dengan ID P51 hingga P53 berjalan sesuai dengan ekspektasi yaitu menunjukan kegagalan saat mengakses \textit{resource} yang diminta. Seluruh rekap pengujian dapat dilihat pada tabel \ref{tab:pengujian-nonfungsional-compatibility}.
Berdasarkan \textit{security testing} dan \textit{compatibility testing}, sistem \textit{remote deployment} yang dibuat telah memenuhi kebutuhan non-fungsional yang telah di definisikan pada tabel kebutuhan yang dapat dilihat pada lampiran \ref{tab:kebutuhan-non-fungsional}