\subsection{Alokasi Sumber Daya}

\subsubsection{Kluster Backend}

Berikut adalah alokasi sumber daya untuk komponen esensial pada kluster:

\begin{table}[htpb]
    \centering
    \caption{Resource Allocation for Services}
    \label{tab:resource_allocation}
    \begin{tabular}{|l|l|l|l|}
        \hline
        \textbf{Service Name} & \textbf{Request Allocation} & \textbf{Limit Allocation} & \textbf{Other} \\ \hline
        Prometheus            & 0.5/2Gi                     & 0.75/4Gi                  & PVC 50Gi       \\ \hline
        Alloy                 & 0.5/0.25Gi                  & 0.75/0.5Gi                & -              \\ \hline
        Grafana               & 0.25/0.5Gi                  & 0.5/0.75Gi                & PVC 10Gi       \\ \hline
        Loki                  & 0.5/1.5Gi                   & 0.75/2Gi                  & -              \\ \hline
        Nginx                 & 2/2Gi                       & 3/2.5Gi                   & -              \\ \hline
        Payment Redis         & 3x @ 0.5/0.75Gi             & 3x @ 0.5/0.75Gi           & -              \\ \hline
        Payment Backend       & 1/2Gi                       & 1/2Gi                     & -              \\ \hline
        Payment Worker        & 0.5/1Gi                     & 0.5/1Gi                   & -              \\ \hline
        Cert Manager          & 0.1/0.25Gi                  & 0.25/0.384Gi              & -              \\ \hline
        PGCat                 & 2/1Gi                       & 2/1Gi                     & -              \\ \hline
        Ticket Sanity         & 0.25/0.25Gi                 & 0.25/0.25Gi               & -              \\ \hline
    \end{tabular}
\end{table}

Tidak semua komponen dimasukkan ke dalam tabel tersebut. Terdapat beberapa komponen tambahan seperti Prometheus yang alokasinya kecil dan dapat diabaikan.

Berikut adalah alokasi sumber daya untuk varian tanpa flow control:

\begin{table}[htbp]
    \centering
    \caption{Alokasi Sumber Daya Tanpa Flow Control}
    \label{tab:nofc-allocation}
    \begin{tabular}{|l|l|l|l|}
        \hline
        \textbf{Service}            & \textbf{Postgres} & \textbf{Citus} & \textbf{YugaByteDB} \\ \hline
        Postgres Primary \& Replica & 2 x 3.75/8Gi      & -              & -                   \\ \hline
        Citusdata Coordinator       & -                 & 4.5/6Gi        & -                   \\ \hline
        Citusdata Worker            & -                 & 2 x 2/5Gi      & -                   \\ \hline
        YugabyteDB Master           & -                 & -              & ?                   \\ \hline
        YugabyteDB TServer          & -                 & -              & ?                   \\ \hline
        Redis Cluster               & 3 x 0.75/1.5Gi    & 3 x 0.75/1.5Gi & ?                   \\ \hline
        Ticket Backend              & 4 x 2/4Gi         & 4 x 2/4Gi      & ?                   \\ \hline
    \end{tabular}
\end{table}

Berikut adlaah alokasi sumebr daya untuk varian dengan flow control:

\begin{table}[htbp]
    \centering
    \caption{Alokasi Sumber Daya dengan Flow Control}
    \label{tab:service_comparison_fc}
    \begin{tabular}{|l|l|l|l|}
        \hline
        \textbf{Service}            & \textbf{Postgres} & \textbf{Citus} & \textbf{YugaByteDB} \\ \hline
        Postgres Primary \& Replica & ?                 & -              & -                   \\ \hline
        Citusdata Coordinator       & -                 & ?              & -                   \\ \hline
        Citusdata Worker            & -                 & ?              & -                   \\ \hline
        YugabyteDB Master           & -                 & -              & ?                   \\ \hline
        YugabyteDB TServer          & -                 & -              & ?                   \\ \hline
        Redis Cluster               & ?                 & ?              & ?                   \\ \hline
        RabbitMQ                    & ?                 & ?              & ?                   \\ \hline
        Ticket Backend              & ?                 & ?              & ?                   \\ \hline
        Ticket Worker               & ?                 & ?              & ?                   \\ \hline
    \end{tabular}
\end{table}

\subsubsection{Kluster Penguji}

Berikut adalah alokasi sumber daya untuk kluster penguji:

\begin{table}[htbp]
    \centering
    \caption{Alokasi Sumber Daya Kluster Penguji}
    \label{tab:test-cluster-allocation}
    \begin{tabular}{|l|l|l|l|}
        \hline
        \textbf{Service} & \textbf{Request Allocation} & \textbf{Limit Allocation} & \textbf{Other} \\ \hline
        Nginx            & 0.5/0.5Gi                   & 1/1Gi                     & -              \\ \hline
        Cert Manager     & 0.1/0.25Gi                  & 0.25/0.385Gi              & -              \\ \hline
        Grafana          & 0.5/0.75Gi                  & 1/1.5Gi                   & PVC 10Gi       \\ \hline
        Prometheus       & 2/4Gi                       & 4/8Gi                     & PVC 50Gi       \\ \hline
        K6 Run           & 9 x 4/8Gi                   & 9x 4/8Gi                  & -              \\ \hline
    \end{tabular}
\end{table}

