\subsection{Alokasi Sumber Daya}

\subsubsection{Kluster Sistem Tiket}

Daftar alokasi sumber daya untuk komponen esensial pada kluster tiket dijelaskan pada tabel \ref{tab:resource_allocation}.

\begin{table}[htpb]
    \centering
    \caption{Alokasi Sumber Daya Komponen Esensial}
    \label{tab:resource_allocation}
    \begin{tabular}{|l|l|l|l|}
        \hline
        \textbf{Komponen} & \textbf{\textit{Request Allocation}} & \textbf{\textit{Limit Allocation}} & \textbf{Tambahan} \\ \hline
        Prometheus            & 0.5/2Gi                     & 0.75/4Gi                  & PVC 50Gi       \\ \hline
        Alloy                 & 0.5/0.25Gi                  & 0.75/0.5Gi                & -              \\ \hline
        Grafana               & 0.25/0.5Gi                  & 0.5/0.75Gi                & PVC 20Gi       \\ \hline
        Loki                  & 0.5/1.5Gi                   & 0.75/2Gi                  & -              \\ \hline
        Cert Manager          & 0.1/0.25Gi                  & 0.25/0.385Gi              & -              \\ \hline
        Nginx                 & 2/2Gi                       & 3/2.5Gi                   & -              \\ \hline
        Payment Redis         & 3x @ 0.5/1.5Gi              & 3x @ 0.5/1.5Gi            & -              \\ \hline
        Payment Server        & 1/2Gi                       & 1/2Gi                     & -              \\ \hline
        Payment Worker        & 0.5/1Gi                     & 0.5/1Gi                   & -              \\ \hline
        PGCat                 & 2/1Gi                       & 2/1Gi                     & -              \\ \hline
        Ticket Sanity         & 0.25/0.25Gi                 & 0.25/0.25Gi               & -              \\ \hline
    \end{tabular}
\end{table}

Tidak semua komponen dimasukkan ke dalam tabel tersebut. Terdapat beberapa komponen tambahan seperti Prometheus Exporter, MinIO, dan lain-lain yang alokasinya kecil dan dapat diabaikan. Selain itu, komponen inti dari sistem ini memiliki besar \textit{request allocation} dan \textit{limit allocation} sama.

Di sisi lain, daftar alokasi sumber daya untuk varian sistem dengan dan tanpa pengendalian aliran sedikit berbeda. Daftar alokasi sumber daya untuk varian tanpa pengendalian aliran dijelaskan pada tabel \ref{tab:nofc-allocation} dan daftar alokasi sumber daya untuk varian dengan pengendalian aliran dijelaskan pada tabel \ref{tab:service_comparison_fc}


\begin{table}[htbp]
    \centering
    \caption{Alokasi Sumber Daya tanpa Pengendalian Aliran}
    \label{tab:nofc-allocation}
    \begin{tabular}{|l|l|l|l|}
        \hline
        \textbf{Komponen}           & \textbf{Postgres} & \textbf{Citus} & \textbf{YugaByteDB} \\ \hline
        Postgres Primary \& Replica & 2 x 8/16Gi        & -              & -                   \\ \hline
        Citusdata Coordinator       & -                 & 6/12Gi        & -                    \\ \hline
        Citusdata Worker            & -                 & 2 x 6/12Gi      & -                  \\ \hline
        YugabyteDB Master           & -                 & -              & 3 x 0.5/1Gi             \\ \hline
        YugabyteDB TServer          & -                 & -              & 3 x 6.5/13Gi            \\ \hline
        Redis Cluster               & 3 x 1/2Gi    & 3 x 1/2Gii & 3 x 1/2Gi                   \\ \hline
        Ticket Server              & 8 x 1/2Gi         & 8 x 1/2Gi      & 8 x 1/2Gi           \\ \hline
    \end{tabular}
\end{table}

\pagebreak

\begin{table}[htbp]
    \centering
    \caption{Alokasi Sumber Daya dengan pengendalian aliran}
    \label{tab:service_comparison_fc}
    \begin{tabular}{|l|l|}
        \hline
        \textbf{Komponen}           & \textbf{Postgres}     \\ \hline
        Postgres Primary \& Replica & 2 x 6/12Gi            \\ \hline
        Redis Cluster               & 3 x 1/2Gi             \\ \hline
        RabbitMQ                    & 2/4Gi                 \\ \hline
        Ticket Server               & 8 x 1/2Gi             \\ \hline
        Ticket Worker               & 4/6Gi                 \\ \hline
    \end{tabular}
\end{table}

Apabila diperhatikan, jumlah instans Ticket Server cukup banyak dengan alokasi sumber daya yang kecil. Hal ini dilakukan karena masalah fragmentasi sumber daya pada kluster Kubernetes. Apabila jumlah instans diperkecil dan alokasi sumber daya setiap instans diperbesar, kemungkinan instans tersebut tidak dapat dijadwalkan oleh kluster meningkat.

\subsubsection{Kluster Penguji}

Daftar alokasi sumber daya untuk kluster penguji dijelaskan pada tabel \ref{tab:test-cluster-allocation}.

\begin{table}[htbp]
    \centering
    \caption{Alokasi Sumber Daya Kluster Penguji}
    \label{tab:test-cluster-allocation}
    \begin{tabular}{|l|l|l|l|}
        \hline
        \textbf{Service} & \textbf{\textit{Request Allocation}} & \textbf{\textit{Limit Allocation}} & \textbf{Tambahan} \\ \hline
        Nginx            & 0.5/0.5Gi                    & 1/1Gi                      & -              \\ \hline
        Cert Manager     & 0.1/0.25Gi                   & 0.25/0.385Gi               & -              \\ \hline
        Grafana          & 0.5/0.75Gi                   & 1/1.5Gi                    & PVC 10Gi       \\ \hline
        Prometheus       & 2/4Gi                        & 4/8Gi                      & PVC 50Gi       \\ \hline
        K6 Run           & 12 x 2/4Gi                   & 12x 2/4Gi                  & -              \\ \hline
    \end{tabular}
\end{table}

