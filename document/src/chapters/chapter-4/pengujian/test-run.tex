\subsection{Daftar Pelaksanaan Pengujian}

Berikut adalah daftar pelaksanaan pengujian yang dijalankan:

\begin{table}[htbp]
    \centering
    \caption{Ringkasan Pelaksanaan Pengujian}
    \label{tab:execution-summary}
    \begin{tabular}{|l|l|l|l|l|l|l|}
        \hline
        \textbf{ID} & \textbf{Test} & \textbf{Data} & \textbf{FC} & \textbf{Database} & \textbf{Waktu Mulai} & \textbf{Waktu Berakhir} \\ \hline
        f1t1        & stress-2      & sf-4          & No          & postgres          & 2025-06-11 20:17     & 2025-06-11 20:35        \\ \hline
        f1t3        & stress-1      & sf-4          & No          & postgres          & 2025-06-11 21:35     & 2025-06-11 21:52        \\ \hline
        f1t2        & stress-0      & sf-4          & No          & postgres          & 2025-06-11 21:16     & 2025-06-11 21:28        \\ \hline
        f1t4        & sim-1         & s10-2         & No          & postgres          & 2025-06-11 21:55     & 2025-06-11 22:08        \\ \hline
        f2t1        & stress-2      & sf-4          & No          & citusdata         & 2025-06-12 08:31     & 2025-06-12 08:49        \\ \hline
        f2t2        & stress-1      & sf-4          & No          & citusdata         & 2025-06-12 08:51     & 2025-06-12 09:09        \\ \hline
        f2t3        & stress-0      & sf-4          & No          & citusdata         & 2025-06-12 09:11     & 2025-06-12 09:29        \\ \hline
        f2t4        & sim-1         & s10-2         & No          & citusdata         & 2025-06-12 09:35     & 2025-06-12 09:48        \\ \hline
        f3t1        & stress-2      & sf-4          & No          & yugabyte          & 2025-06-12 09:57     & 2025-06-12 10:15        \\ \hline
        f3t2        & stress-1      & sf-4          & No          & yugabyte          & 2025-06-12 10:49     & 2025-06-12 11:06        \\ \hline
        -           & stress-0      & sf-4          & No          & yugabyte          & -                    & -                       \\ \hline
        f3t3        & sim-1         & s10-2         & No          & yugabyte          & 2025-06-12 11:30     & 2025-06-12 11:43        \\ \hline
        f5t1        & stress-2      & sf-4          & Yes         & postgres          & 2025-06-16 21:58     & 2025-06-16 22:16        \\ \hline
        f5t2        & stress-1      & sf-4          & Yes         & postgres          & 2025-06-16 23:02     & 2025-06-16 23:19        \\ \hline
        f5t3        & stress-0      & sf-4          & Yes         & postgres          & 2025-06-16 23:21     & 2025-06-16 23:34        \\ \hline
        f5t4        & sim-1         & s10-2         & Yes         & postgres          & 2025-06-16 23:38     & 2025-06-16 23:50        \\ \hline
    \end{tabular}
\end{table}

Kolom "Test" menunjukkan varian skenario pengujian, kolom "Data" menunjukkan varian skenario penjualan, sedangkan kolom "FC" menunjukkan penggunaan pengendalian aliran. Waktu mulai dan waktu berakhir merupakan estimasi yang sudah diestimasikan menjadi 1 menit lebih awal dan 1 menit lebih akhir sejak pengujian dimulai/berakhir.

Sebagaimana ditunjukkan pada tabel di atas, terdapat satu pelaksanaan pengujian yang tidak dijalankan, yaitu varian tanpa pengendalian aliran dengan basis data YugabyteDB untuk skenario pengujian stress-0. Pengujian ini tidak dijalankan karena varian basis data tersebut tidak mampu menangani beban pada skenario pengujian dengan beban di bawahnya. Oleh karena itu, skenario pengujian ini tidak dijalankan.

Terdapat beberapa catatan tambahan terkait pelaksanaan pengujian. Catatan ini dibahas pada bagian lampiran \ref{apx:text-run-notes}.
