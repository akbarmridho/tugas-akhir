\subsubsection{Fungsionalitas pada Domain \textit{Device}}

Pengujian dengan ID P12 dilakukan dengan skenario \textit{user} yang telah terautentikasi ingin menambahkan \textit{device} yang dia miliki ke dalam sistem dengan \textit{node name} yang valid. Langkah langkah yang dilakukan:
\begin{enumerate}
  \item Mengunjungi halaman /devices
  \item Menekan tombol "Add Device"
  \item Mengisi \textit{field name} dengan "raspi-device-1"
  \item Mengisi \textit{field node name} dengan "testing-cluster-two-nodes-control-plane"
  \item Mengisi \textit{field type} dengan nilai "raspi"
  \item Mengisi label dengan "testing=true"
\end{enumerate}

Setelah pengujian dilakukan, muncul deskripsi \textit{device} yang telah dibuat pada tabel yang sebelumnya kosong. Pada bagian kanan bawah juga terdapat modal yang menunjukan bahwa pembuatan \textit{device} berhasil. Hasil dapat dilihat pada lampiran \ref{fig:pengujian-p12-success}.

Pengujian dengan ID P13 dilakukan skenario \textit{user} yang telah terautentikasi ingin menambahkan \textit{device} yang dia miliki ke dalam sistem dengan \textit{node name} yang tidak terdapat pada \textit{cluster}. Langkah langkah yang dilakukan:
\begin{enumerate}
  \item Mengunjungi halaman /devices
  \item Menekan tombol "Add Device"
  \item Mengisi \textit{field name} dengan "raspi-device-1"
  \item Mengisi \textit{field node name} dengan "testing-cluster-two-nodes-control-plan"
  \item Mengisi \textit{field type} dengan nilai "raspi"
  \item Mengisi label dengan "testing=true"
\end{enumerate}

Setelah pengujian dilakukan, muncul sebuah modal pada bagian kanan bawah juga terdapat modal yang menunjukan bahwa pembuatan \textit{device} gagal dengan pesan "node not found". Hasil dapat dilihat pada lampiran \ref{fig:pengujian-p12-failed}.

Pengujian dengan ID P14 dilakukan dengan skenario \textit{user} ingin melihat seluruh \textit{device} yang \textit{company} miliki. Pengujian ini dilakukan dengan cara mengunjungi halaman /devices dan melihat isi dari tabel yang tersedia. Berdasarkan pengujian P11 dapat dilihat bahwa tabel terisi dengan nilai \textit{device} yang telah dibuat, hal ini menunjukan bahwa \textit{user} dapat melihat perangkat yang \textit{company} miliki.

Pengujian dengan ID P15 dilakukan dengan skenario \textit{user} ingin menghapus salah satu \textit{device} yang ia miliki. Langkah langkah yang dilakukan:
\begin{enumerate}
  \item Mengunjungi halaman /devices
  \item Menekan tombol "titik tiga (\textit{elipsis horizontal})" yang berada pada ujung tabel
  \item Menekan tombol "delete" dari popup modal
\end{enumerate}

Setelah melakukan langkah langkah tersebut dapat dilhat bahwa tidak ada data pada tabel. Hal ini menunjukan bahwa \textit{device} berhasil dihapus dari sistem. Hasil pengujian dapat dilihat pada lampiran \ref{fig:pengujian-p15}.

Seluruh rekap pengujian pada domain \textit{devices} dapat dilihat pada lampiran \ref{tab:pengujian-domain-device}. Berdasarkan hasil yang diperoleh, terbukti bahwa kebutuhan fungsional dengan ID F10 hingga F15 telah terimplementasi dengan baik.
