\section{Analisis Hasil Pengujian}

\subsection{Siklus Penjualan Tiket}

\subsubsection{Skenario Beban Berkelanjutan}

grafik k6, request yg banyak apa, dll.

\subsubsection{Skenario Perebutan Tiket}

Grafik k6, request yg banyak apa, dll.

\subsection{Kinerja Selama Pengujian}

Jelaskan overallnya aja gimana (utilization blablabla). kasih tabel i guess terus lempar ke lampiran \ref{apx:test-run-performance}

\subsection{Kinerja Basis Data Terdistribusi}

Bandingkan kinerja setiap basis data. highlight how optimized postgres

postgres, citus, yugabyte.

tabel kueri juga -> jelasin efisiensi kueri tiap varian.

mention juga bahwa selama pengujian ternyata emang banyak banget kueri baca dan bebannya ga bisa dibiarin.

jelaskan citus yugabyte struggle juga buat handle kueri baca.

\subsubsection{Mengapa CitusData tidak perform}

beban pada koordinator. mostly read.

ambil hasil referensi TPC-C dan github issue.

jelaskan bagian citusdata yg bakal lebih optimal kalau pake stored procedure.

\subsubsection{Mengapa YugabyteDB tidak perform}

limit koneksi, butuh sumberdaya yang lebih besar. tidak cocok untuk high concurrent transaction. banyak terjadi kegagalan transaksi jadinya diretry.

ambil hasil referensi pengujian juga (pokoknya artikel yg bandingin raw performance)

\subsection{Pengoptimalan Kueri Baca}

buktikan kalau kueri ini banyak banget dipanggil -> pakai grafik/ data dari grafana.  

refer ke redis. pakai data pengujian fc postgres (jgn nofc). 

apakah hasilnya oke?

\subsection{Integritas Tiket selama Perebutan Tiket}

Apakah terjadi perbedaan data ketersediaan, dropper availability?

Apakah terjadi double booking?

\subsection{Penggunaan Pengendalian Aliran}

Jelasin berapa banyak request yg ditolak (respons 409). Gbs response based on code? atau apa gitu bandingin latensinya fak.

Dampaknya gimana (blm tau).

\subsection{Keterbatasan Pengujian}

List keterbatasan apa aja yang ada selama pengujian/ analisis yang mengakibatkan hasil kinerja bisa jauh berbeda.

jumlah virtual user yang dikit.

harusnya ngikutin distribusi sesuai gambar drpd load test

pakai resource lebih banyak agar lebih fair ke yugabytedb.

pisah penanganan kueri baca tulis agar resource usage utk tiap operasi bisa dibedain -> dampaknya lebih bisa diukur. kalau skrg banyak noise.
