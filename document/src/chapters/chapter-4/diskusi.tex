\section{Diskusi}

\subsection{Keterbatasan Pengujian}

List keterbatasan apa aja yang ada selama pengujian/ analisis yang mengakibatkan hasil kinerja bisa jauh berbeda.

jumlah virtual user yang dikit.

harusnya ngikutin distribusi sesuai gambar drpd load test

pakai resource lebih banyak agar lebih fair ke yugabytedb.

pisah penanganan kueri baca tulis agar resource usage utk tiap operasi bisa dibedain -> dampaknya lebih bisa diukur. kalau skrg banyak noise.

manajemen koneksi buat operasi yang kritikal. mungkin buat processing transaction pakai direct connection aja? atau pisah pool agar gak rebutan koneksi sama operasi baca.

better monitoring system, bisa pakai sampling tracing untuk semua dan backup data logs buat analisis kasus seperti yugabyte yang penuh kegagalan.

\subsection{Apa yang Sebaiknya Menjadi Fokus Pengoptimalan untuk Sistem Tiket?}

primary fokusin buat write, load balance query read ke instans lain baik redis or read replica.

this make latency for write berkurang.

citusdata juga bisa tapi koordinator dibikin replika juga.

buffer might be a good idea but rabbitmq is not it. maybe use in memory buffer or something.

might be filled with more insights kalau misalkan bisa improve sistem tiket ini di iterasi berikutnya (hopefully gak revisi buat apply further optimization dan ngelakuin pengujian ulang).
