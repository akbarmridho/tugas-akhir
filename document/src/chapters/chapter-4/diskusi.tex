\section{Diskusi}

\subsection{Keterbatasan Pengujian}
\label{keterbatasan-pengujian}

Setelah pengujian dan analisis dilakukan, terdapat temuan dan beberapa hal yang menjadi keterbatasan dan kekurangan pengujian pada penelitian ini. Apabila keterbatasan ini berhasil dilalui, hasil pengujian bisa saja memberikan hasil yang berbeda. Beberapa keterbatasan tersebut adalah:

\begin{enumerate}
    \item Karena keterbatasan sumber daya, jumlah pengguna virtual terbatas hingga 15.000 pada saat yang bersamaan. Idealnya, jumlah pengguna virtual ini dibuat lebih banyak hingga setidaknya 100.000 pengguna virtual pada saat yang bersamaan. Jumlah pengguna virtual yang lebih banyak dapat menunjukkan hasil yang lebih representatif sesuai dengan yang terjadi pada sistem tiket sesungguhnya.
    \item Keterbatasan jumlah pengguna juga membuat pengujian dengan distribusi lognormal (kasus perebutan tiket) tidak berjalan pada beban yang tinggi dengan penjualan tiket yang lebih banyak. Skenario pengujian yang berfokus pada \textit{arrival rate} dengan distribusi yang mendekati dunia nyata bisa saja memberikan temuan lain yang lebih representatif.
    \item Utilisasi sumber daya YugabyteDB sangat tinggi selama pengujian. Oleh karena itu, keterbatasan sumber daya dapat menjadi kendala yang membuat kinerja YugabyteDB tidak optimal pada pengujian ini. Oleh karena itu, YugabyteDB dapat diuji lagi dengan alokasi sumber daya yang lebih tinggi untuk mengetahui kinerja YugabyteDB yang sesungguhnya.
    \item Terdapat beberapa data pengujian yang tidak dikumpulkan dan dianalisis lebih mendalam, seperti \textit{log} aplikasi dan sampel \textit{trace} untuk mengetahui pembagian waktu eksekusi hingga level yang lebih kecil. Saat ini, metrik latensi hanya menunjukkan latensi keseluruhan sehingga bagian eksekusi yang lambat tidak dapat dianalisis.
    \item Penanganan kueri baca dan tulis tidak dipisah, sehingga penggunaan sumber daya untuk setiap operasi tidak dapat dibedakan. Oleh karena itu, dampak dan penggunaan sumber daya masing-masing operasi tidak dapat diukur.
\end{enumerate}

\pagebreak

\subsection{Pengembangan Lebih Lanjut}
\label{pengembangan-lebih-lanjut}

Meskipun tidak semua pengoptimalan memberikan hasil yang sesuai harapan, terdapat banyak temuan menarik yang dapat memberikan arah untuk pengembangan sistem tiket yang optimal untuk menangani kasus dengan beban tinggi. Temuan ini memberikan arah yang dapat menjadi fokus dalam pengoptimalan sistem tiket, seperti:

\begin{enumerate}
    \item Manajemen koneksi basis data merupakan hal yang kritis dan harus ditangani dengan baik. Manajemen koneksi ini idealnya juga dilakukan pada level operasi. Sebagai contoh, \textit{pool} koneksi untuk operasi baca dan tulis sebaiknya dipisah agar \textit{contention} pada koneksi untuk operasi baca tidak mengganggu keberjalanan pemrosesan pesanan.
    \item Selain pemisahan koneksi, pengurangan beban pada \textit{primary instance} juga penting. Pengurangan beban ini dapat dilakukan dengan melakukan \textit{query load balancing} untuk operasi baca pada instans \textit{replica}. Selain pada kluster PostgreSQL, penggunaan \textit{primary} dan \textit{replica} ini juga dapat dilakukan pada kluster CitusData, sehingga beban koordinator dapat dibagi dan koordinator utama dapat fokus menangani operasi yang lebih kritis. Selain itu, sebagaimana ditunjukkan pada pengoptimalan operasi ketersediaan, sebagian operasi baca juga dapat dilakukan melalui Redis alih-alih langsung melalui basis data. Dengan begitu, instans \textit{primary} dapat fokus menangani operasi yang kritis seperti pemrosesan tiket.
    \item Pengoptimalan operasi baca ketersediaan kursi (bukan agregat area) dengan Redis alih-alih menggunakan tembolok dengan waktu hidup singkat.
    \item Penggunaan antrean merupakan ide yang baik, tetapi implementasi pendekatan tersebut harus mempertimbangkan aspek latensi yang dapat menyebabkan waktu pemrosesan menjadi jauh lebih tinggi. Antrean dengan menggunakan Redis atau pada level aplikasi merupakan pendekatan yang dapat dieksplorasi lebih lanjut.
    \item Pengoptimalan implementasi yang menyesuaikan dengan kelebihan dan keterbatasan varian basis data agar dapat berjalan dengan jauh lebih optimal.
\end{enumerate}
