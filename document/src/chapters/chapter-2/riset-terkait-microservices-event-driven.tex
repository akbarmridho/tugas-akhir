\subsection{Implementasi Arsitektur \textit{Microservices} Menggunakan Komunikasi \textit{Event-Driven} Pada Aplikasi Pemesanan Tiket Acara (Jeeves)}

Tugas akhir ini membahas aplikasi pemesanan tiket acara dengan arsitektur \textit{microservice} dengan komunikasi berbasis peristiwa. Tujuan dari penelitian tersebut adalah membuat aplikasi tiket yang tahan akan kegagalan sehingga layanan yang masih berjalan tetap dapat memenuhi aksi yang lain. Untuk itu, arsitektur \textit{microservice} diimplementasikan dengan cara \textit{loosely-coupled}. Arsitektur Jeeves diilustrasikan pada Gambar \ref{fig:jeeves-architecture}. Agar hal tersebut dapat dipenuhi, komunikasi berbasis peristiwa digunakan. Agar ketergantungan antar layanan berkurang, seluruh data yang diperlukan oleh sebuah layanan diikutsertakan pada peristiwa yang dikirim \parencite{microservicesEventDriven}.

\begin{figure}[htbp]
    \centering
    \includegraphics[width=1\textwidth]{resources/chapter-2/jeeves.png}
    \caption{Arsitektur Jeeves \parencite{microservicesEventDriven}}
    \label{fig:jeeves-architecture}
\end{figure}

Penelitian ini mengimplementasikan fungsionalitas dasar seperti otentikasi pengguna, pembuatan tiket, reservasi tiket, dan pembayaran tiket. Terdapat lima \textit{microservice} yang dibuat, yaitu layanan otentikasi, layanan tiket, layanan pemesanan, layanan pembayaran, dan layanan kedaluwarsa. Setiap layanan selain layanan kedaluwarsa memiliki basis data masing-masing dengan menggunakan MongoDB. Hanya layanan kedaluwarsa yang menggunakan Redis. Selain itu, setiap komunikasi yang terjadi secara asinkron dilakukan melalui NATS. Untuk melakukan validasi otentikasi, sistem ini menggunakan JWT dengan rahasia bersama yang diatur oleh Kubernetes. Pendekatan ini memungkinkan layanan independen terhadap layanan otentikasi \parencite{microservicesEventDriven}.

Penelitian ini berhasil mengimplementasikan arsitektur aplikasi untuk pemesanan tiket acara yang tahan kegagalan dengan pendekatan \textit{microservice} dan berbasis peristiwa. Meskipun begitu, penelitian ini tidak membahas dan menguji aspek skalabilitas dan elastisitas dari sistem ini.