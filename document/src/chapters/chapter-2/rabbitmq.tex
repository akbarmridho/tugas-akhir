\section{RabbitMQ}

\begin{figure}[htbp]
    \centering
    \includegraphics[width=1\textwidth]{resources/chapter-2/rabbitmq.jpeg}
    \caption{RabbitMQ Flow \parencite{royNatsRmqKafka}}
    \label{fig:rabbitmq-flow}
\end{figure}

RabbitMQ merupakan salah satu message broker yang engimplementasikan model penyimpanan berbasis \textit{queue} dan mendukung \textit{message persistence} baik secara \textit{persistent} maupun \textit{ephemeral}. Dari segi performa, RabbitMQ memiliki throughput hingga 60 ribu pesan per detik dengan latensi tergolong rendah (dalam milidetik). Model konsumennya bersifat \textit{push-based}, artinya broker secara aktif mengirimkan pesan ke konsumen. RabbitMQ mendukung berbagai protokol seperti AMQP, MQTT, dan STOMP. Jaminan urutan pengiriman pesan (ordering guarantee) pada RabbitMQ berlaku pada level antrean \parencite{arshadChoosingTheRightMessaging}.