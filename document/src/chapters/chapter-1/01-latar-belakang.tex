\section{Latar Belakang}
\label{sec:latar-belakang}

Ticketmaster merupakan salah satu \textit{platform} yang melayani penjualan tiket Taylor Swift The Eras Tour. \textit{Platform} ini melayani penjualan dua juta tiket pada tanggal 15 November 2022 \parencite{swiftTicketmaster}. Proses pelayanan ini tidak selalu berjalan lancar karena sistem Ticketmaster seringkali mengalami kegagalan pada saat penjualan tiket. Bagaimana tidak, dari 625 ribu tiket yang dijual saat tur kedua di Amerika Serikat, terdapat 14 juta pengguna yang ingin mendapatkan tiket tersebut \parencite{USTaylorSwift}. Saat penjualan tiket tur di Singapura, terdapat 300 ribu tiket yang akan dijual dengan 22 juta peminat \parencite{asiaTaylorSwift}. Hal serupa terjadi di India pada saat BookMyShow melayani penjualan tiket Coldplay \parencite{coldplayBookMyShow}. Peristiwa ini cukup menarik apabila dilihat dari sisi teknis. Apakah ada pendekatan yang mampu mengoptimalkan sistem ini agar mampu menangani jutaan pengguna pada satu waktu tanpa mengalami kegagalan?

Skalabilitas sistem tiket memiliki karakteristik yang unik, terutama pada kasus penjualan tiket yang tinggi peminat. \textit{Traffic} sistem ini \textit{bursty}, sehingga membutuhkan sistem yang elastis. Selain itu, terjadi \textit{request contention} saat proses pemesanan tiket karena terdapat banyak pihak yang ingin melakukan pemesanan pada kursi yang sama. Tantangan sistem ini dapat dibagi menjadi dua, yaitu bagaimana cara meningkatkan \textit{throughput} penjualan tiket serta bagaimana cara meningkatkan stabilitas sistem saat berada pada beban yang tinggi.

Pendekatan untuk meningkatkan \textit{throughput} telah banyak dikembangkan, mulai dari penggunaan \textit{event-driven system}, \textit{sharding}, basis data tereplikasi, hingga pengembangan basis data terdistribusi. Selain itu, skema \textit{flow control} banyak digunakan untuk menjaga stabilitas sistem. Penelitian ini akan menguji berbagai alternatif basis data terdistribusi serta penggunaan \textit{flow control} dalam menjaga stabilitas sistem saat proses pemesanan tiket. Berbagai alternatif akan diuji untuk mengetahui pendekatan dengan kinerja yang lebih baik.
