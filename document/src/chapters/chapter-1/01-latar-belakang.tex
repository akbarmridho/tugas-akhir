\section{Latar Belakang}
\label{sec:latar-belakang}

Fenomena penjualan tiket untuk acara berskala besar secara daring telah menjadi sorotan utama dalam industri digital. Salah satu contoh paling nyata adalah penjualan tiket Taylor Swift The Eras Tour melalui platform Ticketmaster, yang melayani penjualan dua juta tiket pada tanggal 15 November 2022 \parencite{swiftTicketmaster}. Proses pelayanan ini tidak selalu berjalan lancar karena sistem Ticketmaster seringkali mengalami kegagalan akibat lonjakan permintaan. Skala tantangannya terlihat jelas ketika 14 juta pengguna mencoba memperebutkan 625 ribu tiket yang tersedia untuk tur di Amerika Serikat \parencite{USTaylorSwift}. Hal serupa terjadi saat 22 juta peminat mencoba membeli 300 ribu tiket untuk konser di Singapura \parencite{asiaTaylorSwift}, serta kasus penjualan tiket Coldplay di India oleh BookMyShow \parencite{coldplayBookMyShow}. Peristiwa-peristiwa ini menyoroti sebuah tantangan teknis fundamental: bagaimana merancang sistem tiket berskala besar yang paling optimal?

Secara garis besar, alur sistem penjualan tiket daring dimulai ketika jutaan pengguna secara bersamaan mengakses platform untuk memeriksa ketersediaan tiket. Tahapan utama meliputi:

\begin{enumerate}
    \item Pemeriksaan Ketersediaan (\textit{Read-Heavy}). Mayoritas pengguna melakukan permintaan baca untuk mengetahui tiket yang masih tersedia. Pada fase ini, beban baca sangat dominan dan dapat mencapai jutaan kueri per detik.
    \item Pemesanan dan Reservasi Sementara (\textit{Write-Heavy}). Ketika pengguna memilih tiket, sistem harus segera mencadangkan kursi tersebut agar tidak diperebutkan pengguna lain. Proses ini merupakan transaksi tulis yang kritis karena menyangkut konsistensi data.
    \item Pembayaran dan Konfirmasi. Setelah reservasi, pengguna menyelesaikan pembayaran, dan tiket dikonfirmasi secara permanen. Pada tahap ini, integritas data antara modul tiket dan pembayaran menjadi sangat penting.
\end{enumerate}

Tantangan utama muncul pada fase pertama dan kedua, ketika terjadi kondisi pacu (\textit{race condition}) atas kursi yang sama serta \textit{bottleneck} pada basis data akibat operasi baca masif dan tulis transaksional yang berlangsung bersamaan.

Untuk menangani masalah tersebut, sejumlah pendekatan teknologi telah dikembangkan:

\begin{enumerate}
    \item Basis Data Relasional Tradisional (PostgreSQL). PostgreSQL dalam konfigurasi kluster umum digunakan karena kestabilan dan konsistensinya, tetapi skalabilitas tulisnya terbatas.
    \item Basis Data Relasional Terdistribusi. Solusi seperti \textit{CitusData} (ekstensi distribusi untuk PostgreSQL) dan \textit{YugabyteDB} (basis data terdistribusi \textit{native}) dirancang untuk mendukung skalabilitas horizontal, sehingga transaksi dapat diproses paralel di banyak node.
    \item \textit{In-Memory Data Store} (Redis). Untuk mengurangi beban baca langsung ke basis data, Redis digunakan sebagai cache atau penyimpan agregat yang dapat merespons kueri ketersediaan dalam hitungan milidetik.
    \item Skema Pengendalian Aliran (\textit{Flow Control}). Strategi seperti penolakan permintaan lebih awal dan penggunaan antrean (\textit{queuing system} seperti RabbitMQ) diterapkan untuk mencegah kelebihan beban pada basis data dengan mengatur arus masuk permintaan.
\end{enumerate}

Selain teknologi tersebut, terdapat juga pendekatan berbasis NoSQL seperti MongoDB yang menawarkan fleksibilitas penyimpanan dokumen, meskipun seringkali menghadapi keterbatasan dalam menjaga konsistensi kuat untuk transaksi pemesanan tiket.

Meskipun berbagai teknologi ini tersedia, kajian yang secara khusus menguji dan membandingkan kombinasi strategi arsitektur—mulai dari PostgreSQL, CitusData, YugabyteDB, Redis, hingga mekanisme pengendalian aliran—dalam skenario lonjakan permintaan ekstrem masih terbatas. Belum ada rujukan yang jelas mengenai pendekatan mana yang paling efektif untuk mengoptimalkan laju pemrosesan pesanan (\textit{throughput}), mengatasi beban baca yang masif, serta menjaga kestabilan sistem agar tetap responsif.

Oleh karena itu, tugas akhir ini dilakukan untuk mengisi kesenjangan tersebut dengan menganalisis, mengimplementasikan, serta menguji secara komprehensif berbagai pendekatan arsitektural. Hasil tugas akhir diharapkan dapat memberikan rekomendasi yang lebih jelas mengenai rancangan sistem tiket berskala besar yang optimal, andal, dan dapat diimplementasikan pada kasus nyata.
