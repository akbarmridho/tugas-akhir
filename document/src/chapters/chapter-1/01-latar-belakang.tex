\section{Latar Belakang}
\label{sec:latar-belakang}

Fenomena penjualan tiket untuk acara berskala besar secara daring telah menjadi sorotan utama dalam industri digital. Salah satu contoh paling nyata adalah penjualan tiket Taylor Swift The Ears Tour melalui platform Ticketmaster, yang melayani penjualan dua juta tiket pada tanggal 15 November 2022 \parencite{swiftTicketmaster}. Proses pelayanan ini tidak selalu berjalan lancar karena sistem Ticketmaster seringkali mengalami kegagalan akibat lonjakan permintaan. Skala tantangannya terlihat jelas ketika 14 juta pengguna mencoba memperebutkan 625 ribu tiket yang tersedia untuk tur di Amerika Serikat \parencite{USTaylorSwift}. Hal serupa terjadi saat 22 juta peminat mencoba membeli 300 ribu tiket untuk konser di Singapura \parencite{asiaTaylorSwift}, serta kasus penjualan tiket Coldplay di India oleh BookMyShow \parencite{coldplayBookMyShow}. Peristiwa-peristiwa ini menyoroti sebuah tantangan teknis fundamental: bagaimana merancang sistem yang mampu menangani jutaan pengguna secara bersamaan tanpa mengalami kegagalan?

Skalabilitas sistem tiket memiliki karakteristik yang unik, terutama pada kasus penjualan tiket yang tinggi peminat. Pengguna datang secara eksponensial dalam waktu singkat, sehingga menuntut sistem agar mampu menangani lonjakan permintaan. Lonjakan permintaan ini meliputi dua operasi utama: operasi baca untuk memeriksa ketersediaan tiket dan operasi tulis untuk memproses pemesanan. Selain itu, perebutan sumber daya terjadi saat proses pemesanan tiket karena terdapat banyak pengguna yang ingin melakukan pemesanan pada kursi yang sama. Selain merupakan operasi yang paling masif, operasi baca ketersediaan juga harus menjamin keterbaruan data.

Di sisi lain, berbagai pendekatan arsitektural telah dikembangkan untuk sistem dengan beban tinggi. Beberapa di antaranya adalah penggunaan arsitektur berbasis peristiwa (\textit{event-driven architecture}), pemartisian basis data, replikasi basis data, hingga pengembangan basis data terdistribusi. Selain itu, skema pengendalian aliran (\textit{flow control}) seperti \textit{rate limiting} dan pemutusan beban juga umum digunakan untuk menjaga stabilitas sistem saat berada di bawah tekanan. Meskipun begitu, studi yang secara spesifik menguji kombinasi dan performa dari berbagai pendekatan ini dalam skenario lonjakan permintaan ekstrem pada sistem tiket masih terbatas. Belum ada referensi yang jelas mengenai pendekatan mana yang paling efektif untuk mengoptimalkan laju pemrosesan pesanan, penanganan operasi baca, dan menjamin kestabilan pemrosesan pesanan.

Kesenjangan inilah yang menjadi motivasi utama dilakukannya penelitian ini. Diperlukan sebuah investigasi yang mendalam untuk menganalisis dan menguji berbagai pendekatan arsitektural, sehingga sebuah solusi yang optimal dan andal dapat dirumuskan untuk sistem penjualan tiket. Penelitian ini akan berfokus pada evaluasi penggunaan basis data terdistribusi dengan arsitektur yang berbeda, pengoptimalan operasi ketersediaan baca, serta pengoptimalan untuk menjaga kestabilan laju pemrosesan pesanan.
