\section{Latar Belakang}
\label{sec:latar-belakang}

Ticketmaster merupakan salah satu \textit{platform} yang melayani penjualan tiket Taylor Swift The Eras Tour. \textit{Platform} ini melayani penjualan dua juta tiket pada tanggal 15 November 2022 \parencite{swiftTicketmaster}. Proses pelayanan ini tidak selalu berjalan lancar karena sistem Ticketmaster seringkali mengalami kegagalan pada saat penjualan tiket. Bagaimana tidak, dari 625 ribu tiket yang dijual saat tur kedua di Amerika Serikat, terdapat 14 juta pengguna yang ingin mendapatkan tiket tersebut \parencite{USTaylorSwift}. Saat penjualan tiket tur di Singapura, terdapat 300 ribu tiket yang akan dijual dengan 22 juta peminat \parencite{asiaTaylorSwift}. Hal serupa terjadi di India pada saat BookMyShow melayani penjualan tiket Coldplay \parencite{coldplayBookMyShow}. Peristiwa ini cukup menarik apabila dilihat dari sisi teknis. Apakah ada pendekatan yang mampu mengoptimalkan sistem ini agar mampu menangani jutaan pengguna pada satu waktu tanpa mengalami kegagalan?

Skalabilitas sistem tiket memiliki karakteristik yang unik, terutama pada kasus penjualan tiket yang tinggi peminat. \textit{Traffic} sistem ini \textit{bursty}, sehingga membutuhkan sistem yang elastis. Operasi pembacaan ketersediaan tiket harus selalu mengembalikan data terbaru padahal pada saat yang bersamaan terdapat banyak tiket yang dipesan. Penggunaan \textit{caching} tidak akan cocok karena data cepat \textit{stale}, sedangkan kueri langsung ke basis data akan membebani sistem. Selain itu, akan ada banyak penulisan pada relasi data yang sama pada saat yang bersamaan.

Pendekatan untuk mengoptimalkan sistem telah banyak dikembangkan, mulai dari pendekatan \textit{database inside-out}, penggunaan \textit{event-driven system}, pemrosesan \textit{stream}, serta pengoptimalan pada basis data itu sendiri seperti penggunaan \textit{read replica}, \textit{sharding}, dan membuat basis data menjadi terdistribusi. Tentu tidak semua pendekatan cocok diterapkan pada kasus ini. Oleh karena itu, penelitian ini akan menganalisis berbagai alternatif arsitektur untuk sistem tiket. Setiap alternatif akan diuji untuk mengetahui pendekatan dengan kinerja yang lebih baik.