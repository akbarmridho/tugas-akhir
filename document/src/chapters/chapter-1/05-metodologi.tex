\section{Metodologi}

Berikut adalah metodologi dan tahapan yang akan dilalui selama pelaksanaan tugas akhir:

\begin{enumerate}
      \item \textbf{Kebutuhan dan Analisis Masalah}

            Tahap ini akan menganalisis komponen apa saja yang diperlukan pada sistem tiket, lalu mengidentifikasi karakteristik beban dan \textit{bottleneck} yang mungkin terjadi pada sistem ini.

      \item \textbf{Analisis dan Perancangan Solusi}

            Setelah mengidentifikasi permasalahan, dilakukan analisis perancangan solusi yang bertujuan untuk mengoptimalkan operasi yang mengalami \textit{bottleneck} pada sistem tiket.

      \item \textbf{Implementasi}

            Gagasan hasil rancangan solusi akan dikembangkan dan diimplementasikan. Hasil dari tahap ini berupa implementasi yang mengoptimalkan sistem tiket. Selain itu, tahapan ini juga mengimplementasikan hal lain yang mendukung proses penerapan dan pengujian sistem.

      \item \textbf{Pengujian dan Analisis Hasil}

            Setelah implementasi berhasil dilakukan, dilakukan serangkaian pengujian untuk memastikan kebenaran implementasi dan peningkatan kinerja yang diperoleh dibandingkan dengan alternatif lainnya. Setelah pengujian dilakukan, hasil pengujian akan dianalisis untuk memperoleh wawasan dan perbandingan dari setiap arsitektur solusi.

\end{enumerate}