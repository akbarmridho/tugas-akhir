\section{Batasan Masalah}
\label{sec:batasan-masalah}

Berikut adalah batasan masalah yang diambil dalam pelaksanaan penelitian ini:

\begin{enumerate}
  \item Penelitian ini tidak membahas desain sistem dari sisi pengalaman pengguna, seperti penggunaan sistem antrean virtual yang cukup umum digunakan oleh sistem tiket pada saat ini.
  \item Konteks tiket pada penelitian ini adalah tiket untuk keperluan acara seperti konser atau acara lainnya, bukan tiket untuk keperluan transportasi seperti tiket kereta, bus, atau hal sejenis lainnya.
  \item Penelitian ini berfokus pada pengoptimalan penjualan tiket untuk satu acara pada satu waktu penjualan. Pengoptimalan penjualan tiket untuk banyak acara dengan beban tinggi cukup dioptimalkan dengan membedakan waktu penjualan. Sebuah acara yang dilaksanakan selama beberapa hari pada waktu yang berdekatan dan dilaksanakan pada tempat yang sama dianggap sebagai satu acara yang sama.
  \item Penelitian ini berfokus pada pengoptimalan dengan solusi \textit{open-source} dan solusi yang tidak bergantung pada \textit{vendor} tertentu.
  \item Penelitian ini berfokus pada pengoptimalan sistem untuk melayani pengguna dalam satu \textit{region} yang sama karena mayoritas peminat suatu acara dapat diasumsikan berasal dari satu \textit{region} yang sama.
  \item Penelitian ini tidak mengimplementasikan aspek \textit{fault tolerance} dan tidak menguji skenario \textit{failover}.
\end{enumerate}

