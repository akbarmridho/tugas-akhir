\section{Sistematika Pembahasan}

Proses pengembangan dan pengujian untuk pengoptimalan sistem tiket acara berskala besar ini akan dibagi menjadi lima bab utama, yang terdiri dari:

\begin{enumerate}
      \item \textbf{Pendahuluan}

            Bab I akan menguraikan gagasan utama dari penelitian ini. Bagian ini mencakup latar belakang masalah terkait tantangan penjualan tiket acara berskala besar, rumusan masalah mengenai skalabilitas dan stabilitas sistem, tujuan untuk mengembangkan arsitektur yang optimal, batasan masalah, metodologi penelitian yang digunakan, hingga sistematika pembahasan yang akan memandu alur laporan tugas akhir ini.

      \item \textbf{Kajian Pustaka}

            Bab II akan membahas studi literatur dan landasan teori yang menjadi dasar penelitian. Topik yang dibahas meliputi teknik penskalaan basis data relasional seperti replikasi dan pemartisian, konsep basis data relasional terdistribusi dengan studi kasus pada CitusData dan YugabyteDB, teori mengenai pengendalian aliran (\textit{flow control}), serta teknologi pendukung seperti Redis dan RabbitMQ. Bab ini juga akan meninjau penelitian terkait yang telah ada sebelumnya mengenai arsitektur sistem tiket.

      \item \textbf{Analisis Persoalan dan Rancangan Solusi}

            Bab III akan menjelaskan analisis mendalam terhadap permasalahan yang ada pada sistem tiket konvensional saat menghadapi beban tinggi. Analisis ini diikuti dengan perancangan berbagai alternatif solusi untuk meningkatkan laju pemrosesan transaksi, mengoptimalkan operasi baca, menjaga integritas data, dan menerapkan skema pengendalian aliran. Bab ini juga akan merinci arsitektur solusi yang dipilih untuk diimplementasikan, termasuk variasi penggunaan basis data dan komponen sistem.

      \item \textbf{Implementasi dan Pengujian}

            Bab IV akan memaparkan detail teknis dari proses implementasi rancangan yang telah dibuat, termasuk lingkungan pengembangan, kakas yang digunakan, dan konfigurasi deployment pada kluster Kubernetes. Selanjutnya, bab ini akan menjelaskan secara rinci metodologi pengujian yang dilakukan, mencakup skenario pengujian beban berkelanjutan dan simulasi perebutan tiket. Hasil dari setiap skenario pengujian akan dianalisis secara komparatif untuk mengevaluasi kinerja setiap variasi arsitektur.

      \item \textbf{Kesimpulan dan Saran}

            Bab V akan menjadi bagian penutup dari laporan penelitian. Bab ini berisi rangkuman kesimpulan yang ditarik dari hasil analisis dan pengujian, seperti perbandingan kinerja antara kluster PostgreSQL, CitusData, dan YugabyteDB, serta efektivitas strategi pengendalian aliran. Selain itu, bab ini juga akan memberikan saran untuk pengembangan lebih lanjut yang dapat dilakukan di masa depan guna menyempurnakan sistem yang telah dibangun.

\end{enumerate}
