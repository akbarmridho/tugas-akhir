\section{Sistematika Pembahasan}

Proses pengembangan dan pengujian untuk pengoptimalan sistem tiket acara berskala besar ini akan dibagi menjadi lima bab utama, yang terdiri dari:

\begin{enumerate}
    \item \textbf{Pendahuluan}

          Bab I akan menguraikan gagasan utama dari Tugas akhir ini. Bagian ini mencakup latar belakang masalah terkait tantangan penjualan tiket acara berskala besar, rumusan masalah, tujuan, batasan masalah, metodologi Tugas akhir, hingga sistematika pembahasan yang akan memandu alur laporan tugas akhir ini.

    \item \textbf{Kajian Pustaka}

          Bab II akan membahas studi literatur dan landasan teori yang menjadi dasar Tugas akhir. Topik yang dibahas meliputi teknik penskalaan basis data relasional seperti replikasi dan pemartisian, konsep basis data relasional terdistribusi dengan studi kasus pada CitusData dan YugabyteDB, teori mengenai pengendalian aliran, serta teknologi pendukung seperti Redis dan RabbitMQ. Bab ini juga akan meninjau Tugas akhir terkait yang telah ada sebelumnya mengenai arsitektur sistem tiket.

    \item \textbf{Analisis Persoalan dan Rancangan Solusi}

          Bab III akan menjelaskan analisis mendalam terhadap permasalahan yang ada pada sistem tiket konvensional saat menghadapi beban tinggi. Analisis ini diikuti dengan perancangan berbagai alternatif solusi untuk meningkatkan laju pemrosesan transaksi, mengoptimalkan operasi baca, menjaga integritas data, dan menerapkan skema pengendalian aliran.

    \item \textbf{Implementasi dan Pengujian}

          Bab IV akan memaparkan detail teknis dari proses implementasi rancangan yang telah dibuat, skenario pengujian yang dilakukan, dan analisis hasil pengujian.

    \item \textbf{Kesimpulan dan Saran}

          Bab V akan menjadi bagian penutup dari laporan Tugas akhir. Bab ini berisi rangkuman kesimpulan Tugas akhir dan saran untuk pengembangan lebih lanjut.

\end{enumerate}
