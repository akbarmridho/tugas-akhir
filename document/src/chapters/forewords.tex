\chapter*{Kata Pengantar}
\addcontentsline{toc}{chapter}{KATA PENGANTAR}

Puji dan syukur penulis panjatkan kepada Tuhan Yang Maha Esa atas berkat dan rahmatnya, laporan tugas akhir yang berjudul "\thetitle" dapat diselesaikan dalam rangka memenuhi syarat kelulusan tingkat sarjana. Harus diakui, pengerjaan tugas akhir ini didukung oleh banyak pihak. Penulis ingin mengucapkan terima kasih kepada:

\begin{enumerate}
  \item Bapak Achmad Imam Kistijantoro, S.T, M.Sc., Ph.D. selaku dosen pembimbing atas segala bentuk dukungan yang telah diberikan dan kesabarannya dalam membimbing penulis serta memberikan saran dalam pengerjaan tugas akhir.
  \item Bapak Dr.techn. Saiful Akbar, S.T., M.T. dan Dion Tanjung, S.Kom., M.sc., Ph.D selaku dosen penguji atas segala masukan dan kritik yang telah diberikan terhadap tugas akhir penulis.
  \item Ibu Robithoh Annur, S.T., M.Eng., Ph.D. dan Tricya Esterina Widagdo, ST., M.Sc. selaku dosen koordinator tim tugas akhir.
  \item Seluruh dosen program studi Teknik Informatika ITB yang telah memberikan ilmu pengetahuan yang sangat berharga bagi penulis.
  \item Teman-teman penulis, khususnya anggota dari grup "lesgo wahoo!", anggota Laboratorium Sistem Terdistribusi, serta para penghuni Sekretariat HMIF ITB Lantai 4 yang telah memberikan kenangan berharga, motivasi, hiburan, serta bantuan untuk segala situasi.
  \item Aditya Inas Hamidah yang hadir menemani di sepertiga akhir keberjalanan tugas akhir ini.
  \item Teman-teman SUDO 2021 yang telah menemani, memberikan inspirasi, serta dukungan moral kepada penulis dalam menempuh kuliah pada program studi Teknik Informatika.
  \item Google Gemini 2.5 Pro yang telah menjadi rekan penulis dalam proses implementasi, pengujian, hingga penulisan tugas akhir ini.
  \item Kafe AyamAyaman, Bosscha, Kisah Manis, Kopitera, Jabarano, KOZI Dipatiukur, dan Nuesara yang telah menjadi tempat yang nyaman untuk mengerjakan tugas akhir.
  \item Klub Manchester United yang sudah menemani akhir pekan, menjadi sumber energi dan motivasi, dan menyadarkan saya untuk percaya akan proses.
  \item Airani Iofifteen dari Hololive Indonesia yang telah menemani penulisan laporan ini dengan seri siaran "Rebo Nyunda".
  \item Seluruh pihak lain yang tidak bisa disebutkan disini yang telah membantu dalam proses pengerjaan tugas akhir.
\end{enumerate}

Akhir kata, penulis mengucapkan terima kasih kepada semua pihak yang telah terlibat dalam pengerjaan tugas akhir ini. Penulis juga ingin menyampaikan mohon maaf apabila terdapat kesalahan maupun kekurangan dalam laporan tugas akhir ini. Penulis berharap semoga tugas akhir ini dapat bermanfaat bagi pembaca dan riset-riset kedepannya.

\begin{flushright}
  \vspace{0.5cm}
  Bandung, \tanggalpengesahan


  \vspace{1.5cm}

  Akbar Maulana Ridho
\end{flushright}