\subsubsection{Fungsionalitas pada Domain \textit{Groups}}

Pengujian dengan ID P16 dilakukan dengan skenario \textit{user} ingin membuat \textit{group} baru dengan nama "group-baru" pada sistem. Langkah langkah yang dilakukan:
\begin{enumerate}
  \item Mengunjungi halaman /groups
  \item Menekan tombol "Add Group" pada bagian kanan bawah
  \item Mengisi \textit{field name} dengan nilai "group-baru"
  \item Menekan tombol "submit"
\end{enumerate}

Setelah pengujian dilakukan, tabel yang awalnya kosong terisi dengan deskripsi \textit{group} yang telah dibuat. Selain itu, terdapat modal pada bagian bawah kanan yang menunjukan bahwa pembuatan \textit{group} telah berhasil. Hasil dari pengujian dapat dilihat pada lampiran \ref{fig:pengujian-p16}

Pengujian dengan ID P17 dilakukan dengan skenario \textit{user} ingin membuat \textit{group} baru dengan nama yang duplikat pada sistem. Langkah langkah yang dilakukan:
\begin{enumerate}
  \item Mengunjungi halaman /groups
  \item Menekan tombol "Add Group" pada bagian kanan bawah
  \item Mengisi \textit{field name} dengan nilai "group-baru"
  \item Menekan tombol "submit"
\end{enumerate}

Setelah pengujian dilakukan, muncul modal pada bagian bawah kanan yang menunjukan bahwa pembuatan \textit{group} gagal. Pesan yang tidunjukan yaitu "group-baru is exists please another name". Hasil dari pengujian dapat dilihat pada lampiran \ref{fig:pengujian-p17}

Pengujian dengan ID P18 dilakukan dengan skenario \textit{user} ingin melihat daftar \textit{group} yang tersedia pada sistem. Pengujian ini dilakukan dengan cara mengunjungi halaman /groups. Dapat dilihat dari pengujian P17 bahwa tabel \textit{groups} terisi dengan nilai yang sudah dibuat sebelumnya. Hal ini menunjukan bawha \textit{user} dapat melihat daftar \textit{groups} pada sistem.

Pengujian dengan ID P19 dilakukan dengan skenario \textit{user} ingin menghapus salah satu \textit{group} yang dimiliki. Langkah langkah yang dilakukan:
\begin{enumerate}
  \item Mengunjungi halaman /groups
  \item Menekan tombol "titik tiga (\textit{elipsis horizontal})" yang berada pada ujung tabel
  \item Menekan tombol "delete" dari popup modal
\end{enumerate}

Setelah melakukan langkah langkah tersebut dapat dilhat bahwa tidak ada data pada tabel. Hal ini menunjukan bahwa \textit{device} berhasil dihapus dari sistem. Hasil pengujian dapat dilihat pada lampiran \ref{fig:pengujian-p19}.