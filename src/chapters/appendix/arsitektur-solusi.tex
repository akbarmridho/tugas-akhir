\chapter{Arsitektur Solusi}

Setiap arsitektur solusi memiliki dua layanan eksternal, yaitu layanan pengguna dan layanan gerbang pembayaran. Kedua layanan ini akan dibuat \textit{stateless} sehingga penskalaan dapat dilakukan dengan menambah jumlah \textit{instance} layanan. Pada kasus ini, kedua layanan ini akan diusahakan sehingga tidak menjadi sumber \textit{bottleneck}. Layanan pengguna tidak akan menjadi \textit{bottleneck} karena pada saat pengujian akan diasumsikan seluruh pengguna sudah \textit{login} dan pemeriksaan otentikasi dilakukan pada masing-masing \textit{instance backend}.

Komunikasi dengan layanan pembayaran harus dilakukan secara sinkron, setidaknya ketika layanan tiket memanggil layanan pembayaran untuk membuat tagihan. Apabila layanan pembayaran tiket mengalami \textit{bottleneck} pada basis data, pengoptimalan pada basis data akan dilakukan dengan menggunakan kluster Citus sebagaimana yang dilakukan pada arsitektur yang mengoptimalkan PostgreSQL.

\section{Arsitektur Dasar Acuan}

\begin{figure}[ht]
    \centering
    \includegraphics[width=0.8\textwidth]{resources/chapter-3/architecture-reference.png}
    \caption{Arsitektur Dasar Acuan}
    \label{fig:baseline-architecture}
\end{figure}

Arsitektur ini akan menjadi dasar acuan yang digunakan sebagai dasar perbandingan kinerja. Komponen \textit{backend} utama (layanan tiket) bersifat \textit{stateless}, sehingga dapat di-\textit{scale} dengan meningkatkan jumlah \textit{instance}. Kemudian, gerbang API akan melakukan \textit{load balancing} untuk mendistribusikan beban ke beberapa \textit{instance}.

Basis data merupakan komponen yang sulit di-\textit{scale} secara dinamis berdasarkan beban yang diterima. Biasanya, penskalaan secara vertikal merupakan opsi utama untuk meningkatkan \textit{throughput}, terutama dalam operasi tulis. Meskipun begitu, pada arsitektur ini terdapat kluster PostgreSQL dengan konfigurasi satu node pemimpin dan sisanya node replika. Keberadaan replika memungkinkan peningkatan \textit{throughput} permintaan baca, meski tidak ada peningkatan \textit{throughput} untuk operasi tulis.

Teknik lain yang dapat digunakan adalah penggunaan \textit{micro-batching}. Misalkan kueri pembacaan ketersediaan dilakukan setiap 100 milidetik dan permintaan yang masuk dalam rentang waktu tersebut akan mendapatkan hasil kueri pada \textit{batch} tersebut. Meskipun begitu, pendekatan ini tidak diuji karena perbandingannya tidak setara.

\section{Arsitektur yang Mengoptimalkan PostgreSQL}

\begin{figure}[ht]
    \centering
    \includegraphics[width=0.8\textwidth]{resources/chapter-3/architecture-optimized.png}
    \caption{Arsitektur yang Mengoptimalkan PostgreSQL}
    \label{fig:optimized-architecture}
\end{figure}

Arsitektur ini mengoptimalkan sistem dengan pola CQRS. Tanggung jawab permintaan baca dilimpahkan kepada RisingWave. \textit{Streaming database} ini mengonsumsi \textit{CDC stream} dari kluster PostgreSQL, lalu memperbarui kueri secara inkremental. Hal yang perlu diperhatikan dalam penggunaan RisingWave adalah \textit{replication lag}. Data yang dikembalikan oleh RisingWave tidak valid apabila data tersebut \textit{outdated}. Penggunaan ekstensi Citus memungkinkan pembagian data berdasarkan baris dan \textit{multiple writer}. Selain itu, perintah pemesanan tiket (berupa \textit{command}/ \textit{event sourcing}) akan dimasukkan ke dalam antrean terlebih dahulu, lalu diproses secara bertahap. Redis digunakan untuk menyimpan \textit{uncommited data} dan menolak permintaan pemesanan lebih awal.

Penggunaan ekstensi Citus memungkinkan peningkatan \textit{write throughput} tidak hanya dengan pendekatan \textit{scale up}, tetapi juga dengan pendekatan \textit{scale-out}. Redpanda dapat dibuat kluster dengan pemartisian data untuk meningkatkan \textit{throughput}.

\textit{Persistence} pada Redis bersifat asinkron, sehingga terdapat kemungkinan data hilang ketika terjadi kegagalan. Meskipun begitu, penggunaan \textit{key-value store} lain yang \textit{persistent} berpotensi memperlambat kinerja. Dalam kasus ini, Redis akan dikonfigurasikan dalam mode kluster untuk redundansi dan mode AOF untuk \textit{persistence}. Hilangnya data hanya akan terjadi ketika \textit{master} dan \textit{replica} mengalami kegagalan dalam satu waktu. Selain itu, hilangnya data pada Redis tidak akan mengganggu integritas data karena pemeriksaan kedua masih akan dilakukan saat pemrosesan data.

\section{Arsitektur \textit{Event-Driven}}

\begin{figure}[ht]
    \centering
    \includegraphics[width=0.8\textwidth]{resources/chapter-3/architecture-event-driven.png}
    \caption{Arsitektur \textit{Event-Driven}}
    \label{fig:solution-event-driven-architecture}
\end{figure}

Arsitektur ini tidak menggunakan PostgreSQL sama sekali. Pada dasarnya, basis data relasional terdiri atas komponen \textit{storage} dan \textit{query processor}. Pada arsitektur ini, komponen \textit{storage} diganti menggunakan Redpanda dengan berbagai topik dan \textit{query processor} diganti dengan RisingWave. Meskipun begitu, pendekatan ini tidak memiliki dukungan \textit{transaction} selain \textit{transaction} pada Redpanda yang berupa \textit{push log all or nothing} pada beberapa topik sekaligus. Untuk itu, Redis digunakan untuk menyimpan \textit{dirty data} atau \textit{uncommited data} sehingga untuk mencegah \textit{double booking}.

Redpanda dapat dibuat kluster dengan pemartisian data untuk meningkatkan \textit{throughput}. Selain itu, RisingWave merupakan \textit{streaming database} yang \textit{cloud-native} sehingga dapat di-\textit{scale out} dengan mudah untuk meningkatkan \textit{throughput}.

Isu \textit{persistence} Redis pada arsitektur ini lebih penting daripada arsitektur sebelumnya. Meskipun begitu, penggunaan kluster dan mode AOF masih dianggap cukup dengan konfigurasi tambahan. Untuk menjamin \textit{stronger durability}, Redis akan dikonfigurasikan untuk selalu melakukan penulisan langsung setelah perintah dijalankan (\texttt{appendfsync always}). Konfigurasi ini akan melakukan penulisan langsung setelah perintah dijalankan. Berbeda dengan konfigurasi \textit{default} Redis yang baru melakukan operasi tulis setiap detik. Kluster Redis masih digunakan untuk \textit{sharding}, tetapi replika tidak akan digunakan untuk menghindari \textit{stale read} karena replikasi pada Redis bersifat asinkron.