\section{Kebutuhan Fungsional dan Non-Fungsional Sistem Tiket}

ID mengikuti format \((T|P|U)(F|N)-XX\). Huruf pertama menunjukkan kebutuhan fungsional atau non-fungsional untuk layanan tersebut. T berarti tiket, P berarti pembayaran, dan U berarti pengguna. Huruf kedua menunjukkan kebutuhan fungsional atau non-fungsional. F berarti fungsional dan N berarti non-fungsional. Contoh: TF-01 berarti kebutuhan fungsional nomor 01 untuk layanan tiket.

\subsection{Kebutuhan Fungsional}

Berikut adalah kebutuhan fungsional dan non-fungsional sistem tiket yang menjadi cakupan pada tugas akhir ini:

\begingroup
\footnotesize
\begin{longtable}{|l|p{0.4\textwidth}|p{0.4\textwidth}|}
    \caption{Kebutuhan Fungsional Sistem Tiket}                                                                                                                                                                                                                                                                                                                                                                                                                                          \\
    \hline
    \textbf{ID} & \textbf{Kebutuhan}                                                                                             & \textbf{Deskripsi}                                                                                                                                                                                                                                                                                                                                    \\
    \hline
    \endfirsthead

    \multicolumn{3}{|l|}{\tablename\ \thetable\ -- \textit{Lanjutan dari halaman sebelumnya}}                                                                                                                                                                                                                                                                                                                                                                                            \\
    \hline
    \textbf{ID} & \textbf{Kebutuhan}                                                                                             & \textbf{Deskripsi}                                                                                                                                                                                                                                                                                                                                    \\
    \hline
    \endhead

    \hline
    \multicolumn{3}{|r|}{\textit{Dilanjutkan ke halaman berikutnya}}                                                                                                                                                                                                                                                                                                                                                                                                                     \\
    \endfoot

    \hline
    \endlastfoot

    \hline
    TF-01       & Sistem dapat melayani permintaan ketersediaan acara.                                                           &                                                                                                                                                                                                                                                                                                                                                       \\
    \hline
    \hline
    TF-02       & Sistem dapat melayani permintaan ketersediaan tiket untuk suatu acara .                                        & Ketersediaan tiket dibagi berdasarkan kategori. Data ketersediaan bisa lebih granular (menampilkan ketersediaan per kursi) atau hanya menampilkan jumlah ketersediaan untuk suatu kategori. Perilaku ini dapat diatur per acara atau kategori tiket.                                                                                                  \\
    \hline
    \hline
    TF-03       & Sistem dapat melayani permintaan pemesanan tiket untuk suatu acara.                                            & Seorang pengguna dapat memesan hingga empat tiket dalam kategori yang berbeda sekaligus. Pengguna dapat memesan berdasarkan kursi atau kategori tiket, bergantung pada pengaturan acara atau kategori tiket. Pemesanan akan dibatalkan secara otomatis ketika sudah melewati tenggat waktu pembayaran.                                                \\
    \hline
    \hline
    TF-04       & Pengguna dapat melihat status tiket yang pernah dipesan.                                                       &                                                                                                                                                                                                                                                                                                                                                       \\
    \hline
    TF-05       & Sistem dapat menangani penjualan tiket untuk lebih dari satu acara dalam satu waktu.                           &                                                                                                                                                                                                                                                                                                                                                       \\
    \hline
    \hline
    UF-01       & Sistem dapat melayani registrasi pengguna.                                                                     &                                                                                                                                                                                                                                                                                                                                                       \\
    \hline
    \hline
    UF-02       & Sistem menyediakan mekanisme \textit{login} bagi pengguna.                                                     &                                                                                                                                                                                                                                                                                                                                                       \\
    \hline
    \hline
    UF-03       & Sistem menyediakan \textit{endpoint} bagi layanan lain untuk memperoleh informasi pengguna yang terotentikasi. &                                                                                                                                                                                                                                                                                                                                                       \\
    \hline
    \hline
    PF-01       & Sistem dapat membuat tagihan pembayaran dan pranala pembayaran.                                                & \textit{Mock service}. Pembayaran dilakukan dengan mengirimkan \textit{request} kepada pranala yang diberikan dengan parameter sukses atau gagal. Selain itu, terdapat tenggat waktu pembayaran yang ditentukan berdasarkan parameter pada saat pembuatan tagihan. Pembayaran akan otomatis gagal apabila tidak dipenuhi hingga batas waktu tersebut. \\
    \hline
    \hline
    PF-02       & Sistem memanggil \textit{webhook} yang telah ditentukan ketika terdapat pembayaran yang berhasil atau gagal.   & \textit{Mock service}. Pembayaran dilakukan dengan mengirimkan \textit{request} kepada pranala yang diberikan dengan parameter sukses atau gagal.                                                                                                                                                                                                     \\
    \hline
    \hline
    PF-03       & Sistem menyediakan \textit{endpoint} untuk menampilkan detail tagihan pembayaran.                              &                                                                                                                                                                                                                                                                                                                                                       \\
    \hline
\end{longtable}
\endgroup

Pemesanan tiket dibagi menjadi dua tahap, yaitu fase \textit{hold seating} dan fase pembayaran. Fase \textit{hold seating} akan me-\textit{reserve} tiket sampai batas waktu tertentu hingga pengguna menyelesaikan fase pembayaran. Saat pembayaran berhasil, pembelian tiket baru akan dianggap sukses. Daftar acara dan ketersediaan awal tiket merupakan data yang diisi dari awal, sehingga fitur manajemen acara dan tiket tidak diimplementasikan.

\subsection{Kebutuhan Non-Fungsional}

Berikut adalah kebutuhan non-fungsional yang menjadi cakupan pada tugas akhir ini:

\begingroup
\footnotesize
\begin{longtable}{|l|p{0.4\textwidth}|p{0.4\textwidth}|}
    \caption{Kebutuhan Non-Fungsional Sistem Tiket}                                                                                                                                                               \\
    \hline
    \textbf{ID} & \textbf{Parameter}      & \textbf{Kebutuhan}                                                                                                                                                    \\
    \hline
    \endfirsthead

    \multicolumn{3}{|l|}{\tablename\ \thetable\ -- \textit{Lanjutan dari halaman sebelumnya}}                                                                                                                     \\
    \hline
    \textbf{ID} & \textbf{Parameter}      & \textbf{Kebutuhan}                                                                                                                                                    \\
    \hline
    \endhead

    \hline
    \multicolumn{3}{|r|}{\textit{Dilanjutkan ke halaman berikutnya}}                                                                                                                                              \\
    \endfoot

    \hline
    \endlastfoot

    \hline
    TN-01       & Konsistensi             & Sistem harus memastikan tidak terjadi \textit{double booking} pada saat pemesanan tiket.                                                                              \\
    \hline
    \hline
    TN-02       & \textit{Data Freshness} & Sistem harus selalu mengembalikan data paling terbaru.                                                                                                                \\
    \hline
    \hline
    UN-01       & Otentisitas Pengguna    & Layanan lain dapat memverifikasi otentikasi token pengguna tanpa harus memanggil layanan pengguna, seperti dengan penggunaan token JWT dengan \textit{shared secret}. \\
    \hline
\end{longtable}
\endgroup