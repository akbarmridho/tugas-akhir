\section{Lingkungan Pengujian}

Penelitian ini rencananya akan diuji pada \textit{cloud provider} dengan \textit{dedicated hardware}. Penelitian ini akan menggunakan dua lingkungan yang berbeda, yaitu:

\begin{enumerate}
    \item \textit{Virtual machine} untuk menjalankan basis data PostgreSQL. Basis data relasional akan lebih optimal apabila dijalankan pada lingkungan sendiri, bukan pada kontainer. Meskipun begitu, perlu mekanisme khusus agar \textit{redeployment} dan konfigurasi dapat diotomasi untuk memudahkan proses pengujian.
    \item \textit{Kubernetes} untuk melakukan orkestrasi kontainer layanan lainnya.
    \item \textit{Virtual machine} untuk menjalankan agen yang bertindak sebagai pengguna ketika proses pengujian berlangsung.
\end{enumerate}

Mengingat penelitian ini fokus pada peningkatan \textit{throughput}, setiap node yang dijalankan akan berada pada satu \textit{data center}/ \textit{availability zone} yang sama agar latensi dapat dikurangi. Selain itu, setiap node akan menggunakan server dengan media penyimpanan bertipe SSD NVME untuk memaksimalkan \textit{throughput} perangkat keras.