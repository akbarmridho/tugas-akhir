\section{Penggunaan Redpanda dan Alternatifnya}

Arsitektur perpesanan bisa dibagi menjadi dua, yaitu \textit{brokered} dan \textit{brokerless}. Arsitektur \textit{brokerless} menawarkan latensi yang lebih rendah dan \textit{deployment} yang lebih sederhana, tetapi tidak memiliki fitur seperti \textit{message durability}, \textit{partitioning}, dan \textit{replayability}. Oleh karena itu, ide solusi dengan antrean dan \textit{storage} untuk arsitektur \textit{event-driven} akan menggunakan arsitektur \textit{brokered}.

Model \textit{message broker} yang saling berbeda setidaknya saat ini dapat direpresentasikan oleh Apache Kafka, RabbitMQ, dan NATS. Berikut adalah perbandingan umum dari ketiga \textit{broker} tersebut \parencite{arshadChoosingTheRightMessaging,royNatsRmqKafka,studyOnModeryMessaging}:

\begingroup
\footnotesize
\begin{longtable}{|p{0.14\textwidth}|p{0.24\textwidth}|p{0.24\textwidth}|p{0.24\textwidth}|}
    \caption{Perbandingan Antara Kafka, RabbitMQ, dan NATS}                                                                                                                                                                    \\
    \hline
    \textbf{Aspek}               & \textbf{Kafka}                                                       & \textbf{RabbitMQ}                             & \textbf{NATS}                                                        \\
    \hline
    \endfirsthead

    \multicolumn{4}{|c|}{\tablename\ \thetable\ -- \textit{Lanjutan dari halaman sebelumnya}}                                                                                                                                  \\
    \hline
    \textbf{Aspek}               & \textbf{Kafka}                                                       & \textbf{RabbitMQ}                             & \textbf{NATS}                                                        \\
    \hline
    \endhead

    \hline
    \multicolumn{4}{|r|}{\textit{Dilanjutkan ke halaman berikutnya}}                                                                                                                                                           \\
    \endfoot

    \hline
    \endlastfoot

    \hline
    Ditulis dalam                & Scala, Java                                                          & Erlang                                        & Go                                                                   \\
    \hline
    Model Penyimpanan            & \textit{Log-based}                                                   & \textit{Queue-based}                          & \textit{Log-based}, \textit{Transient} (\textit{in-memory})          \\
    \hline
    \textit{Message Persistence} & \textit{Persistent}                                                  & \textit{Persistent} dan \textit{ephemeral}    & \textit{Persistent} dan \textit{ephemeral}                           \\
    \hline
    \textit{Throughput}          & Hingga 2 juta pesan per detik                                        & Hingga 60 ribu pesan per detik                & Hingga 6 juta pesan per detik                                        \\
    \hline
    Latensi                      & \textit{Low ms}                                                      & \textit{Low ms}                               & \textit{Sub-ms}                                                      \\
    \hline
    Model penskalaan             & Horizontal (dengan partisi)                                          & Terbatas                                      & Horizontal (mode kluster)                                            \\
    \hline
    Model Topik                  & Topik terpartisi                                                     & \textit{Queues}                               & \textit{Subject-based Topics}                                        \\
    \hline
    Model Konsumen               & \textit{Pull-based}                                                  & \textit{Push-based}                           & \textit{Pull or push-based}                                          \\
    \hline
    Protokol yang Didukung       & Kafka                                                                & AMQP, MQTT, STOMP                             & NATS, MQTT                                                           \\
    \hline
    \textit{Ordering Guarantee}  & Level partisi                                                        & Level \textit{queue}                          & Per subjek                                                           \\
    \hline
    \textit{Replayability}       & Ya                                                                   & Tidak                                         & Ya                                                                   \\
    \hline
    Penghapusan Pesan            & Tidak. Disimpan berdasarkan \textit{retention policy}                & Ya. Dihapus setelah pemrosesan                & Dapat diatur                                                         \\
    \hline
    \textit{Delivery Guarantee}  & \textit{At most once}, \textit{at least once}, \textit{exactly once} & \textit{At most once}, \textit{at least once} & \textit{At most once}, \textit{at least once}, \textit{exactly once} \\
    \hline
\end{longtable}
\endgroup

Solusi \textit{queue} dan solusi \textit{event-driven} membutuhkan \textit{messaging platform} yang memiliki semantik \textit{exactly once} dengan \textit{throughput} tinggi. Selain itu, solusi \textit{event-driven} juga membutuhkan \textit{persistence} dan \textit{durability}. Pilihan yang tersedia adalah Apache Kafka dan NATS. RabbitMQ tidak dipilih karena memiliki masalah pada \textit{throughput} dan tidak menawarkan \textit{durability} dan \textit{replayability}. Pada akhirnya, Apache Kafka merupakan \textit{platform} yang dipilih. Meskipun NATS juga mampu memenuhi kebutuhan, fitur-fitur tersebut merupakan fitur tambahan yang bukan menjadi tujuan awal dari pembuatan NATS dan tergolong masih baru.

Meskipun begitu, Apache Kafka memiliki masalah pada efisiensi sumber daya dan sulit dikelola. Di sisi lain, terdapat alternatif \textit{messaging platform} yang Kafka-\textit{compatible} seperti Redpanda. Redpanda memiliki latensi yang lebih rendah, pengoptimalan yang lebih baik, dan lebih mudah untuk dikelola dibandingkan dengan Kafka \parencite{comparingKafkaAlternatives}. Oleh karena itu, Redpanda menjadi pilihan yang lebih layak dibandingkan dengan kafka.