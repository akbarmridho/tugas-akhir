\section{Analisis Masalah}

Seiring dengan berjalannya waktu, bukan hal yang tidak mungkin akan ada lagi penjualan tiket dengan tingkat peminat serupa dengan penjualan tur Taylor Swift pada \textit{platform} Ticketmaster dan tur Coldplay pada \textit{platform} BookMyShow. Meskipun begitu, belum semua \textit{platform} terbiasa menangani beban pada skala ini, entah karena keterbatasan sumber daya, keterbatasan arsitektur sistem, atau pun hal lainnya. Untuk menangani beban, kebanyakan \textit{platform} menggunakan solusi antrean yang memaksa pengguna untuk menunggu selama puluhan menit hingga berjam-jam agar bisa mengakses situs \textit{ticketing}. Di lain sisi, masih sedikit studi yang membahas secara khusus untuk meningkatkan skalabilitas sistem dengan kasus seperti ini. \cite{microservicesEventDriven} membahas desain arsitektur yang \textit{fault-tolerant} dan tidak membahas aspek skalabilitas. \cite{backendForTicketing} juga tidak membahas arsitektur sistem dari sisi skalabilitas. Berdasarkan hal tersebut, diperlukan desain arsitektur yang optimal dan mampu menangani beban seperti ini.

Beban pada kasus \textit{ticketing} dapat dibagi menjadi dua: \textit{query} ketersediaan tiket dan proses \textit{booking} tiket.

Sebagaimana digambarkan pada figur \ref{fig:event-rm} dan \ref{fig:ticket-storage}, sebuah \textit{query} ketersediaan harus melakukan \textit{query} terhadap beberapa entitas yang saling berhubungan dan melakukan beberapa agregasi. \textit{Query} ini akan berat terlebih lagi apabila permintaan atas \textit{query} ini sangat tinggi, seperti yang terjadi saat banyak pengguna mencoba membeli tiket ini dalam satu waktu. Selain itu, \textit{query} ini tidak dapat di-\textit{cache} karena data selalu berubah pada saat beban tinggi sehingga \textit{query} akan selalu \textit{stale}.

Selain itu, akan ada banyak pengguna yang mencoba untuk melakukan operasi \textit{insert} dan \textit{update} secara bersamaan pada satu region data yang sama sehingga banyak terjadi \textit{write contention}. Skema \textit{locking} dan \textit{flow control} diperlukan agar percobaan pembelian tiket yang sama dapat berkurang dan \textit{throughput} penulisan pada basis data perlu ditingkatkan. Sayangnya, \textit{scaling} pada basis data biasanya jarang mendukung \textit{multiple writer}.