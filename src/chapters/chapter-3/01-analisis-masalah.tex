\section{Analisis Masalah}

Penjualan tiket acara dengan tingkat peminat serupa dengan penjualan tiket Taylor Swift atau Coldplay tentu akan terjadi lagi di masa mendatang. Meskipun begitu, tidak semua penyedia layanan terbiasa menangani beban pada skala ini karena berbagai hal. Kebanyakan penyedia layanan menggunakan solusi antrean virtual sehingga jumlah pengguna yang mengakses situs secara bersamaan dapat dibatasi. Pendekatan ini memang membantu meringankan beban sistem sehingga stabilitas tetap terjaga. Meskipun begitu, akan ada banyak pengguna yang harus menunggu lama hanya agar bisa mengakses platform.

Di sisi lain, belum banyak studi yang membahas skalabilitas sistem dengan kasus seperti ini. \cite{microservicesEventDriven} membahas desain arsitektur yang \textit{fault-tolerant} dan tidak membahas aspek skalabilitas. \cite{backendForTicketing} juga tidak membahas arsitektur sistem dari sisi skalabilitas. \cite{barua2024enhancingresiliencescalabilitytravel} membahas desain sistem pemesanan tiket pesawat dengan arsitektur \textit{microservice}. Pengoptimalan \textit{fault tolerance} dan \textit{load balancing} memungkinkan penggunaan sumber daya yang lebih optimal, sehingga \textit{throughput} meningkat. Meskipun begitu, penelitian tersebut tidak membahas pengoptimalan pada pola akses basis data. Berdasarkan hal tersebut, diperlukan desain arsitektur yang optimal dan mampu menangani beban seperti ini. Tujuan dari penelitian ini tidak serta merta ingin mengganti solusi antrean virtual. Dengan adanya arsitektur yang optimal, banyaknya pengguna yang bisa dilayani dalam satu waktu dapat meningkat, sehingga waktu tunggu dan proses penjualan tiket dapat selesai lebih cepat tanpa kendala.

Beban sistem tiket dapat dibagi menjadi dua: permintaan baca ketersediaan tiket dan proses pemesanan tiket. Sebagaimana digambarkan pada figur \ref{fig:event-rm} dan \ref{fig:ticket-storage}, kueri ketersediaan harus menagregasi beberapa entitas yang saling berhubungan. Operasi ini akan sangat berat apabila dilakukan dengan jumlah yang sangat banyak, seperti saat terdapat banyak pengguna yang ingin membeli tiket dalam waktu yang bersamaan. Hasil kueri ini tidak dapat di-\textit{cache} karena data selalu berubah saat penjualan sehingga data akan langsung \textit{stale}. Selain itu, akan ada banyak pengguna yang ingin memesan tiket secara bersamaan, sehingga terjadi \textit{write contention}. Skema \textit{locking} dan \textit{flow control} diperlukan agar percobaan pembelian tiket yang sama dapat berkurang Selain itu, \textit{throughput} pemrosesan pesanan juga perlu ditingkatkan. Sayangnya, penskalaan pada basis data biasanya jarang mendukung \textit{multiple writer}.