\section{Analisis Masalah}

Penjualan tiket acara dengan tingkat peminat serupa dengan penjualan tiket Taylor Swift dan Coldplay tentu akan terjadi lagi di masa mendatang. Meskipun begitu, tidak semua penyedia layanan terbiasa menangani beban pada skala ini. Kebanyakan penyedia layanan menggunakan solusi antrean virtual yang membatasi jumlah pengguna yang mengakses situs secara bersamaan. Pendekatan ini memang membantu meringankan beban sistem dan menjaga stabilitas. Meskipun begitu, banyak pengguna yang harus menunggu lama untuk bisa mengakses platform.

Di sisi lain, belum banyak studi yang membahas skalabilitas sistem dengan kasus seperti ini. \cite{microservicesEventDriven} membahas desain arsitektur yang \textit{fault-tolerant} dan tidak membahas aspek skalabilitas. \cite{backendForTicketing} juga tidak membahas arsitektur sistem dari sisi skalabilitas. \cite{barua2024enhancingresiliencescalabilitytravel} membahas desain sistem pemesanan tiket pesawat dengan arsitektur \textit{microservice}. Pengoptimalan \textit{fault tolerance} dan \textit{load balancing} memungkinkan penggunaan sumber daya yang lebih optimal, sehingga \textit{throughput} meningkat. Meskipun begitu, penelitian tersebut tidak membahas pengoptimalan pada pola akses basis data. Berdasarkan pertimbangan di atas, diperlukan desain arsitektur yang optimal dan mampu menangani beban seperti ini. Solusi ini tidak serta merta mengganti solusi antrean virtual. Dengan adanya arsitektur yang optimal, jumlah pengguna yang bisa dilayani dalam satu waktu dapat meningkat dan proses penjualan menjadi lebih cepat tanpa kendala.

Beban sistem tiket dapat dibagi menjadi dua: permintaan baca ketersediaan tiket dan proses pemesanan tiket. Sebagaimana digambarkan pada figur \ref{fig:event-rm} dan \ref{fig:ticket-storage}, kueri ketersediaan harus mengagregasi beberapa entitas yang saling berhubungan. Operasi ini berat apabila dilakukan dengan jumlah yang sangat banyak, seperti saat banyak pengguna ingin membeli tiket dalam waktu yang bersamaan. Hasil kueri ini tidak dapat di-\textit{cache} karena data selalu berubah saat penjualan sehingga data akan langsung \textit{stale}. Selain itu, akan ada banyak pengguna yang ingin memesan tiket secara bersamaan, sehingga terjadi \textit{write contention}. Skema \textit{locking} dan \textit{flow control} diperlukan agar percobaan pembelian tiket yang sama dapat berkurang Selain itu, \textit{throughput} pemrosesan pesanan juga perlu ditingkatkan.