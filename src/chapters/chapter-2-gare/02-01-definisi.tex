\textit{Service mesh} adalah sebuah infrastruktur yang memanage \textit{Service} komunikasi antar \textit{service}. \textit{Service mesh} dibuat dengan tujuan untuk memberikan layer tambahan untuk architecture yang dibuat tanpa harus membuat kode aplikasi pada setiap \textit{service} untuk berkomunikasi.\parencite{li2019}

Jika terdapat dua buah \textit{service}, kedua \textit{service} tersebut harus membuat sebuah interface ataupun cara menghubungkan masing masing \textit{service} untuk berkomunikasi karena perlu adanya beberapa penyesuaian, seperti penyesuaian bahasa pemrograman karena mungkin saja kedua \textit{service} tersebut menggunakan bahasa pemrograman yang berbeda.

Setiap \textit{service} juga perlu menyesuaikan cara dari setiap \textit{service} tersebut menerima dan mengirim request. Beberapa \textit{service} mungkin saja menggunakan gRPC untuk menerima dan mengirim request dan \textit{service} lain menggunakan REST ataupun graphQL. Bayangkan jika kita memiliki lebih dari 100 \textit{service} yang ingin ber orkestrasi dan berkomunikasi satu sama lain, kita harus membuat interface untuk setiap \textit{service} yang ada dan hal ini sangat memakan waktu.

\textit{Service mesh} hadir untuk menyelesaikan masalah ini terutama masalah komunikasi internal antar \textit{service}. Selain itu, \textit{Service mesh} juga memberikan fitur seperti \textit{device discovery}, \textit{central authentication}, \textit{access control}, \textit{load balancing}, \textit{logging}, serta \textit{monitoring}. Selain memberikan fitur tersebut, \textit{\textit{service} mesh} juga harus memiliki \textit{reliability} serta \textit{fault tolerance}. \parencite{li2019}

\textit{Service mesh} pada umumnya memiliki dua plane, \textit{control plane} dan \textit{data plane}. \textit{control plane} merupakan sebuah tempat terpusat untuk mengontrol jaringan mulai dari \textit{device discovery}, \textit{logging dan monitoring}, ataupun menjaga \textit{service level agreement} (SLA) agar \textit{availability} dari \textit{service} yang ada memiliki nilai yang baik. \textit{data plane} sering disebut sebagai \textit{forwarding plane} karena bertujuan untuk meneruskan ataupun menerima request dari \textit{service} yang dituju sesuai arahan dari \textit{control plane}. Kedua \textit{plane} ini menjadi komponen paling penting dalam pembuatan \textit{\textit{service} mesh}.

\textit{Service mesh} dapat melakukan beberapa jenis interaksi terhadap komponen kubernetes, interaksi ini merupakan keunggulan \textit{service mesh} untuk mendukung berbagai macam fungsionalitas dan desain arsitektur. Berikut merupakan tujuh jenis interaksi yang dapat dilakukan oleh \textit{service mesh} \parencite{ganguli2021}

\begin{enumerate}
  \item Komunikasi antar \textit{pod}
  \item Komunikasi \textit{pod} dengan \textit{service}
  \item Komunikasi \textit{ingress controller} dengan \textit{pod} dan sebaliknya
  \item Komunikasi \textit{load balancer} dengan \textit{pod} dan sebaliknya
  \item Komunikasi \textit{pod} dengan \textit{egress controller}
  \item \textit{API gateway to service and vice-versa}
  \item \textit{TLS termination}
\end{enumerate}

Sudah banyak produk ataupun aplikasi yang menawarkan solusi untuk menyelesaikan masalah \textit{\textit{service} Mesh}, diantaranya Istio, Linkerd, Airbnb Synapse, dan AWS App Mesh. Keempat aplikasi tersebut memiliki cara tersendiri agar \textit{\textit{service} mesh} dapat berjalan dengan baik. Berikut tabel perbandingan dari keempat produk tersebut

\begin{longtable}{|p{1.5cm}|p{1.5cm}|p{1.5cm}|p{1.5cm}|p{1.5cm}|p{1.5cm}|p{1.5cm}|}
  \caption{Perbandingan Platform Solusi \textit{service mesh} \parencite{li2019}} \label{tab:perbandingan-service-mesh}                                                                                                                                \\
  \hline
  \rowcolor{gray!30} \textbf{Aplikasi} & \textbf{\textit{Data plane}} & \textbf{\textit{Open source}} & \textbf{\textit{Activeness}} & \textbf{\textit{Major advantage}}    & \textbf{\textit{Critical limitation}} & \textbf{\textit{Rating overall}} \\
  \hline
  \endfirsthead

  \endhead

  % \textit{Intelligent Workload Factoring for a Hybrid Cloud Computing Model}, \parencite{zhang} & VM & \textit{Request Rate} & Reaktif & ARIMA \tabularnewline
  Istio                                & Envoy                        & Yes                           & Good                         & Growing Community and Fast Iteration & Lack of support                       & Moderate \tabularnewline \hline

  Linkerd2                             & Linkerd-proxy                & Yes                           & Good                         & Stability and CNCF Accepted          & Potential vendor Lock-in              & Good \tabularnewline \hline

  AWS App Mesh                         & Envoy                        & No                            & Good                         & Native Compatibility with AWS        & Closed Ecosystem                      & Preview \tabularnewline \hline

  Airbnb Synapse                       & HAProxy / Nginx              & Yes                           & Poor                         & N/A                                  & Limited Features                      & Poor \tabularnewline \hline

  \hline
\end{longtable}
