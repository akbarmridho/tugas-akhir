\section{\textit{Smart Home System}}

\textit{Smart home system} adalah integrasi dari tekonologi dan layanan melalui jaringan rumah untuk menciptakan kehidupan yang lebih \textit{seamless}. \textit{Smart home system} menggunakan berbagai macam teknologi untuk melakukan control dan monitor segala perangkat dan interaksinya yang berada pada lingkungan rumah. \textit{Smart home system} bertujuan untuk menciptakan lingkungan yang efisien, aman, serta memudahkan pekerjaan harian yang redundan sehingga hal seperti ini dapat diotomasi dengan mudah \parencite{kadam2015smart}.

IoT memainkan peran penting dalam membangun \textit{Smart Home System}. Dengan IoT, hampir setiap objek dalam kehidupan sehari-hari di rumah dapat terhubung ke Internet. IoT memungkinkan pemantauan dan pengendalian semua objek yang terhubung ini tanpa terbatas oleh waktu dan lokasi \parencite{IoTSmartHome}.

Dalam penerapan konsep \textit{smart home}, terdapat beberapa aspek penting yang dapat diterapkan untuk meningkatkan kualitas lingkungan di dalam rumah. Pertama, terdapat aspek kontrol lingkungan yang mencakup area lingkungan rumah seperti taman dan pekarangan. Dalam hal ini, \textit{smart home} dapat memberikan informasi penting seperti suhu udara, kelembapan, dan bahkan detektor banjir yang dapat memantau kondisi lingkungan dengan akurat. Selain itu, tirai elektrik dan perangkat serupa dapat dikendalikan secara otomatis, menjadikan pengaturan lingkungan yang nyaman lebih mudah \parencite{PerancanganPrototypeSmartHome}.

Kedua, terdapat aspek mengenai efisiensi energi. \textit{Smart Home} dapat membantu menghemat sumber daya energi seperti listrik, air, dan gas melalui otomatisasi penggunaan perangkat seperti lemari es, kompor, pencuci piring, dan mesin cuci. Pengguna dapat mengontrol perangkat ini secara otomatis atau dari jarak jauh, sehingga mengurangi konsumsi energi yang tidak perlu \parencite{PerancanganPrototypeSmartHome}.

Selanjutnya, dalam aspek keamanan, \textit{smart home} dapat memberikan perlindungan yang lebih baik baik di dalam maupun di luar rumah. Detektor gerak, sistem kunci pintu dan jendela, serta pengawasan garasi dapat diintegrasikan dalam \textit{smart home system}. Keamanan dapat diawasi dan dikendalikan baik secara otomatis maupun dari jarak jauh, meningkatkan tingkat keselamatan bagi penghuni \parencite{PerancanganPrototypeSmartHome}.

Dengan menerapkan konsep-konsep tersebut, \textit{smart home system} tidak hanya memberikan kenyamanan, tetapi juga efisiensi, dan keamanan. \textit{Smart home system} adalah langkah maju dalam menjadikan rumah menjadi tempat yang lebih pintar dan lebih baik untuk ditinggali \parencite{PerancanganPrototypeSmartHome}.