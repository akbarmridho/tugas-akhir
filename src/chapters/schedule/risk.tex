\section{Risiko Pelaksanaan}

Berikut adalah risiko yang diperkirakan dapat menghambat pelaksanaan tugas akhir:

\begin{enumerate}
    \item Karena tingkat kesulitan yang belum tentu dapat diestimasikan dengan baik, ketiga arsitektur belum tentu dapat diimplementasikan. Apabila terdapat banyak hambatan, hanya dua dari tiga arsitektur yang akan diimplementasikan, yaitu arsitektur acuan dan arsitektur yang mengoptimalkan PostgreSQL.
    \item Mekanisme pengujian dan uji kasus harus dirancang dengan baik agar dapat merepresentasikan beban di dunia nyata. Untuk menangani hal ini, mekanisme pengujian dan uji kasus harus menjadi aspek yang mendapat perhatian khusus. Dalam perancangan pengujian, sebaiknya mempertimbangkan berbagai referensi mekanisme \textit{benchmark} yang khusus dibuat untuk basis data transaksional atau sistem dengan kasus serupa.
    \item Karena keterbatasan biaya, jumlah sumber daya yang akan dialokasikan untuk pengujian tidak akan sepenuhnya sama dengan apa yang terjadi di \textit{production}. Untuk menangani hal ini, berbagai opsi lingkungan pengujian harus dipertimbangkan, seperti penyedia layanan yang lebih murah seperti Hetzner atau DigitalOcean. Lingkugnan pengujian tidak harus dilakukan pada layanan Google Cloud atau AWS.
    \item Perbedaan kinerja antar arsitektur dapat disebabkan karena pemilihan bahasa, \textit{hardware}, \textit{tools}, atau pun hal lain yang tidak diduga. Oleh karena itu, variasi lingkungan pengujian harus dibuat seminimal mungkin, sehingga perbedaan pada hasil pengujian benar-benar disebabkan karena perbedaan arsitektur. Untuk meminimalkan perbedaan lingkungan, setiap arsitektur harus diuji pada lingkungan yang sama dengan \textit{hardware dedicated}. Selain itu, pemilihan bahasa dan kakas harus sama untuk setiap arsitektur.
    \item Apabila skala lingkungan yang diuji merupakan miniatur dari lingkungan sesungguhnya, kinerja pada lingkungan sesungguhnya belum tentu \textit{scale} dengan cara yang sama sebagaimana yang terjadi pada lingkungan pengujian. Agar \textit{scale behavior} juga dapat diprediksi, pengujian sebaiknya dilakukan dengan jumlah beban dan jumlah sumber daya yang berbeda-beda.
\end{enumerate}