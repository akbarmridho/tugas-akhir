\section{Jadwal Pelaksanaan}

Berikut adalah rencana jadwal pengerjaan tugas akhir:

\begin{figure}[ht]
    \centering
    \includegraphics[width=1\textwidth]{resources/schedule/schedule.png}
    \caption{Jadwal Pelaksanaan}
    \label{fig:jadwal pelaksanaan}
\end{figure}

Secara umum, pengerjaan terbagi menjadi empat tahap, yaitu:

\begin{enumerate}
    \item Tahap perencanaan yang meliputi penentuan kebutuhan, perancangan implementasi, dan perancangan uji kasus.
    \item Tahap implementasi yang meliputi implementasi untuk tiga arsitektur, yaitu \textit{baseline architecture}, arsitektur pengoptimalan PostgreSQL, dan arsitektur \textit{event-driven}.
    \item Tahap \textit{deployment} dan pengujian untuk setiap arsitektur.
    \item Tahap penyelesaian yang meliputi pengolahan data dan menyelesaikan sisa pekerjaan laporan tugas akhir.
\end{enumerate}