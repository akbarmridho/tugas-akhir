\section{Metodologi}

Metodologi yang akan digunakan dalam pelaksanaan tugas akhir adalah metodologi Waterfall. Pendekatan ini dipilih karena masalah dan kebutuhan sistem sudah terdefinisikan dengan baik sebelum proses implementasi dilaksanakan. Meskipun begitu, pelaksanaan tugas akhir hanya akan sampai pada tahap pengujian dan tidak sampai tahap pemeliharaan.

Berikut adalah tahapan yang akan dilalui selama pelaksanaan tugas akhir:

\begin{enumerate}
      \item \textbf{\textit{Requirements}}

            Tahap ini akan menganalisis komponen apa saja yang diperlukan pada sistem tiket, lalu mengidentifikasi karakteristik beban dan \textit{bottleneck} yang mungkin terjadi pada sistem ini.

      \item \textbf{Analisis dan Perancangan Solusi}

            Setelah mengidentifikasi permasalahan, dilakukan analisis perancangan solusi yang bertujuan untuk mengoptimalkan operasi yang mengalami \textit{bottleneck} pada sistem tiket.

      \item \textbf{Implementasi}

            Setelah merancang solusi, gagasan tersebut akan dikembangkan dan diimplementasikan. Hasil dari tahap ini berupa desain dan implementasi yang mengoptimalkan sistem tiket.

      \item \textbf{Pengujian dan Evaluasi}

            Setelah implementasi berhasil dilakukan, dilakukan serangkaian pengujian untuk memastikan kebenaran implementasi dan peningkatan kinerja yang diperoleh dibandingkan dengan alternatif lainnya. Setelah pengujian dilakukan, hasil implementasi akan dievaluasi agar mendapatkan umpan balik terkait hal-hal yang dapat ditingkatkan ke depannya.

\end{enumerate}