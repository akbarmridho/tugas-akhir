\section{Metodologi}

Terdapat beberapa metodologi yang digunakan untuk melaksanakan tugas akhir ini, berikut adalah tahapan pelaksanaannya:
\begin{enumerate}
      \item \textbf{Identifikasi Permasalahan}

            Pada tahap ini, karakteristik operasi dan \textit{bottleneck} pada sistem \textit{ticketing} diidentifikasi permasalahannya.

      \item \textbf{Analisis dan Perancangan Solusi}

            Setelah mengidentifikasi permasalahan, dilakukan analisis perancangan solusi yang bertujuan untuk mengoptimalkan operasi yang mengalami \textit{bottleneck} pada sistem \textit{ticketing}. Analisis ini dimulai dari eksplorasi metode melalui studi literatur lalu dilanjutkan dengan penelitian yang pernah dilakukan.

      \item \textbf{Implementasi}

            Setelah merancang solusi, gagasan tersebut akan dikembangkan dan diimplementasikan. Hasil dari tahap ini berupa desain dan implementasi yang mengoptimalkan sistem \textit{ticketing}.

      \item \textbf{Pengujian dan Evaluasi}

            Setelah implementasi berhasil dilakukan, dilakukan serangkaian pengujian untuk memastikan kebenaran implementasi dan peningkatan kinerja yang diperoleh dibandingkan dengan alternatif lainnya. Setelah pengujian dilakukan, hasil implementasi akan dievaluasi agar mendapatkan \textit{feedback} terkait hal-hal yang dapat ditingkatkan kedepannya.

\end{enumerate}