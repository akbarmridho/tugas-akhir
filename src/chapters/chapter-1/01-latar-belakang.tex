\section{Latar Belakang}
\label{sec:latar-belakang}

Pada tahun 2022, Ticketmaster melayani penjualan dua juta tiket konser The Eras Tour Taylor Swift. Proses pelayanan ini tidak berjalan lancar karena sistem Ticket master mengalami \textit{crash} pada saat penjualan tiket \parencite{swiftTicketmaster}. Hal serupa terjadi di India pada saat BookMyShow melayani penjualan tiket Coldplay \parencite{coldplayBookMyShow}. Peristiwa ini cukup menarik apabila dilihat dari sisi teknis. Apakah ada cara untuk mengoptimalkan OLTP database seperti PostgreSQL agar dapat menangani jutaan pengguna pada satu waktu tanpa mengalami kegagalan sistem?

Skalabilitas sistem \textit{ticketing} memiliki karakterisitk yang unik, terutama pada kasus penjualan tiket dengan \textit{traffic} tinggi. \textit{Traffic} sistem ini \textit{bursty}, sehingga membutuhkan sistem yang elastis. Operasi \textit{read} harus selalu mengembalikan data terbaru padahal pada saat yang bersamaan terdapat banyak operasi \textit{write}. Penggunaan \textit{caching} tidak akan cocok karena data cepat \textit{stale}, sedangkan \textit{query} langsung ke basis data akan membebani sistem. Selain itu, akan ada banyak \textit{operasi} write pada \textit{region} data yang sama pada saat yang bersamaan.

Pendekatan untuk mengurangi beban pada basis data telah banyak dikembangkan, termasuk penggunaan pola \textit{command query responsibility segregation} (CQRS). Pola ini memisahkan tanggung jawab operasi \textit{read} dan \textit{write} pada bagian sistem berbeda yang terspesialisasi.

Untuk mengoptimalkan operasi \textit{read}, \textit{change data capture} (CDC) pada PostgreSQL dapat dimanfaatkan oleh \textit{streaming database} seperti Risingwave untuk membangun \textit{prebuilt query} secara \textit{incremental}. Dengan mengonsumsi \textit{event} perubahan data, \textit{streaming database} dapat melayani operasi \textit{read} dengan \textit{subsecond freshness} tanpa harus membebani OLTP database.

Untuk mengoptimalkan operasi \textit{write}, pemrosesan \textit{command} dapat diimplementasikan dengan \textit{queue} menggunakan \textit{event sourcing} dan \textit{event streaming platform} berlatensi rendah seperti Redpanda. Dengan pendekatan ini, \textit{write request} dapat diproses secara bertahap sesuai dengan kapasitas OLTP \textit{database}, sehingga \textit{burst request} dapat ditangani tanpa mengorbankan stabilitas sistem.

Penelitian ini berfokus pada pengoptimalan sistem \textit{ticketing} agar mampu melayani penjualan tiket dalam skala besar dengan menggunakan \textit{design pattern} CQRS. Operasi \textit{read} dioptimalkan dengan penggunaan \textit{streaming database} dan operasi \textit{write} dioptimalkan dengan penggunaan \textit{event sourcing} dan \textit{event streaming platform}.

