\section{Latar Belakang}
\label{sec:latar-belakang}

Pada tahun 2022, Ticketmaster melayani penjualan dua juta tiket konser The Eras Tour Taylor Swift. Proses pelayanan ini tidak berjalan lancar karena sistem Ticket master mengalami \textit{crash} pada saat penjualan tiket \parencite{swiftTicketmaster}. Hal serupa terjadi di India pada saat BookMyShow melayani penjualan tiket Coldplay \parencite{coldplayBookMyShow}. Peristiwa ini cukup menarik apabila dilihat dari sisi teknis. Apakah ada cara untuk mengoptimalkan OLTP database seperti PostgreSQL agar dapat menangani jutaan pengguna pada satu waktu tanpa mengalami kegagalan sistem? Terdapat dua kategori operasi yang harus dioptimalkan, yaitu operasi \textit{write} (termasuk \textit{update} dan \textit{delete}) dan operasi \textit{read}.

Untuk mengoptimalkan operasi \textit{read}, \textit{change data capture} (CDC) pada PostgreSQL dapat dimanfaatkan oleh \textit{streaming database} seperti Risingwave untuk membangun \textit{predefined query} secara \textit{incremental}. Dengan mengonsumsi \textit{event} perubahan data, \textit{streaming database} dapat melayani operasi \textit{read} dengan \textit{subsecond freshness} tanpa harus membebani OLTP database.

Untuk mengoptimalkan operasi \textit{write}, pendekatan \textit{event-based processing} dapat diterapkan sehingga \textit{write request} dapat di-\textit{queue} terlebih dahulu dan diproses berdasarkan kapasitas OLTP database. Untuk menangani penyimpanan dan distribusi \textit{event}, \textit{messaging platform} dengan latensi rendah seperti Redpanda dapat digunakan.

Penelitian ini berfokus pada pengoptimalan sistem yang mampu melayani \textit{concurrent read and write transaction} dalam skala besar dengan pendekatan \textit{event-based processing} dan \textit{stream processing}. Pengoptimalan ini akan diuji pada kasus \textit{ticketing system}.
