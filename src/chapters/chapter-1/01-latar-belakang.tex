\section{Latar Belakang}
\label{sec:latar-belakang}

Di era digital saat ini, digitalisasi pada berbagai sektor sudah marak terjadi. Hal ini didukung dengan penetrasi internet yang sudah mencapai 67.1\% dari total populasi dunia per tahun 2024 \parencite{internetPenetration}. Seiring dengan hal ini, perkembangan jumlah pengguna untuk aplikasi tertentu juga meningkat tajam. Pada tahun 2022, Ticketmaster melayani penjualan dua juta tiket konser The Eras Tour Taylor Swift. Proses pelayanan ini tidak berjalan lancar karena sistem Ticket master mengalami \textit{crash} pada saat penjualan tiket \parencite{swiftTicketmaster}. Hal serupa terjadi di India pada saat BookMyShow melayani penjualan tiket Coldplay \parencite{coldplayBookMyShow}. Peristiwa ini cukup menarik apabila dilihat dari sisi teknis. Apakah ada pendekatan alternatif yang bisa diambil untuk menangani jutaan pengguna pada satu waktu tanpa mengalami kegagalan sistem?

Sekarang ini memang sudah banyak sistem yang mampu menangani jutaan pengguna dalam satu waktu, seperti Instagram, Youtube, atau Twitter (sekarang X). Meskipun begitu, kasus ini memiliki karakteristik unik tersendiri. Sebuah sistem \textit{ticketing} harus mampu menangani \textit{concurrent read and write transaction} pada satu waktu dalam jumlah yang sangat besar. Selain itu, sistem \textit{ticketing} harus selalu konsisten dan \textit{real time}, berbeda dengan aplikasi lain seperti Instagram atau Youtube yang mengizinkan \textit{eventual consistency} dan bisa menoleransi \textit{delay}.

Di sisi lain, pendekatan \textit{database} nonrelasional berkembang dengan pesat. Perkembangan ini tertuju pada spesialisasi \textit{database} yang mahir dalam menyelesaikan tugas spesifik. Selain itu, terdapat banyak alternatif model data selain data relasional, seperti \textit{document}, \textit{columnar}, \textit{key-value}, \textit{append-only log}, dan lain-lain. Di antara keseluruhan model data tersebut, \textit{append-only log} merupakan model yang menarik karena hanya mendukung operasi \textit{read} dan \textit{write}, sehingga mengurangi kompleksitas bila dibandingkan dengan model tradisional yang mendukung operasi \textit{write}, \textit{read}, \textit{update}, and \textit{delete}.

Paradigma arsitektur sistem juga sudah berkembang. Selain pola \textit{request-response} tradisional, pendekatan \textit{event-driven} dan \textit{stream processing} juga menjadi populer. Pada pendekatan \textit{stream processing}, sistem melakukan \textit{state maintenance} sambil mendengarkan perubahan data. Ketika terjadi perubahan, sistem akan memperbarui \textit{state} berdasarkan perubahan yang diterima (\textit{incremental update}). Pendekatan ini menarik karena \textit{write-path} dan \textit{read-path} bisa dipisahkan dan \textit{query} tertentu bisa \textit{precomputed}.

Penelitian ini berfokus pada pengembangan alternatif sistem yang mampu melayani \textit{concurrent read and write transaction} dalam skala besar dengan pendekatan \textit{append-only log} dan \textit{stream processing}. \textit{Design pattern} ini memang bukan hal yang baru. Meskipun begitu, pendekatan ini masih perlu diuji apakah mampu menangani kasus yang membutuhkan \textit{high concurrent read} sambil melayani \textit{write transaction} dalam jumlah besar sambil mempertahankan konsistensi dan \textit{fault tolerance}. Pengembangan alternatif sistem ini akan diuji pada kasus \textit{ticketing system}.
