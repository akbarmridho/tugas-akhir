\section{Latar Belakang}
\label{sec:latar-belakang}

Pada tahun 2022, Ticketmaster melayani penjualan dua juta tiket konser The Eras Tour Taylor Swift. Proses pelayanan ini tidak berjalan lancar karena sistem Ticket master mengalami kegagalan pada saat penjualan tiket \parencite{swiftTicketmaster}. Hal serupa terjadi di India pada saat BookMyShow melayani penjualan tiket Coldplay \parencite{coldplayBookMyShow}. Peristiwa ini cukup menarik apabila dilihat dari sisi teknis. Apakah ada pendekatan yang mampu mengoptimalkan sistem ini agar mampu menangani jutaan pengguna pada satu waktu tanpa mengalami kegagalan?

Skalabilitas sistem tiket memiliki karakterisitk yang unik, terutama pada kasus penjualan tiket yang tinggi peminat. \textit{Traffic} sistem ini \textit{bursty}, sehingga membutuhkan sistem yang elastis. Operasi pembacaan ketersediaan tiket harus selalu mengembalikan data terbaru padahal pada saat yang bersamaan terdapat banyak tiket yang dipesan. Penggunaan \textit{caching} tidak akan cocok karena data cepat \textit{stale}, sedangkan \textit{query} langsung ke basis data akan membebani sistem. Selain itu, akan ada banyak penulisan pada relasi data yang sama pada saat yang bersamaan.

Pendekatan untuk mengoptimalkan sistem telah banyak dikembangkan, mulai dari pendekatan \textit{database inside-out}, penggunaan \textit{event-driven system}, pemrosesan \textit{stream}, serta pengoptimalan pada basis data itu sendiri seperti penggunaan \textit{read replica} dan \textit{sharding}. Dari berbagai pendekatan yang ada, tentu tidak semuanya cocok diterapkan pada kasus ini. Oleh karena itu, penelitian ini akan menganalisis berbagai alternatif arsitektur untuk sistem tiket. Setiap alternatif akan diuji untuk mengetahui pendekatan mana yang kinerjanya lebih baik.