\section{Rumusan Masalah}

Saat ini, masih banyak \textit{platform} yang kesulitan dalam menangani penjualan tiket yang tinggi peminat, terutama dengan skala penjualan tiket yang memiliki jutaan peminat. Kegagalan dalam menangani beban ini mengakibatkan terganggunya proses penjualan tiket dan mengakibatkan pengalaman pengguna yang buruk. Bukan tidak mungkin kasus penjualan tiket dengan peminat tinggi akan terjadi lagi di masa mendatang. Oleh karena itu, diperlukan pengoptimalan pada penanganan tiket agar mampu menangani penjualan dengan skala seperti ini.

Rumusan masalah untuk penelitian ini adalah mengeksplorasi pengoptimalan sistem \textit{ticketing} agar mampu menangani penjualan tiket berskala besar dengan menggunakan \textit{design pattern} CQRS (\textit{Command Query Responsibility Segregation}). Terdapat dua alur yang harus dioptimalkan, yaitu operasi \textit{read} dan \textit{write}.

Operasi \textit{read} akan dioptimalkan dengan menggunakan \textit{change data stream} PostgreSQL yang di-\textit{consume} oleh \textit{streaming database} Risingwave. Risingwave ini akan membangun \textit{materialized view} yang diperbarui dengan \textit{subsecond latency} setiap ada perubahan data. Pembacaan melalui Risingwave diharapkan memberikan hasil yang lebih cepat tanpa membebani basis data utama sambil menjaga konsistensi data dengan latensi yang rendah.

Operasi \textit{write} akan dioptimalkan dengan menggunakan \textit{queueing}, \textit{event soucing}, dan \textit{event streaming platform}. Setiap permintaan yang masuk akan diubah ke dalam bentuk \textit{command} berupa \textit{event sourcing}, lalu di-\textit{queue} melalui \textit{event streaming platform} Redpanda. Kemudian, basis data akan memproses \textit{command} yang masuk sesuai dengan kapasitas sistem.

