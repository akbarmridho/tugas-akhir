\section{Batasan Masalah}
\label{sec:batasan-masalah}

Terdapat batasan yang diambil dalam pelaksanaan tugas akhir ini, yaitu sebagai berikut.

\begin{enumerate}
  \item Penlitian ini berfokus pada pengoptimalan operasi \textit{read} dan \textit{write} dengan \textit{design pattern} CQRS pada kasus \textit{ticketing system}.
  \item \textit{Streaming database} yang digunakan adalah Risingwave.
  \item \textit{Event streaming platform} yang digunakan adalah Redpanda.
  \item Pengoptimalan ini berfokus pada \textit{single tenant ticketing system}, yaitu mengoptimalkan \textit{ticketing} untuk satu \textit{event} dengan \textit{traffic} tinggi dan bukan mengoptimalkan \textit{ticketing} untuk banyak \textit{event} pada waktu yang bersamaan.
  \item OLTP \textit{database} yang digunakan adalah PostgreSQL.
  \item Berfokus pada pengoptimalan sistem sehingga mampu menangani besarnya jumlah pengguna dan tidak membahas desain sistem dari sisi pengalaman pengguna.
\end{enumerate}

