\subsection{\textit{Event Sourcing} dan \textit{Change Data Capture}}

Menurut \cite{dataIntensiveApplications}, \textit{change data capture (CDC)} merupakan sebuah proses yang mengobservasi setiap perubahan pada data yang ditulis ke dalam basis data dan mengekstraknya ke dalam bentuk yang bisa direplikasi oleh sistem lain. Sebagai contoh, perubahan pada database bisa di-\textit{capture} lalu diterapkan pada \textit{search index} untuk menyamakan data pada basis data. Apabila \textit{log} diaplikasikan dalam urutan yang sama, data pada \textit{search index} dan basis data bisa dipastikan sama.

Mirip seperti \textit{change data capture}, \textit{event sourcing} juga menyimpan setiap perubahan pada \textit{state} aplikasi sebagai \textit{log of events}. Perbedaan terbesarnya terletak pada level abstraksinya. Pada \textit{change data capture}, aplikasi menggunakan basis data \textit{in a mutable way} dan dapat memperbarui atau menghapus \textit{record} sesuka hati. Aplikasi yang menulis pada basis data tidak harus \textit{aware} bahwa terdapat CDC. Berbeda dengan \textit{event sourcing}, logika aplikasi dibandung secara eksplisit di atas asumsi \textit{immutable event} yang ditulis pada \textit{event log}. Pada kasus ini, \textit{event} berupa \textit{append only}. Singkatnya, \textit{event sourcing} didesain untuk bisa merefleksikan hal yang terjadi pada level aplikasi dan bukan pada \textit{low-level state changes}.