\subsection{Kategori}
Pada penelitian yang dilakukan oleh \parencite{wurster2020essential}, \textit{deployment} dapat dikategorikan menjadi tiga bagian, yaitu \textit{General-Purpose (GP)}, \textit{Provider-Specific (ProvS)}, serta \textit{Platform-Specific (PlatS)}. Ketiga kategori ini berfokus untuk melakukan \textit{deployment} secara otomatis.

\begin{enumerate}
  \item \textit{General-Purpose} (GP)

        Teknologi ini mendukung semua fitur dan mekanisme \textit{deployment} mulai dari \textit{single-cloud}, \textit{hybrid}, \textit{multi-cloud}, serta berbagai jenis \textit{layanan cloud (XaaS)}. Beberapa teknologi yang ada pada kategori ini: Puppet, Chef, Ansible, OpenStack Heat, Terraform, SaltStack, Juju, dan Cloudify.

  \item \textit{Provider-Specific} (ProvS)

        Kategori ini menyediakan fitur untuk membuat \textit{reusable entity}. ProvS hanya mendukung \textit{deployment single-cloud} karena ditawarkan oleh penyedia \textit{cloud} tertentu, sehingga hanya mendukung layanan cloud yang ditawarkan oleh penyedia tersebut. Beberapa teknologi yang ada pada kategori ini: AWS CloudFormation dan Azure Resource Manager.

  \item \textit{Platform-Specific} (PlatS)

        Kategori ini mendukung \textit{multi-cloud} dan \textit{reusable deployment}. Kategori ini dibatasi dalam hal model pengiriman dan penggunaan bundel pada platform tertentu untuk membuat \textit{deployment}. Misalnya, Kubernetes hanya \textit{deployment} dengan \textit{container}. Beberapa teknologi yang ada pada kategori ini: Kubernetes, CFEngine, dan Docker Compose.

\end{enumerate}