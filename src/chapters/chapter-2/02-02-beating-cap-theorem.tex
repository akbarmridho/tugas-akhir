\subsection{\textit{Beating CAP Theorem}}

Cara paling sederhana untuk melakukan \textit{query} adalah dengan menjalankan perhitungan terhadap keseluruhan \textit{dataset}. Dengan pendekatan \textit{append-only log}, \textit{consistency} dan \textit{availability} tetap harus dipilih salah satu. Meskipun begitu, masalahnya hanya sampai di situ saja. Bila memilih \textit{consistency}, akan terdapat masa ketika operasi \textit{read} dan \textit{write} tidak bisa dilakukan. Bila memilih \textit{availability}, sistem akan \textit{eventual consistent}, tetapi tanpa kompleksitas \textit{eventual consistency}. Operasi \textit{read} dan \textit{write} tetap dapat dilakukan. Pada operasi \textit{read}, hasil yang dikembalikan tidak akan memperhitungkan semua data terbaru. Meskipun begitu, \textit{eventually} data akan konsisten. Hal ini terjadi karena sifat data yang \textit{immutable}, sehingga tidak mungkin sebuah replika dari suatu data tidak konsisten.