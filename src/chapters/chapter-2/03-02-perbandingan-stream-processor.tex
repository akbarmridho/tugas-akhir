\subsection{Perbandingan \textit{Stream Processor}}

Terdapat berbagai \textit{streaming framework} yang umum dipakai, yaitu: Apache Storm, Apache Spark, Apache Samza, Apache Flink, dan Apache Kafka Streams. Berikut adalah perbandingannya \parencite{streamProcessingComparison}:

\begin{enumerate}
    \item Apache Storm
          Kelebihan:
          \begin{enumerate}
              \item Latensi rendah, \textit{true streaming}, \textit{mature}, \textit{throughput} tinggi.
              \item Cocok untuk kasus yang tidak rumit.
          \end{enumerate}
          Kekurangan:
          \begin{enumerate}
              \item Tidak ada \textit{state management}.
              \item Tidak ada fitur lanjutan seperti \textit{event time processing}, \textit{aggregation}, \textit{windowing}, dan lain-lain.
              \item \textit{Guarantee at least once}.
          \end{enumerate}
    \item Apache Spark
          Kelebihan:
          \begin{enumerate}
              \item \textit{Throughput} tinggi.
              \item \textit{Fault tolerance} karena penggunaan \textit{micro-batching}.
              \item \textit{Higher level APIs} yang sederhana.
              \item \textit{Exactly once}.
          \end{enumerate}
          Kekurangan:
          \begin{enumerate}
              \item Bukan \textit{true streaming}, sehingga tidak cocok untuk sistem dengan kebutuhan latensi rendah.
              \item \textit{Stateless}.
          \end{enumerate}
    \item Apache Samza
          Kelebihan:
          \begin{enumerate}
              \item Sangat baik dalam menangani \textit{state} yang besar.
              \item \textit{Fault tolerant} dengan menggunakan Kafka.
              \item \textit{low latency}, \textit{high throughput}, dan \textit{matured}.
          \end{enumerate}
          Kekurangan:
          \begin{enumerate}
              \item Terhubung secara erat dengan Kafka dan Yarn.
              \item \textit{At least once}.
              \item Tidak ada fitur \textit{streaming} lanjutan seperti \textit{watermarks}, \textit{sessions}, \textit{triggers}, dan lain-lain.
          \end{enumerate}
    \item Apache Flink
          Kelebihan:
          \begin{enumerate}
              \item \textit{True streaming framework} dengan fitur lanjutan yang lengkap, seperti \textit{event time processing}, \textit{watermarks}, dan lain-lain.
              \item \textit{low latency} dan \textit{high throughput}.
              \item \textit{Auto-adjusting} sehingga tidak banyak \textit{parameter} yang harus diatur.
          \end{enumerate}
          Kekurangan:
          \begin{enumerate}
              \item Masih agak baru.
              \item Komunitas belum sebesar Spark atau \textit{streaming framework} lainnya.
          \end{enumerate}
    \item Apache Kafka Streams
          Kelebihan:
          \begin{enumerate}
              \item Ringan sehingga cocok untuk \textit{microservices} atau aplikasi IOT.
              \item Tidak membutuhkan \textit{dedicated cluster}.
              \item Mendukung \textit{stream joins}.
              \item \textit{Exactly once}.
          \end{enumerate}
          Kekurangan:
          \begin{enumerate}
              \item Terhubung dengan erat dengan Kafka.
              \item Belum \textit{battle proven} karena masih baru.
              \item Bukan untuk pekerjaan yang berat, seperti Spark atau Flink.
          \end{enumerate}
\end{enumerate}

Kasus ini membutuhkan \textit{stream processing framework} dengan karakteristik \textit{stateful}, \textit{true streaming} dan \textit{exactly once}. Berdasarkan kebutuhan tersebut, Apache Kafka Streams dan Apache Flink merupakan kandidat yang paling baik. Meskipun begitu, Apache Flink akhirnya dipilih karena fiturnya yang lebih lengkap.