\subsection{\textit{Event Sourcing}}

Mirip seperti \textit{change data capture}, \textit{event sourcing} juga menyimpan setiap perubahan pada \textit{state} aplikasi sebagai \textit{log of events}. Perbedaan terbesarnya terletak pada level abstraksinya. Pada \textit{change data capture}, aplikasi menggunakan basis data \textit{in a mutable way} dan dapat memperbarui atau menghapus \textit{record} sesuka hati. Aplikasi yang menulis pada basis data tidak harus \textit{aware} bahwa terdapat CDC. Berbeda dengan \textit{event sourcing}, logika aplikasi dibangun secara eksplisit di atas asumsi \textit{immutable event} yang ditulis pada \textit{event log}. Pada kasus ini, \textit{event} berupa \textit{append only}. Singkatnya, \textit{event sourcing} didesain untuk bisa merefleksikan hal yang terjadi pada level aplikasi dan bukan pada \textit{low-level state changes} \parencite{dataIntensiveApplications}.

Sebagai contoh, \textit{event} yang berisi "seorang murid membatalkan keikutsertaannya dalam suatu kelas" menyampaikan \textit{intent} dari sebuah aksi (\textit{event sourcing}), sedangkan \textit{event} yang berisi "sebuah entri dihapus dari tabel kepesertaan mata kuliah dan satu alasan pembatalan ditambahkan pada tabel umpan balik mahasiswa" tidak memberikan informasi yang jelas atas hal apa yang terjadi (\textit{change data capture}).