\subsection{\textit{Event Sourcing}}

Mirip seperti CDC, \textit{event sourcing} juga menyimpan setiap perubahan pada keadaan aplikasi sebagai \textit{log of events}. Perbedaan terbesarnya terletak pada level abstraksinya. Pada CDC, aplikasi menggunakan basis data \textit{in a mutable way} dan dapat memperbarui atau menghapus catatan sesuka hati. Aplikasi yang menulis pada basis data tidak harus menyadari bahwa terdapat CDC. Berbeda dengan \textit{event sourcing}, logika aplikasi dibangun secara eksplisit di atas asumsi \textit{immutable event} yang ditulis pada \textit{event log}. Pada kasus ini, \textit{event} bersifat \textit{append only} (hanya bisa dibaca dan ditambahkan, tidak bisa diperbarui atau dihapus). Singkatnya, \textit{event sourcing} didesain agar bisa merefleksikan hal yang terjadi pada level aplikasi dan bukan pada \textit{low-level state changes} \parencite{dataIntensiveApplications}.

Sebagai contoh, \textit{event} yang berisi "seorang murid membatalkan keikutsertaannya dalam suatu kelas" menyampaikan maksud dari sebuah aksi (\textit{event sourcing}), sedangkan \textit{event} yang berisi "sebuah entri dihapus dari tabel kepesertaan mata kuliah dan satu alasan pembatalan ditambahkan pada tabel umpan balik mahasiswa" tidak memberikan informasi yang jelas atas hal apa yang terjadi (CDC).