\subsection{\textit{Rethinking Data System}}

Aplikasi data terdiri atas berbagai hal, mulai dari penyimpanan data, pengambilan data, \textit{joins}, agregasi, \textit{stream processing}, dan lain-lain. Meskipun begitu, hal tersebut bisa disederhanakan menjadi hal berikut.

\[Query = Function(All Data)\]

Sebuah \textit{data system} menjawab pertanyaan atas sebuah \textit{dataset}. Pertanyaan tersebut dinamakan sebagai \textit{queries}. Persamaan di atas menyatakan bahwa \textit{query} merupakan fungsi pada keseluruhan data yang dimiliki.

Menurut \cite{howToBeatCAP}, \textit{A piece of data is an indivisible unit that you hold to be true for no other reason that it exists}. Terdapat dua \textit{properties} yang dimiliki data. Pertama, data terikat pada waktu. Sebuah data adalah fakta yang diketahui benar pada satu titik waktu tertentu. Kedua, data pada dasarnya \textit{immutable}. Karena hubungannya dengan waktu, kebenaran sebuah data tidak berubah. Misalkan, Budi tinggal di Bandung. Setelahnya, Budi tinggal di Jakarta. Data terbaru yang menyatakan bahwa Budi tinggal di Jakarta tidak lantas membuat fakta bahwa Budi (pernah) tinggal di Bandung salah.

Operasi \textit{update} bisa dihilangkan karena tidak logis dalam konteks \textit{immutable data}. Misalnya, men-\textit{update} lokasi Budi berarti bahwa kita menambahkan data lokasi Budi yang lebih baru.

Operasi \textit{delete} bisa dihilangkan karena operasi ini juga dapat direpresentasikan dengan data baru. Misalkan, bila Upin berhenti mengikuti Ipin di sosial media, hal tersebut tidak mengubah fakta bahwa Upin pernah mengikuti Ipin. Hanya saja, Upin berhenti mengikuti Ipin pada titik waktu tertentu.

Konsep kedua adalah \textit{query}. \textit{Query} adalah turunan dari data. Misalkan pertanyaan "Di mana lokasi Budi sekarang?" merupakan sebuah \textit{query}. \textit{Query} ini dapat diturunkan dengan mengembalikan data terakhir yang berkaitan dengan lokasi Budi.