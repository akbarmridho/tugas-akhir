\subsection{\textit{Enhancing Resilience and Scalability in Travel Booking Systems: A Microservices Approach to Fault Tolerance, Load Balancing, and Service Discovery}}

Penelitian ini membahas penggunaan arsitektur \textit{microservice} pada pengembangan sistem reservasi tiket pesawat dalam aspek skalabilitas dan keandalan. Penelitian ini menggunakan pola \textit{circuit breaker} untuk menjaga toleransi kegagalan saat mengonsumsi layanan eksternal seperti \textit{flight} API dan sistem pembayaran. Pendekatan \textit{microservice} unggul dalam aspek ketahanan dan skalabilitas karena layanan tidak tergantung satu sama lain dan bisa di-\textit{deploy} secara independen. Berdasarkan hasil pengujian, penggunaan \textit{round robin load balancing} meningkatkan kinerja sebesar 35\%. Selain itu, penggunaan arsitektur \textit{microservices} memungkinkan layanan pemesanan melayani 40\% lebih banyak individu dan mengurangi \textit{down time} sistem sebesar 50\% \parencite{barua2024enhancingresiliencescalabilitytravel}.

Berikut adalah komponen yang menjadi bahasan pada penelitian:

\begin{enumerate}
    \item Layanan pencarian penerbangan. Layanan ini memanggil API layanan penerbangan eksternal.
    \item Layanan pemesanan. Layanan ini berinteraksi dengan layanan lain seperti pembayaran dan notifikasi.
    \item Layanan pembayaran. Layanan ini mengatur transfer dana/ pembayaran melalui berbagai vendor pemrosesan.
    \item Layanan notifikasi. Layanan ini mengirimkan pemberitahuan pemesanan sebagai surat elektronik atau pesan singkat.
\end{enumerate}

Hal yang menarik dari penelitian ini adalah penggunaan pola \textit{circuit breaker} dan \textit{load balancing}. Pola \textit{circuit breaker} diimplementasikan dengan penggunaan \textit{healthcheck} yang dilakukan secara aktif. Selain itu, \textit{load balancer} mencatat banyaknya permintaan yang gagal untuk \textit{instance} tertentu. Apabila jumlah kegagalan mencapai batas tertentu, \textit{instance} tersebut tidak akan diberikan \textit{request} untuk sementara. Dengan demikian, kegagalan dapat dideteksi lebih awal dan \textit{instance} yang mengalami kegagalan dapat melakukan \textit{recovery} tanpa terbebani dengan \textit{request}. Peningkatan \textit{throughput} dicapai dengan menurunkan \textit{downtime} dan melakukan \textit{routing request} kepada \textit{instance} yang berjalan dengan lancar. Berdasarkan uraian di atas, penelitian ini meningkatkan kinerja dan \textit{throughput} dengan meminimalkan \textit{downtime}. Peningkatan skalabilitas yang dibahas pada penelitian ini berbeda dengan fokus pada tugas akhir ini yang meningkatkan skalabilitas dengan menggunakan arsitektur dan \textit{tools} yang berbeda. Meskipun begitu, penelitian ini menarik untuk dijadikan bahan pertimbangan dalam mendesain arsitektur solusi.